\textbf{Instructions on the use of \cppclass{ode45}, see \lref{cpp:ode45}.}

The class \cppclass{ode45} is header-only, meaning you just include the file and use it
right away (no linking required). The file \cppfile{ode45.hpp} defines the class
\cppclass{ode45} implementing an embedded Runge-Kutta-Fehlberg method of order 4(5), see
\lref{par:embrk} and \lref{ex:embrk}, with an adaptive stepsize control as presented in
\lref{rem:refssctrl}. 

\begin{enumerate}
\item Construct an object of \cppclass{ode45} type:
  create an instance of the class, passing the right-hand-side function $\Vf$ as a
  functor object to the constructor
  \begin{lstlisting}[style=cppsimple,emph={ode45},emph={[2]StateType,RhsType}]
    template <class StateType,
    class RhsType = std::function<StateType(const StateType &)>>
    class ode45 {
      public:
      ode45(const RhsType &rhs);
      // ......
    }
  \end{lstlisting}

  Template parameters are
  \begin{itemize}
  \item \cppclass{StateType}: type of initial data and solution (state space), the only requirement
    is that the type possesses a normed vector-space structure, that is, it must
    implement the operations \verb|+|, \verb|*|, \verb|*=|, \verb|+=| and
    assignment/copy operators. Moreover a \cppfn{norm()} method must be available.
    \eigen's vector and matrix types, as well as fundamental types are eligible as
    \cppclass{StateType}.
  \item \cppclass{RhsType}:    type of rhs function (automatically deduced).
  \end{itemize}
  The argument \cppcode{rhs} must be of a functor type that provides an evaluation operator
  \begin{lstlisting}[style=cppsimple,emph={[2]StateType}]
    StateType operator()(const StateType & vec);
  \end{lstlisting}
  It can also be a lambda function. 
\item (optional) Set the integration options: set data members of the data structure
  \cppclass{ode45.options} to configure the solver:
  \begin{lstlisting}[style=cppsimple]
    O.options.<option_you_want_to_set> = <value>;
  \end{lstlisting}
  Examples:
  \begin{itemize}
  \item \texttt{rtol}:       relative tolerance for error control (default is \texttt{10e-6})
  \item \texttt{atol}:       absolute tolerance for error control (default is \texttt{10e-8})
  \end{itemize}
  e.g.:
  \begin{lstlisting}[style=cppsimple]
    O.options.rtol = 10e-5;
  \end{lstlisting}

\item Solve stage: invoke the single-step method through calling the method
  \begin{lstlisting}[emph={solve},style=cppsimple]
    template <class NormFunc = decltype(_norm<StateType>)>
    std::vector<std::pair<StateType, double>>
    solve(const StateType &y0, double T,
    const NormFunc &norm = _norm<StateType>);
  \end{lstlisting}
  The type \cppclass{NormType} should provide a norm for vectors of type \cppclass{StateType}.
  However, this type can be deduced automatically and the argument \cppcode{norm} is optional.
  The other arguments are 
  \begin{itemize}
  \item \texttt{y0}: initial value of type \cppclass{StateType} (\cob{$\Vy_{0}=$}
    \texttt{y0})
  \item \texttt{T}:
    final time of integration
  \item \texttt{norm}: (optional) norm function to call for objects of
    \cppclass{StateType}, for the computation of the error
  \end{itemize}
  \textbf{Return value} The function returns the solution of the IVP, as a
  \texttt{std::vector} of \texttt{std::pair} \cob{$(y(t), t)$} for every snapshot.
\end{enumerate}



For more explanations and details, please consult the in-class documentation provided in
the comments.
