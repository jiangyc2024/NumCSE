\textbf{Instructions on the usage of \texttt{ode45}.}

The class \texttt{ode45} is header-only, meaning you just include the file and use it right away (no linking required). 
The file \texttt{ode45.hpp} defines class \texttt{ode45} 
implementing an emulation of Matlab's Rosenbrock method of order 4(3): \texttt{ode45}.

\begin{enumerate}
\item Construct an object of \texttt{ode45} type:
    create an instance of the class, passing the r.h.s. function $f$ to constructor:
    \begin{lstlisting}
ode45<StateType> = O(f);
    \end{lstlisting}
    Template parameters:
    \begin{itemize}
        \item StateType:  type of initial data and solution (state space), the only
        requirement is that the tpye possesses a normed vector-space structure;
        \item RhsType:    type of rhs function (automatically deduced).
    \end{itemize}
    The function \texttt{f} must be a function handle with 
    \begin{lstlisting}
operator()(const StateType & vec) -> StateType
    \end{lstlisting}
    
\item (optional) Set the integration options: set members of struct \texttt{ode45.options} to configure the solver:
    \begin{lstlisting}
O.options.<option_you_want_to_set> = <value>
    \end{lstlisting}
    Examples:
    \begin{itemize}
        \item \texttt{rtol}:       relative tolerance for error control (default is \texttt{10e-6})
        \item \texttt{atol}:       absolute tolerance for error control (default is \texttt{10e-8})
    \end{itemize}
    e.g.:
    \begin{lstlisting}
O.options.rtol = 10e-5;
    \end{lstlisting}

\item Solve stage: call the solver:
    \begin{lstlisting}
std::vectro<std::pair<StateType> sol = O.solve(y0, T, norm)
    \end{lstlisting}

    Template parameters:
    \begin{itemize}
        \item \texttt{NormType}: type of norm function, automatically deduced
    \end{itemize}

    Arguments:
    \begin{itemize}
        \item \texttt{y0}:         initial value in StateType ($y(0) =$ \texttt{y0})
        \item \texttt{T}:          final time of integration
        \item \texttt{norm}:       (optional) norm function to call on member of StateType, for the computation of the error
    \end{itemize}

    The function returns the solution of the IVP, as a \texttt{std::vector} 
    of \texttt{std::pair} $(y(t), t)$ for every snapshot.
\end{enumerate}

For more documentation, consult the in-class documentation or the file \texttt{NumCSE/Utils/README.md}.
