\renewcommand{\chpt}{ch_matvec}

\begin{problem}[Structured matrix--vector product]
  \label{prb:StructuredMatrixVector}

In \lref{ex:lrtrisum} we saw how the particular structure of a matrix can be
exploited to compute a matrix-vector product with substantially reduced
computational effort. This problem presents a similar case.

Consider the real $n\times n$ matrix $\VA$ defined by $(\VA)_{i,j}=a_{i,j}=\min\{i,j\}$, for $i,j=1,\ldots,n$.
The matrix-vector product $\Vy=\VA\Vx$ can be implemented in \Matlab~ as
\begin{gather} \label{eq:structAx}
  \mathtt{ y = min(ones(n,1)*(1:n), (1:n)'*ones(1,n)) * x;}
\end{gather}

%%%%%%%%%%%% SUBPROBLEM 1
\begin{subproblem}[1] \label{subprb:StructuredMatrixVector_1}
What is the asymptotic complexity (for $n\to\infty$) of the evaluation of the \Matlab~ command displayed above, with respect to the problem size parameter $n$?

\begin{solution}
Matrix--vector multiplication:  quadratic dependence $O(n^2)$.
\end{solution}

\end{subproblem}

%%%%%%%%%%%% SUBPROBLEM 2
\begin{subproblem}[3] \label{subprb:StructuredMatrixVector_2}
%
Write an \emph{efficient} \Matlab~ function
%
\begin{center}
\texttt{function y = multAmin(x)}
\end{center}
%
that computes the same multiplication as \eqref{eq:structAx} but with a better asymptotic complexity with respect to {$n$}.

\begin{hint}
you can test your implementation by comparing the returned values with the ones obtained with code \eqref{eq:structAx}.
\end{hint}

\begin{solution}
%
For every $j$ we have $y_j = \sum_{k=1}^j k x_k + j \sum_{k=j+1}^n x_k$, so we pre-compute the two terms for every $j$ only once.
%
\lstinputlisting[caption={implementation for the function multAmin},label={mc:StructuredMatrixVector_multAmin}]
{\problems/\chpt/MATLAB/multAmin.m}
\end{solution}

\end{subproblem}

%%%%%%%%%%%% SUBPROBLEM 3
\begin{subproblem}[1] \label{subprb:StructuredMatrixVector_3}
What is the asymptotic complexity (in terms of problem size parameter $n$) of your function \texttt{multAmin}?

\begin{solution}
Linear dependence: $O(n)$.
\end{solution}
%
\end{subproblem}

%%%%%%%%%%%% SUBPROBLEM 4

\begin{subproblem}[2] \label{subprb:StructuredMatrixVector_4}
Compare the runtime of your implementation and the implementation given in \eqref{eq:structAx} for $n=2^{5,6,\ldots,12}$.
Use the routines \texttt{tic} and \texttt{toc} as explained in example \ncseex{ex:effmatmult} of the Lecture Slides.

\begin{solution}

The matrix multiplication in \eqref{eq:structAx} has runtimes growing with $O(n^2)$.
The runtimes of the more efficient implementation with hand-coded loops, or using the \Matlab function \texttt{cumsum} are growing with $O(n)$.
%
\lstinputlisting[caption={comparison of execution timings}, label={mc:StructuredMatrixVector}]
{\problems/\chpt/MATLAB/multAmin_timings.m}
%
% \begin{figure}[ht]
% \centering
% \label{fig:multAmin_timings}
% \includegraphics[width=0.8\textwidth]{\problems/\chpt/PICTURES/multAmin_timings.eps}
% \caption{Timings for different implementations of $\Vy=\VA\Vx$}
% \end{figure}
%
% \filippo{Decide wether to include next pic or not}

\begin{figure}[ht]
\centering
\label{fig:multAmin_timings}
\includegraphics[width=0.8\textwidth]{\problems/\chpt/PICTURES/multAmin_timings_cpp.eps}
\caption{Timings for different implementations of $\Vy=\VA\Vx$ with both \Matlab and \Cpp.}
\end{figure}

\end{solution}
\end{subproblem}

%%%%%%%%%%%% SUBPROBLEM 5
\begin{subproblem}[3] \label{subprb:StructuredMatrixVector_5}
Can you solve task \ref{subprb:StructuredMatrixVector_2} without using any \texttt{for}- or \texttt{while}-loop?\\
Implement it in the function
%
\begin{center}
\texttt{function y = multAmin2(x)}
\end{center}
%
\begin{hint}
 you may use the \Matlab built-in command \texttt{cumsum}.
\end{hint}

\begin{solution}
Using \texttt{cumsum} to avoid the \texttt{for} loops:
\lstinputlisting[caption={implementation for the function multAmin without loops},label={mc:StructuredMatrixVector_multAmin2}]
{\problems/\chpt/MATLAB/multAmin2.m}
\end{solution}
\end{subproblem}

%%%%%%%%%%%% SUBPROBLEM 6
\begin{subproblem}[1]  \label{subprb:StructuredMatrixVector_6}
 Consider the following \Matlab script \texttt{multAB.m}:
%
\lstinputlisting[caption={\Matlab script calling \texttt{multAmin}},label={mc:StructuredMatrixVector_multAB}]
{\problems/\chpt/MATLAB/multAB.m}
%
Sketch the matrix $\VB$ created in line 3 of \texttt{multAB.m}.

\begin{hint}
this \Matlab script is provided as file \texttt{multAB.m}.
\end{hint}

\begin{solution}
$$\VB:= \begin{pmatrix}
 2 & -1 &  0 & \cdots & 0 \\
-1 &  2 & -1 & \ddots & \vdots \\
 0 & \ddots & \ddots & \ddots & 0\\
\vdots & \ddots& -1 & 2 & -1\\
 0 & \cdots & 0 & -1 & 1
\end{pmatrix}$$
%
Notice the value 1 in the entry $(n,n)$.
\end{solution}
\end{subproblem}

%%%%%%%%%%%% SUBPROBLEM 7
\begin{subproblem}[2] \label{subprb:StructuredMatrixVector_7}
Run the code of
Listing~\ref{mc:StructuredMatrixVector_multAB} several times and conjecture a
relationship between the matrices $\VA$ and $\VB$ from the output. Prove your
conjecture.

\begin{hint}
You must take into account that computers inevitably commit round-off errors, see
\ncsesect{sec:marith}.
\end{hint}

\begin{solution}
It is easy to verify with \Matlab (or to prove) that $\VB=\VA^{-1}$.\\
For $2 \leq j \leq n-1$, we obtain:
%
\begin{equation*}
\begin{aligned}
(\VA\VB)_{i,j} &= \sum_{k=1}^n a_{i,k} b_{k,j}
= a_{i,j-1} b_{j-1,j} + 2 a_{i,j} b_{j,j} + a_{i,j+1} b_{j+1,j} \\
&= -\min(i,j-1) + 2 \min(i,j) - \min(i,j+1)
= \begin{cases}
-i + 2i - i = 0 & \text{ if } i<j, \\
-(i-1) + 2i - i = 1 & \text{ if } i=j, \\
-(j-1) + 2j - (j+1) = 0 & \text{ if } i>j.
\end{cases}
\end{aligned}
\end{equation*}
%
Furthermore, $(\VA\VB)_{i,1} = \delta_{i,1}$ and $(\VA\VB)_{i,n} = \delta_{i,n}$, hence $\VA\VB = {\bf I}$.\\
The last line of \texttt{multAB.m} prints the value of $\|\VA\,\VB\,\Vx-\Vx\|=\|\Vx-\Vx\|=0$.\\
The returned values are not exactly zero due to round-off errors.

\end{solution}
\end{subproblem}

\begin{subproblem}[2] \label{sp:smv:cpp1}
  Implement a \Cpp{} function with declaration
\begin{lstlisting}
template <class Vector>
void minmatmv(const Vector &x,Vector &y);
\end{lstlisting}
that realizes the efficient version of the \matlab{} line of
code \eqref{eq:structAx}. Test your function by comparing
with output from the equivalent \matlab{} functions.

  \begin{solution}

 \lstinputlisting[caption={\Cpp script implementing \texttt{multAmin}},label={cppcc:StructuredMatrixVector_multAB}]
{\problems/\chpt/CPP/multAmin.cpp}

  \end{solution}

\end{subproblem}

\end{problem}

%Several runs of \texttt{multAB} from the previous sub-problem
%produced the following output:
%\begin{verbatim}
%>> multAB
%|x-y| = 3.587240e-15
%>> multAB
%|x-y| = 1.864381e-15
%>> multAB
%|x-y| = 5.324443e-16
%\end{verbatim}
%Explain these results.
%
%\hint: what is the relationship of the matrices $\VA$ and $\VB$?
%A fully rigorous proof is not required here.
