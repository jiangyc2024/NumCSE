\renewcommand{\ProblemName}{Kronecker}

\begin{samproblem}*
  {prb:Kron}
  {Kronecker product}[1](1)
  {
    In \lref{def:kron} we learned about the so-called Kronecker product.
    In this problem we revisit the
    discussion of \lref{ex:kron}. Please refresh yourself on this example and
    study \lref{mc:kronmultv} again.
  }

%%%%%%%%%%%%%%%%%%%%%%%%%%%%%%%%%%%%%%%%
%%%%%%%%%%%% SUBPROBLEM 1: Compute kron
%%%%%%%%%%%%%%%%%%%%%%%%%%%%%%%%%%%%%%%%
\begin{subproblem}{subprb:Kron_2}
  Compute the kronecker product
  $\mathbf{C} = \mathbf{A} \otimes \mathbf{B}$ of the matrices
\begin{align*}
  \mathbf{A} = \begin{pmatrix}
         1 & 2   \\
         3 & 4   \\
       \end{pmatrix}
\end{align*}
and
\begin{align*}
\mathbf{B}=\begin{pmatrix}
         5 & 6   \\
         7 & 8   \\
       \end{pmatrix}.
\end{align*}

% ==================== BEGIN SOLUTION
\begin{samwriteprbpart}{cmpm}
  \begin{writeverbatim}{prbfile}
    \begin{samsolution}
      \begin{align*}
\Vy=\begin{pmatrix}
         5 & 6 & 10 & 12  \\
         7 & 8 & 14 & 16  \\
         15 & 18 & 20 & 24  \\
         21 & 24 & 28 & 32  \\
       \end{pmatrix}\Vx.
        \end{align*}
      \end{samsolution}
    \end{writeverbatim}
  \end{samwriteprbpart}
% ==================== END SOLUTION
\end{subproblem}

%%%%%%%%%%%%%%%%%%%%%%%%%%%%%%%%%%%%%%%%
%%%%%%%%%%%% SUBPROBLEM 2: Complexity
%%%%%%%%%%%%%%%%%%%%%%%%%%%%%%%%%%%%%%%%
\begin{subproblem}{subprb:Kron_3}
What is the asymptotic complexity ($\to$ \lref{def:comp})
of the \Matlab~ code \eqref{eq:ex_Kron}? Use the Landau symbol
from \lref{def:O} to state your answer.

% ==================== BEGIN SOLUTION
\begin{samwriteprbpart}{cmpk}
  \begin{writeverbatim}{prbfile}
    \begin{samsolution}
      \texttt{kron(A,B)} results in a matrix of size $n^2 \times n^2$ and $x$
      has length $n^2$. So the complexity is the same as a matrix-vector
      multiplication for the resulting sizes. In total this is $O(n^2*n^2)=O(n^4)$.
      \end{samsolution}
    \end{writeverbatim}
  \end{samwriteprbpart}
% ==================== END SOLUTION
\end{subproblem}

%%%%%%%%%%%%%%%%%%%%%%%%%%%%%%%%%%%%%%%%
%%%%%%%%%%%% SUBPROBLEM 3: Implement
%%%%%%%%%%%%%%%%%%%%%%%%%%%%%%%%%%%%%%%%
\begin{subproblem}{subprb:Kron_6}
Implement a C++ function
\begin{lstlisting}[style=cppsimple]
void kron(const MatrixXd & A, const MatrixXd & B,
          MatrixXd & C);
\end{lstlisting}
that computes the Kronecker product of the argument matrices
\texttt{A} and \texttt{B} and stores the result in the matrix \texttt{C}.

% ==================== BEGIN SOLUTION
\begin{samwriteprbpart}{cmpk}
  \begin{writeverbatim}{prbfile}
    \begin{samsolution}
      See \texttt{kron.cpp} or \cref{cppc:mainKron}.
    \end{samsolution}
  \end{writeverbatim}
\end{samwriteprbpart}
% ==================== END SOLUTION
\end{subproblem}

%%%%%%%%%%%%%%%%%%%%%%%%%%%%%%%%%%%%%%%%
%%%%%%%%%%%% SUBPROBLEM 3: Implement
%%%%%%%%%%%%%%%%%%%%%%%%%%%%%%%%%%%%%%%%
\begin{subproblem}{subprb:Kron_7}
  Devise an implementation of the \matlab{} code \eqref{eq:ex_Kron}
  in C++ according
  to the function definition
  \begin{lstlisting}[style=cppsimple]
void kron_mv(const MatrixXd & A, const MatrixXd & B,
             const VectorXd & x, VectorXd & y);
  \end{lstlisting}
  The meaning of the arguments should be self-explanatory.

% ==================== BEGIN SOLUTION
\begin{samwriteprbpart}{cmpk}
  \begin{writeverbatim}{prbfile}
    \begin{samsolution}
    See \texttt{kron.cpp} or \cref{cppc:mainKron}.
    \end{samsolution}
  \end{writeverbatim}
\end{samwriteprbpart}
% ==================== END SOLUTION
\end{subproblem}

%%%%%%%%%%%%%%%%%%%%%%%%%%%%%%%%%%%%%%%%
%%%%%%%%%%%% SUBPROBLEM 3: Implement one liner problem
%%%%%%%%%%%%%%%%%%%%%%%%%%%%%%%%%%%%%%%%
\begin{subproblem}{subprb:kron:8}
  Now, using a function definition similar to that of the previous sub-problem,
  implement the C++ equivalent of \eqref{eq:Kron:sl} in the function
  \texttt{kron\_mv\_fast}.
  Study \lref{rem:eigrs} about ``reshaping'' matrices in \eigen{}.
\end{subproblem}

%%%%%%%%%%%%%%%%%%%%%%%%%%%%%%%%%%%%%%%%
%%%%%%%%%%%% SUBPROBLEM 3: Conmpute runtime
%%%%%%%%%%%%%%%%%%%%%%%%%%%%%%%%%%%%%
\begin{subproblem}{subprb:kron:9}
  Compare the runtimes of your two implementations as you did for the \matlab{}
  implementations in sub-problem \ref{subprb:Kron_5}.

% ==================== BEGIN SOLUTION
\begin{samwriteprbpart}{cmpk}
  \begin{writeverbatim}{prbfile}
    \begin{samsolution}
      \begin{samcode}[C++11-code]{}{}
        \samincludecpp{\codes/MatVec/Kronecker/solutions/kron.cpp}
      \end{samcode}

      % \samplot{\codes/MatVec/Kronecker/kron_timings_cpp.eps}
              % {fig:kron_timings}
    \end{samsolution}
  \end{writeverbatim}
\end{samwriteprbpart}
% ==================== END SOLUTION
\end{subproblem}

\end{samproblem}
