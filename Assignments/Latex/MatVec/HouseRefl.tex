\begin{problem}*
  {prb:HouseRefl}
  {Householder reflections}
  {
    This problem is a
    supplement to \lref{sec:gs} and is related to Gram-Schmidt
    orthogonalization, see \lref{gramschmidt}.
    Before starting this problem, please make sure that you
    remember and understand the notion of a QR-decomposition
    of a matrix, see \lref{thm:qr}. This problem will put to
    the test your advanced linear algebra skills.
  }

  %%%%%%%%%%%%%%%%%%%%%%%%%%%%%%%%%%%%%%%%
  %%%%%%%%%%%% SUBPROBLEM 1: Compute kron
  %%%%%%%%%%%%%%%%%%%%%%%%%%%%%%%%%%%%%%%%
  \begin{subproblem}{subprb:HouseRefl_0}[1]
    \cref{mc:HouseRefl} contains preudo-code of a particular
    algorithm.
  \begin{lstlisting}

  \end{lstlisting}
  Write a \Cpp~ function with declaration:
  \begin{lstlisting}
 void houserefl(const VectorXd &v, MatrixXd &Z);
  \end{lstlisting}
  that implements \ref{}.
  Use data types from \eigen{}.

  % BEGIN ================ SOLUTION  ================   %
  \begin{samwriteprbpart}{impl}
    \begin{writeverbatim}{prbfile}
      \begin{samsolution}
        \begin{samcode}[C++-code]{cppc:HouseRefl}
          {C++ implementation of \ref{prb:HouseRefl}}
          \samincludecpp
          {\codes/MatVec/Houserefl/solutions/housrerefl.cpp}[1]
          \gitlabSolF{MatVec}{HouseRefl}{houserefl.cpp}
        \end{samcode}
      \end{samsolution}
    \end{writeverbatim}
  \end{samwriteprbpart}
  % END ================ SOLUTION  ================   %
  \end{subproblem}

  %%%%%%%%%%%%%%%%%%%%%%%%%%%%%%%%%%%%%%%%
  %%%%%%%%%%%% SUBPROBLEM 1: Compute kron
  %%%%%%%%%%%%%%%%%%%%%%%%%%%%%%%%%%%%%%%%
  \begin{subproblem}{subprb:HouseRefl_a}[3]
  Show that the matrix $\vec{X}$,  defined at line 10 in
  \autoref{mc:HouseRefl}, satisfies:
  \[
    \vec{X}^\top \vec{X} = \vec{I}_n
  \]
  \begin{samwriteprbpart}{impl}
    \begin{writeverbatim}{prbfile
        \begin{samhint}
          Notice that $\norm{\vec{q}}^2 = 1$.
        \end{samhint}
    \end{writeverbatim}
  \end{samwriteprbpart}

  % BEGIN ================ SOLUTION  ================   %
  \begin{samwriteprbpart}{impl}
    \begin{writeverbatim}{prbfile}
      \begin{samsolution}
        \begin{align*}
          \vec{X}^\top \vec{X} = (\vec{I}_n - 2 \vec{q}\vec{q}^\top)(\vec{I}_n - 2 \vec{q}\vec{q}^\top) \\
   = \vec{I}_n - 4 \vec{q}\vec{q}^\top + 4 \vec{q} \underbrace{\vec{q}^\top\vec{q}}_{=\norm{\vec{q}}=1}\vec{q}^\top \\
   = \vec{I}_n - 4 \vec{q}\vec{q}^\top + 4 \vec{q}\vec{q}^\top \\
   = \vec{I}_n
        \end{align*}
      \end{samsolution}
    \end{writeverbatim}
  \end{samwriteprbpart}
  % END ================ SOLUTION  ================   %
\end{subproblem}

%%%%%%%%%%%%%%%%%%%%%%%%%%%%%%%%%%%%%%%%
%%%%%%%%%%%% SUBPROBLEM 1: Compute kron
%%%%%%%%%%%%%%%%%%%%%%%%%%%%%%%%%%%%%%%%
\begin{subproblem}{subprb:HouseRefl_b}{4}
  Show that the first column of $\vec{X}$, after line 9 of the
  function \func{houserefl}, is a multiple of the vector $\vec{v}$.

  \begin{samwriteprbpart}{impl}
    \begin{writeverbatim}{prbfile
        \begin{samhint}
          Use the previous hint, and the facts that
          \begin{align*}
            \vec{u} = \vec{w} + \begin{bmatrix}
              1 \\
              0 \\
              \vdots \\
              0
            \end{bmatrix}
          \end{align*}
          and $\norm{\vec{w}} = 1$.
      \end{samhint}
    \end{writeverbatim}
  \end{samwriteprbpart}


  % BEGIN ================ SOLUTION  ================   %
  \begin{samwriteprbpart}{impl}
    \begin{writeverbatim}{prbfile}
      \begin{samsolution}
        Let $\vec{X} = [\vec{X}_1, \cdots, \vec{X}_n]$ be the matrix
        of line 9 in \autoref{mc:HouseRefl}. In view of the identity
        $\vec{X}_1 = \;  \vec{e}^{(1)} - 2 q_1 \vec{q} $ we have
        \[
          \vec{X}_1 = \begin{bmatrix}
            1 -2q_1^2 \\ -2q_1q_2 \\ \vdots \\ -2q_1q_n
          \end{bmatrix}
          = \begin{bmatrix}
            1 - 2 \frac{u_1^2}{\sum_{i = 1}^n u_i^2} \\
            - 2 \frac{u_1 u_2}{\sum_{i = 1}^n u_i^2} \\
            \vdots \\
            - 2 \frac{u_1 u_n}{\sum_{i = 1}^n u_i^2}
          \end{bmatrix}
          \overset{\textsc{Hint}}{=} \begin{bmatrix}
            \frac{(w_1+1)^2 + w_2^2 + \cdots + w_n^2 - 2(w_1+1)^2}{(w_1+1)^2 + w_2^2 + \cdots + w_n^2} \\
            - \frac{2(w_1+1)w_2}{(w_1+1)^2 + w_2^2 + \cdots + w_n^2} \\
            \cdots \\
            - \frac{2(w_1+1)w_n}{(w_1+1)^2 + w_2^2 + \cdots + w_n^2}
          \end{bmatrix}
          \overset{\norm{\vec{w}} = 1}{=} \begin{bmatrix}
            \frac{2w_1(w_1+1)}{2(w_1+1)} \\
            \frac{2(w_1+1)w_2}{2(w_1+1)} \\
            \cdots \\
            \frac{2(w_1+1)w_n}{2(w_1+1)} \\
          \end{bmatrix}
          = - \vec{w},
        \]
        which is a multiple of $\vec{v}$, since
        $\vec{w} =  \frac{\vec{v}}{\norm{\vec{v}}}$.
      \end{samsolution}
    \end{writeverbatim}
  \end{samwriteprbpart}
  % END ================ SOLUTION  ================   %
  \end{subproblem}

%%%%%%%%%%%%%%%%%%%%%%%%%%%%%%%%%%%%%%%%
%%%%%%%%%%%% SUBPROBLEM 1: Compute kron
%%%%%%%%%%%%%%%%%%%%%%%%%%%%%%%%%%%%%%%%
\begin{subproblem}{subprb:4}
  What property does the set of columns of the matrix $\vec{Z}$ have?
  What is the purpose of the function \func{houserefl}?

  \begin{samwriteprbpart}{impl}
    \begin{writeverbatim}{prbfile
        \begin{samhint}
          Use \ref{subprb:HouseRefl_a} and \ref{subprb:HouseRefl_b}.
        \end{samhint}
    \end{writeverbatim}
  \end{samwriteprbpart}

  % BEGIN ================ SOLUTION  ================   %
  \begin{samwriteprbpart}{impl}
    \begin{writeverbatim}{prbfile}
      \begin{samsolution}
        The columns of $\vec{X} = [\vec{X}_1, \cdots, \vec{X}_n]$
        are an orthonormal basis (ONB) of $\IR^n$ (cf.
        \ref{subprb:HouseRefl_a}). Thus, the columns of
        $\vec{Z} = [\vec{X}_2, \cdots, \vec{X}_n]$ are an ONB of
        the complement of
        $\mathrm{Span}(\vec{X}_1) \overset{\ref{subprb:HouseRefl_b}}{=}
        \mathrm{Span}(\vec{v})$.
        The function \func{houserefl} computes an ONB of the
        complement of $\vec{v}$.
      \end{samsolution}
    \end{writeverbatim}
  \end{samwriteprbpart}
  % END ================ SOLUTION  ================   %
  \end{subproblem}

  %%%%%%%%%%%%%%%%%%%%%%%%%%%%%%%%%%%%%%%%
  %%%%%%%%%%%% SUBPROBLEM 1: Compute kron
  %%%%%%%%%%%%%%%%%%%%%%%%%%%%%%%%%%%%%%%%
  \begin{subproblem}{subprb:cmpl}[1]
    What is the asymptotic complexity of the function \func{houserefl}
    as the length $n$ of the input vector $\vec{v}$ goes to $\infty$?

    % BEGIN ================ SOLUTION  ================   %
    \begin{samwriteprbpart}{impl}
      \begin{writeverbatim}{prbfile}
        \begin{samsolution}
          $O(n^2)$: this is the asymptotic complexity of the construction
          of the tensor product at line $9$ of \autoref{cppc:HouseRefl}.
        \end{samsolution}
      \end{writeverbatim}
    \end{samwriteprbpart}
    % END ================ SOLUTION  ================   %
  \end{subproblem}


%%%%%%%%%%%%%%%%%%%%%%%%%%%%%%%%%%%%%%%%
%%%%%%%%%%%% SUBPROBLEM
%%%%%%%%%%%%%%%%%%%%%%%%%%%%%%%%%%%%%%%%
\begin{subproblem}{subprb:compute}[3]
  Rewrite the function as C++ function and use a
  \emph{standard function} of
  \Matlab~ to achieve the same result of lines 5-9 with a single
  call to this function.

  % BEGIN ================ SOLUTION  ================   %
  \begin{samwriteprbpart}{impl}
    \begin{writeverbatim}{prbfile}
      \begin{samsolution}
        Check the code in \autoref{mc:HouseRefl} for the porting
        to \Matlab~ code. Using the QR-decomposition \func{qr} one can
        rewrite (cf. \autoref{mc:HouseRefl_qr}) the \Cpp~ code in
        \Matlab~ with a few lines.

        \begin{samcode}[C++11-code]{cpp:houserefl}
          {Implementation of \ref{prb:HouseRefl}}
          \samincludecpp{\codes/MatVec/\ProblemName/solutions/houserefl.cpp}[3]
          \gitlabSolF{MatVec}{\ProblemName}{kron.cpp}
        \end{samcode}

      \end{samsolution}
    \end{writeverbatim}
  \end{samwriteprbpart}
  % END ================ SOLUTION  ================   %
  \end{subproblem}

\end{samproblem}
