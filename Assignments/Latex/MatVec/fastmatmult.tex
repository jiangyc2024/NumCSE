\begin{samproblem}*{prb:fastmatmult}{Fast matrix multiplication with \eigen{}}[1]{
\cref{rem:strassen} presents Strassen's algorithm that can achieve
  the multiplication of two dense square matrices of size {$n=2^{k}$},
  $k\in\bbN$, with an asymptotic complexity better than $O(n^{3})$.
}

\begin{subproblem}{sp:1}[1]
	Using \eigen~ implement a function
	\begin{lstlisting}[style=cppsimple]
	 MatrixXd strassenMatMult(const MatrixXd & A, const MatrixXd & B)
	\end{lstlisting}
	that uses Strassen's algorithm to multiply the two matrices $\VA$ and
	$\VB$ and return the result as output. 

  % BEGIN ================ SOLUTION  ================   %
  \begin{samsolution}
    \begin{samcode}[C++-code]{\cpl:cpp:solution}{Strassen's algorithm with \eigen{}}
      \lstincludecpp[style=cpp_problem]{../Solutions/MatVec/fastmatmult/strassen.cpp}{1}
    \end{samcode}
  \end{samsolution}
  % END ================ SOLUTION  ================   %

\end{subproblem}

\begin{subproblem}{sp:2}[1]
	Validate the correctness of your code by comparing the result
    with \eigen's built-in matrix multiplication.

  \begin{samsolution}
    \begin{samcode}[C++-code]{\cpl:cpp:solution}{Strassen's algorithm with \eigen{}}
      \lstincludecpp[style=cpp_problem]{../Solutions/MatVec/fastmatmult/strassen.cpp}{1}
    \end{samcode}
  \end{samsolution}
\end{subproblem}

\begin{subproblem}{sp:3}[1]
	Measure the runtime of your function
    \lstinline[style=cppsimple]|strassenMatMult|
    for random matrices of sizes $2^{k}$, $k=4,\ldots,10$, and compare
    with the matrix multiplication offered by the $\ast$-operator
    of \eigen{}.

  \begin{samhint}
    Use the optimization capabilities of your C++ compiler by appending \\ \lstinline[style=cppsimple]|-DCMAKE_BUILD_TYPE=RELEASE| to the cmake invocation in order to get reliable timings. Please note that this also disables various integrity checks.
  \end{samhint}
  \begin{samhint}
    Ensure that the compiler does not optimize away your computation by using the result of the multiplication in some fashion (e.g. by summing one coefficient of the result matrix and printing it after all measurements finished).
  \end{samhint}
\end{subproblem}

\end{samproblem}
