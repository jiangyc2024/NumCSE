\renewcommand{\chpt}{ch_matvec}

\begin{problem}[Matrix powers] \label{prb:Powers}

\begin{subproblem}[2] \label{subprb:Powers_1}
Implement a \Matlab~ function
$$\mathtt{Pow(A,k)}$$
that, using only basic linear algebra operations (including matrix-vector or matrix-matrix multiplications),
computes efficiently the $k^{th}$ power of the $n\times n$ matrix $\VA$.

\begin{hint}
 use the \Matlab~ operator $\wedge$ to test your implementation on random matrices $\VA$.
\end{hint}

\begin{hint}
 use the \Matlab~ functions \texttt{de2bi} to extract the ``binary digits'' of an integer.
\end{hint}

\begin{solution}
Write $k$ in binary format: $k=\sum_{j=0}^M b_j\;2^j$,  $\:b_j\in\{0,1\}$. Then
$$\VA^k = \prod_{j=0}^M \VA^{2^j\; b_j} = \prod_{j\;s.t.\;b_j=1} \VA^{2^j}. $$
We compute $\VA,\;\VA^2,\;\VA^4,\;\ldots,\VA^{2^M}$ (one matrix-matrix multiplication each) and we multiply only the matrices $\VA^{2^j}$ such that $b_j\ne 0$.

\lstinputlisting[caption={An efficient implementation for \ref{prb:Powers}},label={mc:Pow}]
{\problems/\chpt/MATLAB/Pow.m}
\end{solution}
\end{subproblem}


%======================================================================================

\begin{subproblem}[1] \label{subprb:Powers_2}
Find the asymptotic complexity in $k$ (and $n$) taking into account that in \Matlab~ a matrix-matrix multiplication requires a $O(n^3)$ effort.

\begin{solution}
Using the simplest implementation:
$$A^k = \underbrace{\bigg(\ldots\Big(\big(A\cdot A\big)\cdot A\Big)\ldots \cdot A \bigg)\cdot A}_{k} \qquad \rightarrow \qquad O\big((k-1)n^3\big).$$
Using the efficient implementation from Listing~\ref{mc:Pow}, for each $j\in\{1,2,\ldots,\log_2(k)\}$
we have to perform at most two multiplications ($X*A$ and $A*A$):
$$ \text{complexity}\quad\leq \quad 2 * M * \text{matrix-matrix mult.} \quad \approx \quad 2 * \lceil\log_2 k\rceil * n^3.$$
($\lceil a\rceil = \mathtt{ceil}(a) =\inf\{ b\in\IZ,\; a\leq b\} $).
\end{solution}
\end{subproblem}


%======================================================================================

\begin{subproblem}[1] \label{subprb:Powers_3}
Plot the runtime of the built-in \Matlab~ power ($\wedge$) function and find out the complexity.
Compare it with the function \texttt{Pow} from \ref{subprb:Powers_1}.\\
Use the matrix
$$A_{j,k}= \frac1{\sqrt n} \;\exp\Big(\frac{2\pi i \;jk}n\Big)$$
to test the two functions.

\begin{solution}
\lstinputlisting[caption={Timing plots for \ref{prb:Powers}},label={mc:mainPow}]
{\problems/\chpt/MATLAB/mainPow.m}

\begin{figure}[htb]
\centering
\label{fig:Pow_timings}
\includegraphics[width=0.95\textwidth]{\problems/\chpt/PICTURES/Pow_timings.eps}
\caption{Timings for \ref{prb:Powers}}
\end{figure}

The \Matlab $\wedge$-function has (at most) logarithmic complexity in $k$ but the timing is slightly better than our implementation. % ~68\% on average

All the eigenvalues of the Vandermonde matrix $A$ have absolute value $1$, so the powers $A^k$ are ``stable'':
the eigenvalues of $A^k$ are not approaching neither $0$ nor $\infty$ when $k$ grows.
\end{solution}
\end{subproblem}

\begin{subproblem}[2] \label{sp:pow:o1}
  Using \eigen{}, devise a \Cpp{} function with the calling sequence
\begin{lstlisting}
template <class Matrix>
void matPow(const Matrix &A,unsigned int k);
\end{lstlisting}
that computes the $k^{\mathtt{th}}$ power of the square matrix $\VA$ (passed in
the argument \texttt{A}). Of course, your implementation should be as efficient
as the \matlab{} version from sub-problem \ref{subprb:Powers_1}.

\begin{hint}
 matrix multiplication suffers no aliasing issues (you can safely write \texttt{A = A*A}).
\end{hint}

\begin{hint}
 feel free to use the provided \func{matPow.cpp}.
\end{hint}

\begin{hint}
 you may want to use \texttt{log} and \texttt{ceil}.
\end{hint}

\begin{hint}
 \Eigen~ implementation of power (\texttt{A.pow(k)}) can be found in:
 \begin{verbatim}
  #include <unsupported/Eigen/MatrixFunctions>
 \end{verbatim}
\end{hint}

\begin{solution}
\lstinputlisting[caption={Implementation of \func{matPow}},label={cppc:matPow}]
{\problems/\chpt/CPP/matPow.cpp}
\end{solution}
\end{subproblem}
\end{problem}
