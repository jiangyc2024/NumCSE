\begin{samproblem}*{prb:gramschmidteigen}
        {Gram-Schmidt orthogonalization with \eigen{}}[1]{
The code in \cref{gramschmidt} performs a Gram-Schmidt
orthogonalization of the columns of a givenn argument matrix.
}

\begin{subproblem}{sp:1}[1]
  Based on the C++ linear algebra library \eigen{} implement a function
  \begin{lstlisting}[style=cpp]
MatrixXd gramschmidt(const MatrixXd &A);
  \end{lstlisting}
  that performs the same computations as \cref{gramschmidt}.

  \begin{samhint}
    Use \eigen{}s block operations
    (see \href{https://eigen.tuxfamily.org/dox/group__TutorialBlockOperations.html}
    {here})
  \end{samhint}

  % BEGIN ================ SOLUTION  ================   %
  \begin{samsolution}
    \begin{samcode}[C++-code]{\cpl:cpp:solution}
        {Gram-Schmidt orthogonalization with \eigen{}}
        \samincludecpp
        {./Assignments/Codes/MatVec/GramSchmidt/solutions/gramschmidt.cpp}[1]
    \end{samcode}
  \end{samsolution}
  % END ================ SOLUTION  ================   %

\end{subproblem}

\begin{subproblem}{sp:2}[1]
  \label{sp:strassen:2}
  Test your implementation by applying the function \ccode{gram\_schmidt}
  to a small random matrix
  and checking the orthonormality of the columns of the output
  matrix.

  % BEGIN ================ SOLUTION  ================   %
  \begin{samsolution}
    \begin{samcode}[C++-code]{\cpl:cpp:solution}
    {Test of Gram-Schmidt orthogonalization with \eigen{}}
      \samincludecpp
      {./Assignments/Codes/MatVec/GramSchmidt/solutions/gramschmidt.cpp}[2]
    \end{samcode}
  \end{samsolution}
  % END ================ SOLUTION  ================   %

\end{subproblem}

\end{samproblem}
