% NumCSE/Assignments/Chapters/PolynomialInterpolation/ChebAnalytic.tex
% Exercise requires:
% Solution requires: chebyApprox.hpp chebyApproxDriver.cpp

\begin{samproblem}*{prb:chebanalytic}{Chebychev interpolation of analytic functions (core problem)}{
  This problem concerns Chebychev interpolation (cf. \cref{sec:ChebychevInterpolation}). Using techniques from complex
  analysis, notably the residue theorem \cref{thm:residue}, in class we derived an expression for the interpolation error 
  \cref{eq:recipform} and from it an error bound \cref{eq:intperrbd}, as much sharper alternative to \cref{cor:polintperr}
  and \cref{thm:polintperr} for \emph{analytic} interpolands. 
  The bound tells us that for all $t\in [a,b]$
  \begin{align*}
    \lvert f(t) - \Op{L}_{\Ct} f(t) \rvert 
      \leq 
    \left\lvert \frac{w(x)}{2 \pi i} \int_\gamma 
    \frac{f(z)}{(z-t) w(z)} dz \right\rvert 
      \leq 
    \frac{\lvert \gamma \rvert}{2 \pi} 
    \frac{\max_{a \leq \tau \leq b}\lvert w(\tau) \rvert}{\min_{z \in  \gamma}\lvert w(z) \rvert} 
    \frac{ \max_{z \in \gamma} \lvert f(z) \rvert }{d([a,b], \gamma)}\;,
  \end{align*}
  where {$d([a,b],\gamma)$} is the geometric distance of the integration contour $\gamma\subset\mathbb{C}$ from the interval 
  $[a,b]\subset\mathbb{C}$ in the complex plane. The contour $\gamma$ must be contractible in the domain $D$ of analyticity of 
  $f$ and must wind around $[a,b]$ exactly once, see \cref{analintp}. 

  Now we consider the interval $[-1,1]$. Following \cref{rem:chebipanal}, our task is to find an upper bound for this 
  expression, in the case where $f$ possesses an analytical extension to a complex neighbourhood of $[-1,1]$.

  For the analysis of the Chebychev interpolation of analytic functions we used the
  elliptical contours, see \cref{ellipses},
  \begin{align}
    \gamma_\rho(\theta) := \cos(\theta - i \log(\rho) )\;,\quad
    \forall 0 \leq \theta \leq 2 \pi\;,\quad \rho > 1\;.
  \end{align}
}

%%%%%%%%%%%%%%%%%%%% SUBPROBLEM 1

\begin{subproblem}{sp:1}[2]
 Find an upper bound for the length $\lvert \gamma_\rho \rvert$ of the contour $\gamma_\rho$.
 
 \begin{samhint}
   You may use the arc-length formula for a curve $\gamma: I \rightarrow \mathbb{R}^2$:
  \begin{align}
   \lvert \gamma \rvert = \int_I \lVert \dot\gamma(\tau) \rVert d\tau,
  \end{align}
  where $\dot\gamma$ is the derivative of $\gamma$ w.r.t the parameter
  $\tau$. Recall that the ``length'' of a complex number $z$ viewed as a vector in 
  $\mathbb{R}^{2}$ is just its modulus. 
 \end{samhint}

% BEGIN ================ SOLUTION ===============   %
  \begin{samwriteprbpart}{solfile}
    \begin{writeverbatim}{prbfile}
      \begin{samsolution}
        $\frac{\partial \gamma_\rho(\theta)}{\partial \theta} := -\sin(\theta - i \log(\rho) )$, therefore:
        \begin{align}
          \lvert \gamma_\rho \rvert & = \int_{[0,2 \pi]} \lvert \sin(\tau - i \log(\rho) ) \rvert d\tau \\
            & = \frac{1}{2 \rho} \int_{[0,2 \pi]} \sqrt{ \sin^2(\tau) (1 + \rho^2)^2 + \cos^2(\tau) (1 - \rho^2)^2 } d\tau \\
            & \leq \frac{1}{2 \rho} \int_{[0,2 \pi]} \sqrt{ 2 + 2\rho^4 } d\tau \\
            & \leq \frac{1}{\rho} \pi \sqrt{ 2 (1 + \rho^4) }
        \end{align}
      \end{samsolution}
    \end{writeverbatim}
  \end{samwriteprbpart}
% END ================== SOLUTION ===============   %

\end{subproblem}

Now consider the $S$-curve function (the logistic function):
\begin{align*}
  f(t) := \frac{1}{1+e^{-3t}}\;,\quad t \in \mathbb{R}\;.
\end{align*}

%%%%%%%%%%%%%%%%%%%% SUBPROBLEM 2

\begin{subproblem}{sp:2}[2]
  Determine the maximal domain of analyticity of the extension of $f$ to the complex plane $\mathbb{C}$.
  \begin{samhint}
    Consult \cref{rem:analfunc}.
  \end{samhint}

% BEGIN ================ SOLUTION ===============   %
  \begin{samwriteprbpart}{solfile}
    \begin{writeverbatim}{prbfile}
      \begin{samsolution}
        $f$ is analytic in $D := \mathbb{C} \setminus \{ \frac{2}{3} \pi i c - \frac{1}{3} \pi i \; | \; c \in \mathbb{Z} \}$.

        In fact, $g(t) := 1 + \exp(-3t)$ is an entire function, whereas $h(x) := \frac{1}{x}$ is analytic in $\mathbb{C} \setminus \{ 0 \}$. 
        Therefore, using \cref{thm:analchain}, $f$ is analytic in $\mathbb{C} \setminus \{ z \in \mathbb{C} \; | \; g(z) = 0 \} =: \mathbb{C} \setminus S$. 
        Let $z := a + ib$, $a,b \in \mathbb{R}$. Since:
        \begin{align}
          -1 &= \exp(z) = \exp(a + ib) = \exp(a) \exp(ib) \Leftrightarrow a = 0,\; b \in 2 \pi \mathbb{Z} + \pi \\
          -1 &= \exp(z) \Leftrightarrow z \in i(2 \pi  \mathbb{Z} + \pi) \\
          -1 &= \exp(-3z) \Leftrightarrow z \in \frac{i(2 \pi \mathbb{Z} - \pi)}{3}
        \end{align}
      \end{samsolution}
    \end{writeverbatim}
  \end{samwriteprbpart}
% END ================== SOLUTION ===============   %

\end{subproblem}

%%%%%%%%%%%%%%%%%%%% SUBPROBLEM 3

\begin{subproblem}{sp:3}[2]
  Write a C++ function that computes an approximation $M$ of:
  \begin{align}
    \min_{\rho > 1} \frac{ \max_{z \in \gamma_\rho} \lvert f(z) \rvert }{d([-1,1], \gamma_\rho)},
  \end{align}
  by sampling, where the distance of $[a,b]$ from $\gamma_{\rho}$ is formally defined as
  \begin{align}
    d([a,b], \gamma) := \inf \{ \lvert z - t \rvert \; | \; z \in \gamma, t \in [a,b] \}.
  \end{align}

  \begin{samhint}
    The result of \ref{prb:chebanalytic:sp:2}, together with the knowledge that $\gamma_\rho$ describes an ellipsis, tells you the maximal range 
    $(1,\rho_{max})$ of $\rho$. Sample this interval with $1000$ equidistant steps.
  \end{samhint}
 
  \begin{samhint}
    Apply geometric reasoning to establish that the distance of $\gamma_\rho$ and $[-1,1]$ is $\frac{1}{2}(\rho + \rho^{-1}) - 1$.
  \end{samhint}
 
  \begin{samhint}
    If you cannot find $\rho_{max}$ use $\rho_{max} = 2.4$.
  \end{samhint}
 
  \begin{samhint}
    You can exploit the properties of $\cos$ and the hyperbolic trigonometric functions $\cosh$ and $\sinh$.
  \end{samhint}

% BEGIN ================ SOLUTION ===============   %
  \begin{samwriteprbpart}{solfile}
    \begin{writeverbatim}{prbfile}
      \begin{samsolution}
        The ellipse must be restricted such that the minor axis has length $\leq 2 \pi / 3$ ($2$ times the smallest point, in absolute value, 
        where $f$ is not-analytic). Since this corresponds to the imaginary part of $\gamma_\delta(\theta)$, when $\theta = \pi / 2$, we find:
        \begin{align*}
          \cos(\pi / 2 - i \log(\rho_{max})) = 1 / 3 \pi i \Leftrightarrow \sinh(\log(\rho_{max})) = 
          \pi/3 \Leftrightarrow \rho_{max} = \exp(\sinh^{-1}(\pi / 3)).
        \end{align*}
        See \verb|chebyApprox.hpp| and \verb|chebyApproxDriver.cpp| for the C++ code:

        Results:
        {\small \ttfamily
        \begin{lstlisting}
using maxRho = 2.49517:
max = 5.07452, rho = 1.9863
using maxRho = 2.4 (suggested alternative):
max = 5.07449, rho = 1.98659
        \end{lstlisting}
        }

        \begin{samcode}{cpp:chebyapprox}{chebyAppox.hpp}
          \small
          \lstinputlisting[style=cpp]{./Chapters/PolynomialInterpolation/CPP/chebyApprox.hpp}
        \end{samcode}
      \end{samsolution}
    \end{writeverbatim}
  \end{samwriteprbpart}
% END ================== SOLUTION ===============   %

\end{subproblem}

%%%%%%%%%%%%%%%%%%%% SUBPROBLEM 4

\begin{subproblem}{sp:4}[2]
  Based on the result of \ref{prb:chebanalytic:sp:3}, and  \cref{eq:chebanalest}, give an ``optimal'' bound for
  \begin{align*}
    \lVert f - L_n f \rVert_{L^\infty([-1,1])},
  \end{align*}
  where $L_n$ is the operator of Chebychev interpolation on $[-1,1]$ into the space of polynomials of degree $\leq n$.

% BEGIN ================ SOLUTION ===============   %
  \begin{samwriteprbpart}{solfile}
    \begin{writeverbatim}{prbfile}
      \begin{samsolution}
        In problem \ref{prb:chebanalytic:sp:1} we computed an upper bound for the length of the curve for a given rho.
        Let $M$ be the approximation of \ref{prb:chebanalytic:sp:3}. Then
        \begin{align*}
          \lVert f - L_n f \rVert_{L^\infty([-1,1])} \lesssim \frac{M \sqrt{2 (\rho^{-2} + \rho^{2})}}{\rho^{n+1} - 1}.
        \end{align*}
     \end{samsolution}
    \end{writeverbatim}
  \end{samwriteprbpart}
% END ================== SOLUTION ===============   %
\end{subproblem}

%%%%%%%%%%%%%%%%%%%% SUBPROBLEM 5

\begin{subproblem}{sp:5}[1]
  Graphically compare your result from \ref{prb:chebanalytic:sp:4} with the measured supremum norm of the approximation error of Chebychev interpolation 
  of $f$ on $[-1,1]$ for polynomial degree $n = 1,\dots,20$. To that end, write a C++-code and rely on the provided function 
  \verb|intpolyval| (cf. \cref{trigpolyval}).

  \begin{samhint}
    The function \texttt{intpolyval} has the signature:
    \begin{lstlisting}[style=cpp]
void intpolyval(const VectorXd& t, const VectorXd& y, const VectorXd& x, VectorXd& p);
    \end{lstlisting}
    Where $(t,y)$ is the interpolation data and $p$ is used to save the values of the interpolant evaluated in $x$.
  \end{samhint}

  \begin{samhint}
    Use semi-logarithmic scale for your plot:
    \begin{lstlisting}[style=cpp]
mgl::Figure fig;
fig.setlog(false, true);
    \end{lstlisting}
  \end{samhint}

% BEGIN ================ SOLUTION ===============   %
  \begin{samwriteprbpart}{solfile}
    \begin{writeverbatim}{prbfile}
      \begin{samsolution}
        \begin{samcode}{cpp:chebyErr}{chebyErr.cpp}
          \small
          \lstinputlisting[style=cpp]{./Chapters/PolynomialInterpolation/CPP/chebyErr.cpp}
        \end{samcode}
      \end{samsolution}
    \end{writeverbatim}
  \end{samwriteprbpart}
% END ================== SOLUTION ===============   %

\end{subproblem}


%%%%%%%%%%%%%%%%%%%% SUBPROBLEM 6

\begin{subproblem}{sp:6}[4]
  Rely on pullback to $[-1,1]$ to discuss how the error bounds in \cref{eq:chebanalest} will change when we consider Chebychev interpolation on
 $[-a,a], a > 0$, instead of $[-1,1]$, whilst keeping the function $f$ fixed.
  
\iffalse
% BEGIN ================ SOLUTION ===============   %
  \begin{samwriteprbpart}{solfile}
    \begin{writeverbatim}{prbfile}
      \begin{samsolution}
        The rescaled function $\Phi^{\ast}f$ will have a different domain of analyticity and a different growth behavior in the complex plane. 
        The larger $a$, the closer the pole of $\Phi^{\ast}f$ will move to $[-1,1]$, the more the choice of the ellipses is restricted 
        (i.e. $\rho_{max}$ becomes smaller). This will result in a larger bound.
   
        Using \cref{eq:tpdft}, it follows immediately that the asymptotic behaviour of the interpolation does not change after rescaling of the interval. 
        In fact, if $\Phi^*$ is the affine pullback from $[-a,a]$ to $[-1,1]$, then:
        \begin{align*}
          \lVert f - \hat{(L_n)} f \rVert_{L^\infty([-a,a])} = \lVert \Phi^* f - L_n \Phi^* f \rVert_{L^\infty([-1,1])},
        \end{align*}
        where $\hat{(L_n)}$ is the interpolation on $[-a,a]$.
      \end{samsolution}
    \end{writeverbatim}
  \end{samwriteprbpart}
% END ================== SOLUTION ===============   %
 \begin{solution}
   The rescaled function $\Phi^{\ast}f$ will have a different domain of analyticity and a different growth behavior in the complex plane. 
   The larger $a$, the closer the pole of $\Phi^{\ast}f$ will move to $[-1,1]$, the more the choice of the ellipses is restricted 
   (i.e. $\rho_{max}$ becomes smaller). This will result in a larger bound.
   
   Using \cref{eq:tpdft}, it follows immediately that the asymptotic behaviour of the interpolation does not change after rescaling of the interval. 
   In fact, if $\Phi^*$ is the affine pullback from $[-a,a]$ to $[-1,1]$, then:
   \begin{align*}
     \lVert f - \hat{(L_n)} f \rVert_{L^\infty([-a,a])} = \lVert \Phi^* f - L_n \Phi^* f \rVert_{L^\infty([-1,1])},
   \end{align*}
   where $\hat{(L_n)}$ is the interpolation on $[-a,a]$.
 \end{solution}
\fi
\end{subproblem}

\end{samproblem}
