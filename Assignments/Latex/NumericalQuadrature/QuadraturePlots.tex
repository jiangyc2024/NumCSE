% ncse_new/p2_InterpolationApproximation/ch4_NumericalQuadrature/ex_QuadraturePlots.tex
% exercise requires:   ExQuadPlots.jpg     ExQuadPlots.eps
% solutions require:   -

\begin{samproblem}{prb:QuadraturePlots}{Quadrature plots}{
  We consider three different functions on the interval $I=[0,1]$:
  \begin{align*}
    \text{function A:}\quad f_A&\in C^\infty(I)\;,\quad f_A\notin\Cp_k\; \forall\;k\in\mathbb{N}\;;\\
    \text{function B:}\quad f_B&\in C^0(I)\;,\quad f_B\notin C^1(I)\;;\\
    \text{function C:}\quad f_C&\in\Cp_{12}\;,
  \end{align*}
  where $\Cp_k$ is the space of the polynomials of degree at most $k$ defined on $I$.
  The following quadrature rules are applied to these functions:
  \begin{itemize}
    \item quadrature rule A,\quad global Gauss quadrature;
    \item quadrature rule B,\quad composite trapezoidal rule;
    \item quadrature rule C,\quad composite 2-point Gauss quadrature.
  \end{itemize}
  The corresponding absolute values of the quadrature errors are plotted against the number of function evaluations in Figure~\ref{fig:QuadraturePlots}.
  Notice that only the quadrature errors obtained with an even number of function evaluations are shown.
}

  \begin{figure}[hbt]
    \centering
    \hspace{8mm}
    \samplot{./Chapters/NumericalQuadrature/PICTURES/ExQuadPlots}[fig:QuadraturePlots][](0.9\textwidth)
    \caption{Quadrature convergence plots for different functions and different rules.}
  \end{figure}

%%%%%%%%%%%% SUBPROBLEM 1

\begin{subproblem}{sp:1}[3] 
  Match the three plots (plot \#1, \#2 and \#3) with the three quadrature rules (quadrature rule A, B, and  C). Justify your answer.\\[1.5ex]
  \begin{samhint}
    Notice the different axis scales in the plots.
  \end{samhint}

  
% BEGIN ================ SOLUTION ===============   %
  \begin{samwriteprbpart}{solfile}
    \begin{writeverbatim}{prbfile}
      \begin{samsolution}
        Plot \#1 --- Quadrature rule C, Composite 2-point Gauss:\\
        algebraic convergence for every function, about $4^{th}$ order for two functions.

        Plot \#2 --- Quadrature rule B, Composite trapezoidal:\\ algebraic convergence for every function, about $2^{nd}$ order.

        Plot \#3 --- Quadrature rule A, Global Gauss:\\ algebraic convergence for one function, 
        exponential for another one, exact integration with 8 evaluations for the third one.
     \end{samsolution}
    \end{writeverbatim}
  \end{samwriteprbpart}
% END ================== SOLUTION ===============   %

\end{subproblem}

%%%%%%%%%%%% SUBPROBLEM 2

\begin{subproblem}{sp:2}[3]
  The quadrature error curves for a particular function $f_A$, $f_B$ and $f_C$ are plotted in the same style 
  (curve 1 as red line with small circles, curve 2 means the blue solid line, curve 3 is the black dashed line).
  Which curve corresponds to which function ($f_A$, $f_B$, $f_C$)?
  Justify your answer.

% BEGIN ================ SOLUTION ===============   %
  \begin{samwriteprbpart}{solfile}
    \begin{writeverbatim}{prbfile}
      \begin{samsolution}
        Curve 1 red line and small circles --- $f_C$ polynomial of degree 12:\\
        integrated exactly with 8 evaluations with global Gauss quadrature.

        Curve 2 blue continuous line only --- $f_A$ analytic function:\\
        exponential convergence with global Gauss quadrature.

        Curve 3 black dashed line --- $f_B$ non smooth function:\\
        algebraic convergence with global Gauss quadrature.
     \end{samsolution}
    \end{writeverbatim}
  \end{samwriteprbpart}
% END ================== SOLUTION ===============   %

\end{subproblem}

\end{samproblem}
