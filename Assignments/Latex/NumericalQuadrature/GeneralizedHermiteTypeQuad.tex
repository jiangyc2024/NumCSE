\begin{samproblem}{prb:gen_hermite}{Generalize ``Hermite-type'' quadrature formula}{
  In this exercise we will recall on how to compute a quadrature formula and then use it in an application.

  You may want to have a look at the methods used in \cref{ex:gq4}, as they will come in handy.
}
 
\begin{subproblem}{sp:1}[4]
  Determine $A,B,C, x_1 \in \mathbb{R}$ such that the quadrature formula:
  \begin{align}\label{eq:gen_hermite_quad}
    \int_0^1 f(x) dx \approx A f(0) + B f'(0) + C f(x_1)
  \end{align}
  is exact for polynomials of highest possible degree.
 
% BEGIN ================ SOLUTION ===============   %
  \begin{samwriteprbpart}{solfile}
    \begin{writeverbatim}{prbfile}
      \begin{samsolution}
        The quadrature is exact for every polynomial $p(x) \in \mathcal{P}^n$, 
        if and only if it is exact for $1,x,x^2,\dots,x^n$. 
        If we apply the quadrature to the first monomials:
        \begin{align}
          1 = \int_0^1 1 dx & = A \cdot 1 + B \cdot 0 + C \cdot 1 = A + C\\
          \frac{1}{2} = \int_0^1 x dx & = A \cdot 0 + B \cdot 1 + C \cdot x_1 = B + C x_1 \\
          \frac{1}{3} = \int_0^1 x^2 dx & = A \cdot 0 + B \cdot 0 + C \cdot x_1^2 = C x_1^2 \\
          \frac{1}{4} = \int_0^1 x^3 dx & = A \cdot 0 + B \cdot 0 + C \cdot x_1^3 = C x_1^3
        \end{align}
        $\Rightarrow B = \frac{1}{2} - C x_1, C = \frac{1}{3 x_1^2} \Rightarrow \frac{1}{4} = \frac{1}{3 x_1^2} x_1^3 = \frac{1}{3} x_1, A = \frac{11}{27}$, i.e.
        \begin{align}
          x_1 = \frac{3}{4}, C = \frac{16}{27}, B = \frac{1}{18}, A = \frac{11}{27}.
        \end{align}
        Then
        \begin{align}
          \frac{1}{5} = \int_0^1 x^4 dx \neq A \cdot 0 + B \cdot 0 + C \cdot x_1^4 = C \cdot x_1^4 = \frac{16}{27} \frac{81}{256}.
        \end{align}
        Hence, the quadrature is exact for polynomials up to degree $3$.
      \end{samsolution}
    \end{writeverbatim}
  \end{samwriteprbpart}
% END ================== SOLUTION ===============   %

\end{subproblem}
 
\begin{subproblem}{sp:2}[4]
 
  Compute an approximation of $z(2)$, where the function $z$ is defined as the solution of the initial value problem
  \begin{align}
    z'(t) = \frac{t}{1 + t^2}\quad,\quad z(1) = 1\;.
  \end{align}

  \begin{samhint}
    Use the fundamental theorem of calculus and your result from the previous exercise.
  \end{samhint}
 
% BEGIN ================ SOLUTION ===============   %
  \begin{samwriteprbpart}{solfile}
    \begin{writeverbatim}{prbfile}
      \begin{samsolution}
        We know that
        \begin{align}
          z(2) - z(1) = \int_1^2 z'(x) dx,
        \end{align}
        hence, applying \eqref{eq:gen_hermite_quad} and the transformation $x \mapsto x + 1$, we obtain:
        \begin{align}
          z(2) = \int_0^1 z'(x + 1) dx + z(1) \approx \frac{11}{27} \cdot z'(1) + \frac{1}{18} \cdot z''(1) + \frac{16}{27} \cdot z'\left(\frac{7}{4}\right) + z(1).
        \end{align}
          With $z''(x) = - \frac{2 \cdot x}{(1 + x^2)^2}$ and:
        \begin{align*}
          z(1) = 1, \\
          z'(1) = \frac{1}{1 + 1^2} = \frac{1}{2}, \\
          z''(1) = - \frac{2 \cdot 1}{(1 + 1^2)^2} = -\frac{1}{2}, \\
          z'\left(\frac{7}{4}\right) = \frac{(\frac{7}{4})}{1 + (\frac{7}{4})^2} = \frac{28}{65},
        \end{align*}
        we obtain
        \begin{align*}
          z(2) = \int_0^1 z'(x + 1) dx + z(1) \approx \frac{11}{27} \cdot \frac{1}{2} - \frac{1}{18} \cdot \frac{1}{2} + \frac{16}{27} \cdot \frac{28}{65} + 1 = 1.43 \dots
        \end{align*}
        For sake of completeness, using the antiderivative of $z'$:
        \begin{align*}
          z(2) = \int_1^2 z'(x) dx + z(1) = \frac{1}{2} \log(x^2 + 1) |_1^2 + 1 = 1.45 \dots
        \end{align*}
      \end{samsolution}
    \end{writeverbatim}
  \end{samwriteprbpart}
% END ================== SOLUTION ===============   %

\end{subproblem}
 
\end{samproblem}
