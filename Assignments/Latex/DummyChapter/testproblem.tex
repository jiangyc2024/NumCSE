\chapter{Test Problems}
\Label{cha:prb}

%%%%%%%%%%%%%%%%%%%%%%%%%%%%%%%%%%%%%%%%%%%%%%
% Test problem for new environment 'samproblem'
%%%%%%%%%%%%%%%%%%%%%%%%%%%%%%%%%%%%%%%%%%%%%%

\begin{samproblem}*{prb:test}{Test problem}[3](15){This problem tests the new \texttt{samproblem} \LaTeX environment.

It is related to \cref{sec:cpp} and considers the following formula
\begin{sammath}{gather}
  \prblabel{eq:0}
  s = \sum\limits_{k=0}^{n}f(k)\;\bdot
 \end{sammath}%
}

\begin{samhint}
  This is a lonely hint at the toplevel of \cref{prb:test}. 
\end{samhint}

\begin{subproblem}{sp:1}[3](3)
  \par\begin{lstlisting}[style=cpp]
template <typename Function>
double proccess(Function &f,const std::vector<double> &x) {
  double s = 0;
  // Sum according to \prbeqref{eq:0}
  for(auto i : x) { s += f(i); } 
  return s; 
}
\end{lstlisting}

We consider the integral 
\begin{sammath}{gather}
  \prblabel{eq:1}
  \int\limits_{0}^{1}e^{x}\,\textrm{d}e = ?
\end{sammath}%
In \prbref{eq:1} we see the disturbing impact of alien notation, see also \cref{par:ops}.

  % BEGIN ================ OUTPUT  ================   %
  \begin{samwriteprbpart}{h1}
    \begin{writeverbatim}{prbfile}
      \begin{samhint}
        Recall \prbcref{eq:1}!
      \end{samhint}
    \end{writeverbatim}
  \end{samwriteprbpart}
  
  % BEGIN ================ OUTPUT  ================   %
  \begin{samwriteprbpart}{h2}
    \begin{writeverbatim}{prbfile}
      \begin{samhint}
        Take into account \cref{mc:sa3} and
        \begin{sammath}{gather}
          \prblabel{h:1}
          a^{2} - b^{2} = (a+b)(a-b)\;\bdot
        \end{sammath}%
      \end{samhint}
    \end{writeverbatim}
  \end{samwriteprbpart}

  % BEGIN ================ OUTPUT  ================   %
  \begin{samwriteprbpart}{testsol}
    \begin{writeverbatim}{prbfile}
      \begin{samsolution}
        We make use of \prbeqref{h:1} and \prbeqref{eq:1} and conclude
        \begin{sammath}{gather}
          \prblabel{s:1}
          1*1 = 1 !
        \end{sammath}%
      \end{samsolution}
    \end{writeverbatim}
  \end{samwriteprbpart}

\end{subproblem}

This is text between \prbcref{sp:1} and \prbcref{sp:2}. 

\begin{subproblem}{sp:2}<\prbref{sp:1}>
  This subproblem continues \prbcref{sp:1} and looks at
  \begin{sammath}{gather}
    \prblabel{eq:2}
    \btext{\prbeqref{eq:1}}\quad \Black{\Rightarrow}\quad
    x^{2} = x-1\;\bdot
  \end{sammath}%
  The entire sub-problem is indented. 

  
  % BEGIN ================ OUTPUT  ================   %
  \begin{samwriteprbpart}{sol}
    \begin{writeverbatim}{prbfile}
      \begin{samsolution}
        This is the solution of \prbcref{sp:2}!

        \begin{minipage}[c]{0.5\textwidth}
          This means
          \begin{sammath}{gather}
            \prblabel{pl:1}
            1 + 1 = 2 !
          \end{sammath}%
          and this might be related to \prbeqref{h:1}, see
          \cref{drawing}.
        \end{minipage}%
        \begin{minipage}[c]{0.5\textwidth}
          \samplot{drawing}
        \end{minipage}%
      \end{samsolution}
    \end{writeverbatim}
  \end{samwriteprbpart}
\end{subproblem}

\end{samproblem}

\bigskip 

Outside the problem we can reference equations: \eqref{prb:test:eq:0} or 
\cref{prb:test:eq:1}.

We can also referennce \cref{prb:test:sp:2} or \ref{prb:test:sp:2}.
\bigskip


