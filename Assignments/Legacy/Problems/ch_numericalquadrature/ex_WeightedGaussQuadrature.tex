\begin{problem}[Weighted Gauss quadrature]
The development of an alternative quadrature formula for \eqref{eq:effquadsingint} relies on the Chebyshev polynomials of the second kind $U_n$, defined as
\begin{equation*}
U_n(t)=\frac{\sin((n+1)\arccos t)}{\sin(\arccos t)}, \qquad n\in\n.
\end{equation*}
Recall the role of the orthogonal Legendre polynomials in the derivation and definition of Gauss-Legendre quadrature rules (see \lref{par:LP}).

As regards the integral \eqref{eq:effquadsingint}, this role is played by the $U_n$, which are orthogonal polynomials with respect to a weighted $L^2$ inner product, see \lref{eq:wSPLtwo}, with weight given by $w(\tau)=\sqrt{1-\tau^2}$.
% SUBPROBLEM 1
\begin{subproblem}[2]
Show that the $U_n$ satisfy the 3-term recursion
\[
U_{n+1}(t)=2tU_n(t)-U_{n-1}(t),\qquad U_0(t)=1,\qquad U_1(t)=2t,
\]
for every $n\ge 1$.
\begin{solution}
The case $n=0$ is trivial, since $U_0(t)=\frac{\sin(\arccos t)}{\sin(\arccos t)}=1$, as desired. Using the trigonometric identity $\sin 2x = \sin x \cos x$, we have $U_1(t)=\frac{2\sin(\arccos t)}{\sin(\arccos t)}=2\cos \arccos t = 2t$, as desired. Finally, using the identity $\sin (x+y) = \sin x \cos y + \sin y \cos x$, we obtain for $n\ge 2$
\[
\begin{split}
U_{n+1}(t)   &= \frac{\sin((n+1)\arccos t)t + \cos((n+1)\arccos t) \sin (\arccos t)}{\sin(\arccos t)} \\
& = U_n(t) t+\cos((n+1)\arccos t).
\end{split}
\]
Similarly, we have
\[
\begin{split}
U_{n-1}(t)&=\frac{\sin((n+1 -1 )\arccos t)}{\sin(\arccos t)} \\ 
&= \frac{\sin((n+1)\arccos t)t - \cos((n+1)\arccos t) \sin (\arccos t)}{\sin(\arccos t)}\\
& = U_n(t) t-\cos((n+1)\arccos t).
\end{split}
\]
Combining the last two equalities we obtain the desired 3-term recursion.
\end{solution}
\end{subproblem}

% SUBPROBLEM 2
\begin{subproblem}[1]
Show that $U_n\in \Pol{n}$ with leading coefficient $2^n$.
\begin{solution}
Let us prove the claim by induction. The case $n=0$ is trivial, since $U_0(t)=1$. Let us now assume that the statement is true for every $k=0,\dots,n$ and let us prove it for $n+1$. In view of $U_{n+1}(t)=2tU_n(t)-U_{n-1}(t)$, since by inductive hypothesis $U_n\in\Pol{n}$ and $U_{n-1}\in\Pol{n-1}$, we have that $U_{n+1}\in \Pol{{n+1}}$. Moreover, the leading coefficient will be $2$ times the leading order coefficient of $U_n$, namely $2^{n+1}$, as desired.
\end{solution}
\end{subproblem}

% SUBPROBLEM 3
\begin{subproblem}[2]\label{subpb:orthogonal}
Show that for every $m,n\in\n_0$ we have
\[
\int_{-1}^1 \sqrt{1-t^2}\,U_m(t) U_n(t)\,dt=\frac{\pi}{2}\delta_{mn}.
\]
\begin{solution}
With the substitution $t=\cos s$ we obtain
\[
\begin{split}
\int_{-1}^1 \sqrt{1-t^2}U_m(t) U_n(t)\,dt &= \int_{-1}^1 \sqrt{1-t^2}  \frac{\sin((m+1)\arccos t)\sin((n+1)\arccos t)}{\sin^2(\arccos t)}\,dt \\
&=  \int_{0}^\pi \sin s  \frac{\sin((m+1)s)\sin((n+1)s)}{\sin^2 s}\sin s\,ds \\
&=  \int_{0}^\pi  \sin((m+1)s)\sin((n+1)s)\,ds \\
&=\frac{1}{2} \int_{0}^\pi  \cos((m-n)s) - \cos((m+n+2)s)\,ds. \\
\end{split}
\]
The claim immediately follows, as it was done in \ref{PS9}, \ref{prob:ChebPolyProp}.
\end{solution}
\end{subproblem}

% SUBPROBLEM 4
\begin{subproblem}[1]
What are the zeros $\xi^n_j$ ($j=1,\dots,n$) of $U_n$, $n\ge 1$? Give an explicit formula similar to the formula for the Chebyshev nodes in $[-1,1]$.
\begin{solution}
From the definition of $U_n$ we immediately find that the zeros are given by
\begin{equation}\label{eq:zeros}
\xi^n_j= \cos\left(\frac{j}{n+1}\pi\right),\qquad j=1,\dots,n.
\end{equation}
\end{solution}
\end{subproblem}

% SUBPROBLEM 5
\begin{subproblem}[4]
Show that the choice of weights
\begin{equation*}
w_j=\frac{\pi}{n+1}\sin^2\left(\frac{j}{n+1}\pi\right),\qquad j=1,\dots, n,
\end{equation*}
ensures that the quadrature formula
\begin{equation}\label{eq:quadrature_formula}
Q^U_n(f)=\sum_{j=1}^n w_j f(\xi^n_j)
\end{equation}
provides the exact value of  \eqref{eq:effquadsingint} for $f\in\Pol{n-1}$ (assuming exact arithmetic).
 \begin{hint}
Use  all the previous subproblems.
\end{hint}
\begin{solution}
Since $U_{k}$ is a polynomial of degree exactly $k$, the set $\{U_k:k=0,\dots,n-1\}$ is a basis of $\Pol{n-1}$. Therefore, by linearity it suffices to prove the above identity for $f=U_k$ for every $k$. Fix $k=0,\dots,n-1$. Setting $x=\pi/(n+1)$, from \eqref{eq:zeros} we readily derive
\[
\begin{split}
\sum_{j=1}^n w_j U_k(\xi^n_j) &=  \sum_{j=1}^n \frac{\pi}{n+1}\sin^2\left(\frac{j}{n+1}\pi\right) \frac{\sin((k+1)\arccos \xi^n_j)}{\sin(\arccos \xi^n_j)}\\
&=  x \sum_{j=1}^n \sin(jx) \sin((k+1)jx)\\
&=  \frac{x}{2} \sum_{j=1}^n\left(\cos((k+1-1)jx)-\cos((k+1+1)jx)\right)\\
&=  \frac{x}{2}\Re \sum_{j=0}^n \left( e^{i k x j}-e^{i(k+2) x j}\right)\\
&=  \frac{x}{2}\Re\left( \sum_{j=0}^n  e^{i k x j}-\frac{1-e^{i\pi(k+2)}}{1-e^{i (k+2) x }}\right).
\end{split}
\]
Thus, for $k=0$ we have
\[
\sum_{j=1}^n w_j U_0(\xi^n_j) = \frac{x}{2}\Re\left( \sum_{j=0}^n  1-\frac{1-e^{2\pi i}}{1-e^{ 2 x i }}\right)
= \frac{x}{2}\Re\left( (n+1)-0\right) = \frac{\pi}{2}.
\]

On the other hand, if $k=1,\dots,n-1$ we obtain
\[
\sum_{j=1}^n w_j U_k(\xi^n_j) =  \frac{x}{2}\Re\left( \frac{1-e^{i\pi k}}{1-e^{i k x}}-\frac{1-e^{i\pi(k+2)}}{1-e^{i (k+2) x }}\right) =  \frac{(1-(-1)^k)x}{2}\Re\left( \frac{1}{1-e^{i k x}}-\frac{1}{1-e^{i (k+2) x}}\right).
\]
In view of the elementary equality $(a+ib)(a-ib)=a^2+b^2$ we have $\Re(1/(a+ib)) = a/(a^2+b^2)$. Thus
\[
\Re\left( \frac{1}{1-e^{i k x}}\right) = \Re\left( \frac{1}{1-\cos(kx)-i\sin(kx)} \right)= \frac{1-\cos(kx)}{(1-\cos(kx))^2 + \sin(kx)^2}  = \frac{1}{2}.
\]
Arguing in a similar way we have $\Re\,(1-e^{i (k+2) x})^{-1}= 1/2$. Therefore for $k=1,\dots,n-1$ we have
\[
\sum_{j=1}^n w_j U_k(\xi^n_j) = \frac{(1-(-1)^k)x}{2}\left( \frac{1}{2} -\frac{1}{2} \right) = 0.
\]
To summarise, we have proved that
\[
 \sum_{j=1}^n w_j U_k(\xi^n_j) = \frac{\pi}{2}\delta_{k0},\qquad k=0,\dots,n-1.
  \]
  
Finally, the claim follows from \ref{subpb:orthogonal}, since $U_0(t)=1$ and so the integral in  \eqref{eq:effquadsingint} is nothing else than the weighted scalar product between $U_k$ and $U_0$.
\end{solution}
\end{subproblem}

% SUBPROBLEM 6
\begin{subproblem}[2]
Show that the quadrature formula \eqref{eq:quadrature_formula} gives the exact value of \eqref{eq:effquadsingint}  even for every $f\in \Pol{2n-1}$.
 \begin{hint}
See \lref{thm:Gaussquad}.
\end{hint}
\begin{solution}
The conclusion follows by applying the same argument given in  \lref{thm:Gaussquad} with the weighted $L^2$ scalar product with weight $w$ defined above.
\end{solution}
\end{subproblem}

% SUBPROBLEM 7
\begin{subproblem}[3]
Show that the quadrature error
\[
|Q^U_n(f)-W(f)|
\]
decays to $0$ exponentially as $n\to \infty$ for every $f\in C^\infty([-1,1])$ that admits an analytic extension to an open subset of the complex plane.
\begin{hint}
See \lref{par:quadbest}.
\end{hint}
\begin{solution}
By definition, the weights defined above are positive, and the quadrature rule is exact for polynomials up to order $2n-1$. Therefore, arguing as in \lref{par:quadbest}, we obtain the exponential decay, as desired.
\end{solution}
\end{subproblem}

% SUBPROBLEM 8
\begin{subproblem}[2]
Write a \Cpp{} function
\begin{lstlisting}
template<typename Function>
double quadU(const Function &f, unsigned int n)
\end{lstlisting}
that gives $Q^U_n(f)$ as output, where \texttt{f} is an object with an evaluation operator, like a lambda function, representing $f$, e.g.
\begin{lstlisting}
auto f = [] (double & t) {return 1/(2 + exp(3*t));};
\end{lstlisting}
\begin{solution}
See file \texttt{quadU.cpp}.
\end{solution}
\end{subproblem}


% SUBPROBLEM 8
\begin{subproblem}[2]
Test your implementation with the function $f(t)=1/(2+e^{3t})$ and
$n=1,\dots,25$. Tabulate the quadrature error $E_n(f)=|W(f)-Q^U_n(f)|$ using the
``exact'' value $W(f)=0.483296828976607$. Estimate the parameter $0\le q<1$ in the
asymptotic decay law $E_n(f)\approx Cq^n$ characterizing (sharp) exponential
convergence, see \lref{def:cvgtype}. 
\begin{solution}
See file \texttt{quadU.cpp}.  An approximation of $q$ is given by $E_n(f)/E_{n-1}(f)$.
\end{solution}
\end{subproblem}


\end{problem}