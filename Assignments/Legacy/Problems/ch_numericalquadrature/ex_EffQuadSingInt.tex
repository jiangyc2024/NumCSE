
\begin{problem}[Efficient quadrature of singular integrands \coreproblem]
\label{prb:effquadsing}
 
 This problem deals with efficient numerical quadrature of non-smooth integrands with a special structure. Before you tackle this problem, read about regularization of integrands by transformation \ncserem{rem:sqrtquad}.

 Our task is to develop quadrature formulas for integrals of the form:
 \begin{align} \label{eq:effquadsingint}
  W(f) := \int_{-1}^1 \sqrt{1 - t^2}\, f(t) dt,
 \end{align}
 where $f$ possesses an analytic extension to a complex neighbourhood of $[-1,1]$.

 \begin{subproblem}[1]
 The provided function
 \begin{lstlisting}[language=c++]
  QuadRule gauleg(unsigned int n);
 \end{lstlisting}
 returns a structure \verb|QuadRule| containing nodes $(x_j)$ and weights $(w_j)$
 of a Gauss-Legendre quadrature ($\to$ \lref{def:gaussquad}) on $[-1,1]$ with $n$
 nodes. Have a look at the file \verb|gauleg.hpp| and \verb|gauleg.cpp|, and
 understand how the implementation works and how to use it.
 
  \cprotEnv \begin{hint}
   Learn/remember how linking works in \Cpp{}. To use the function \verb|gauleg| (declared in \verb|gauleg.hpp| and defined in \verb|gauleg.cpp|) in a file \verb|file.cpp|, first \verb|include| the header file \verb|gauleg.hpp| in the file \verb|file.cpp|, and then compile and link the files \verb|gauleg.cpp| and \verb|file.cpp|. Using \verb|gcc|:
   \begin{lstlisting}
g++ [compiler opts.] -c gauleg.cpp
g++ [compiler opts.] -c file.cpp
g++ [compiler opts.]  gauleg.o file.o -o exec_name
   \end{lstlisting}
   If you want to use \verb|CMake|, have a look at the file \verb|CMakeLists.txt|.
  \end{hint}
  
  \cprotEnv \begin{solution}
   See documentation in \verb|gauleg.hpp| and \verb|gauleg.cpp|.
  \end{solution}
 \end{subproblem}

 \begin{subproblem}[1]
   Study \lref{par:quadbest} in order to learn about the convergence of
   Gauss-Legendre quadrature. 
 \end{subproblem}

 \begin{subproblem}[3] \label{subprb:effquadsingint}
  Based on the function \verb|gauleg|, implement a \Cpp{} function
 \begin{lstlisting}[language=c++]
 template <class func>
 double quadsingint(func&& f, unsigned int n);
 \end{lstlisting}
 that approximately evaluates \eqref{eq:effquadsingint} using $2n$ evaluations of $f$. An object of type \verb|func| must provide an evaluation operator
 \begin{lstlisting}[language=c++]
 double operator(double t) const;
 \end{lstlisting}
 For the quadrature error asymptotic exponential convergence to zero for
 $n \rightarrow \infty$ must be ensured by your function.
 
 
 \begin{hint}
  A \Cpp{} lambda function provides such operator.
 \end{hint}
 
 \begin{hint}
   You may use the classical binomial formula
   $\sqrt{1 - t^2} = \sqrt{ 1 - t} \sqrt{1 + t}$.
 \end{hint}
 
  \cprotEnv \begin{hint}
  You can use the template \verb|quadsingint_template.cpp|.
  \end{hint}

  \cprotEnv \begin{solution}
   Exploiting the hint, we see that the integrand is non-smooth in $\pm 1$. 
   
   The first possible solution is the following (I): we split the integration domain $[-1,1]$ in $[0,1]$ and $[-1,0]$. Applying the substitution $s = \sqrt{1 \pm t}$ (sign depending on which part of the integrals considered), $t = \pm(s^2 - 1)$:
   \begin{align*}
    \frac{dt}{ds} = \pm2s \\
    W(f) := \int_{-1}^1 \sqrt{1 - t^2} f(t) dt = \int_{-1}^0 \sqrt{1 - t^2} f(t) dt + \int_0^1 \sqrt{1 - t^2} f(t) dt \\
    = \int_0^{1} 2 \cdot s^2 \sqrt{2 - s^2} f(-s^2+1) ds + \int_0^{1} 2 \cdot s^2 \sqrt{2 - s^2} f(s^2-1) ds.
   \end{align*}
   Notice how the resulting integrand is analytic in a neighbourhood of the domain
   of integration because, for instant, {$t\mapsto \sqrt{1+t}$} is analytic
   in a neighborhood of $[0,1]$. 
   
   Alternatively (II), one may use the trigonometric substitution $t = \sin s$, with $\frac{dt}{ds} = \cos s$ obtaining
   \begin{align*}
    W(f) & := \int_{-1}^1 \sqrt{1 - t^2} f(t) dt \\
    & = \int_{-\pi / 2}^{\pi / 2} \sqrt{1 - \sin^2\! s}\, f(\sin s) \cos s\,ds = \int_{-\pi / 2}^{\pi / 2} \cos^2\! s \, f(\sin s) ds.
   \end{align*}
   This integrand is also analytic.
%    hence, using $r = \sqrt{\sqrt(2) - s}, s = \sqrt{2} - r^2$:
%    \begin{align*}
%     \frac{ds}{dr} = -2r \\
%     W(f) := \int_0^{\sqrt{2}} 2 \cdot s^2 \sqrt{2 - s^2} f(1 - s^2) ds  = \int_0^{\sqrt{\sqrt{2}}} 4 r^2 \sqrt{2 \sqrt{2}-r^2} (\sqrt{2} - r^2)^2 f(1 - (\sqrt{2} - r^2)^2) dr
%    \end{align*}
   The \Cpp{} implementation is in \verb|quadsingint.cpp|.
  \end{solution}

 \end{subproblem}

 \begin{subproblem}[2]
  Give formulas for the nodes $c_j$ and weights $\tilde{w}_j$ of a $2n$-point quadrature rule on $[-1,1]$, whose application to the integrand $f$ will produce the same results as the function \verb|quadsingint| that you implemented in \ref{subprb:effquadsingint}.
  
  \cprotEnv \begin{solution}
   Using substitution (I). Let $(x_j, w_j)$, $j=1,\dots,n$ be the Gauss nodes and weights relative to the Gauss quadrature of order $n$ in the interval $[0,1]$. The nodes are mapped from $x_j$ in $[0,1]$ to $c_l$ for $l \in 1,\dots,2n$ in $[-1,1]$ as follows:
    \begin{align*}
     c_{2j-i}  = (-1)^i (1-x_j^2),\qquad j=1,\dots,n,\;i=0,1.  
      \end{align*}
    The weights $\tilde{w}_{l}, l = 1,\dots,2n$, become:
    \begin{align*}
     \tilde{w}_{2j-i} = 2 w_j x_j^2 \sqrt{2 - x_j^2},\qquad j=1,\dots,n,\;i=0,1.
    \end{align*}
    
   Using substitution (II). Let $(x_j, w_j)$, $j=1,\dots,n$ be the Gauss nodes and weights relative to the Gauss quadrature of order $n$ in the interval $[-1,1]$.  The nodes are mapped from $x_j$ to $c_j$  as follows:
    \begin{align*}
     c_j = \sin(x_j  \pi / 2),\qquad j=1,\dots,n
    \end{align*}
    The weights $\tilde{w}_{j}, j = 1,\dots,n$, become:
    \begin{align*}
     \tilde{w}_{j} = w_j \cos^2(x_j  \pi / 2) \pi / 2.
    \end{align*}
  \end{solution}

 \end{subproblem}

%  \begin{subproblem}[4]
%   What is the maximal degree of polynomials $f$, for which \verb|quadsingint|, in absence of round-off, will return the exact value of \eqref{eq:effquadsingint}?
%   
%    \begin{solution}
%     
%    \end{solution}
%  \end{subproblem}
 
 \begin{subproblem}[1]
  Tabulate the quadrature error:
   \begin{align*}
    \lvert W(f) - \texttt{quadsingint(f,n)} \rvert
   \end{align*}
   for $f(t) := \frac{1}{2+\exp(3t)}$ and $n = 1,2,...,25$. Estimate the $0 < q <
   1$ in the decay law of exponential convergence, see \lref{def:cvgtype}. 

  \cprotEnv   \begin{solution}
   The convergence is exponential with both methods. The \Cpp{} implementation is in \verb|quadsingint.cpp|.
   \end{solution}

 \end{subproblem}

 
\end{problem}
