\begin{problem}[Numerical integration of improper integrals] 

 We want to devise a numerical method for the computation of improper integrals of the form $\int_{-\infty}^{\infty} f(t) dt$ for continuous functions $f: \IR \rightarrow \IR$ that decay sufficiently fast for $\lvert t \rvert \rightarrow \infty$ (such that they are integrable on $\IR$). 
 
 A first option $(T)$ is the truncation of the domain to a bounded interval $[-b,b], b \leq \infty$, that is, we approximate:
 \begin{align*}
  \int_{-\infty}^{\infty} f(t) dt \approx \int_{-b}^{b} f(t) dt
 \end{align*}
 and then use a standard quadrature rule (like Gauss-Legendre quadrature) on $[-b,b]$.
 
 \begin{subproblem}[1]
  For the integrand $g(t) := 1 / (1 + t^2)$ determine $b$ such that the truncation error $E_T$ satisfies:
  \begin{align}
   E_T := \left\lvert  \int_{-\infty}^{\infty} g(t) dt - \int_{-b}^{b} g(t) dt \right \rvert \leq 10^{-6}
  \end{align}
  \begin{solution}
  An antiderivative of $g$ is $\mathrm{atan}$. The function $g$ is even.
  \begin{align}
   E_T = 2 \int_b^\infty g(t) dt = \lim_{x \rightarrow \infty} 2 \mathrm{atan}(x) - 2 \mathrm{atan}(b) = \pi - 2 \mathrm{atan}(b) \overset{!}{<} 10^{-6}
  \end{align}
  i.e. $b > \tan((\pi -10^{-6}) / 2) = \cot(10^{-6} / 2)$.
  \end{solution}
 \end{subproblem}
 
 \begin{subproblem}[1]
  What is the algorithmic difficulty faced in the implementation of the truncation approach for a generic integrand?
  \begin{solution}
   A good choice of $b$ requires a detailed knowledge about the decay of $f$, which may not be available for $f$ defined implicitly.
  \end{solution}
 \end{subproblem}

 A second option $(S)$ is the transformation of the improper integral to a bounded domain by substitution. For instance, we may use the map $t = \cot(s)$.
 
 \begin{subproblem}[2]
  Into which integral does the substitution $t = \cot(s)$ convert $\int_{-\infty}^{\infty} f(t) dt$?
  \begin{solution}
  \begin{align}
   &\frac{dt}{ds} = - (1+\cot^2(s)) = -(1+t^2) \\
   & \int_{-\infty}^{\infty} f(t) dt = - \int_\pi^0 f(\cot(s)) (1+\cot^2(s)) ds = \int_0^\pi \frac{f(\cot(s))}{\sin^2(s)} ds,
  \end{align}
  because $\sin^2(\theta) = \frac{1}{1+\cot^2(\theta)}$.
  \end{solution}
 \end{subproblem}
 
 \begin{subproblem}[1] \label{subprb:transformed_int_f_M}
  Write down the transformed integral explicitly for $g(t) := \frac{1}{1+t^2}$. Simplify the integrand.
  \begin{solution}
  \begin{align}
   \int_0^\pi \frac{1}{1 + \cot^2(s)} \frac{1}{\sin^2(s)} ds = \int_0^\pi \frac{1}{\sin^2(s) + \cos^2(s)} ds = \int_0^\pi ds = \pi
  \end{align}
  \end{solution}
 \end{subproblem}

 \begin{subproblem}[2] \label{subprb:transformed_int_cpp_impl}
  Write a \Cpp{} function:
  \begin{lstlisting}
   template <typename function>
   double quadinf(int n, const function &f);
  \end{lstlisting}
  that uses the transformation from \ref{subprb:transformed_int_f_M} together with $n$-point Gauss-Legendre quadrature to evaluate $\int_{-\infty}^{\infty} f(t) dt$. $f$ passes an object that provides an evaluation operator of the form:
  \begin{lstlisting}
   double operator() (double x) const;
  \end{lstlisting}
  \begin{hint}
   See \verb|quadinf_template.cpp|.
  \end{hint}
  \begin{hint}
   A lambda function with signature
  \begin{lstlisting}
  (double) -> double
  \end{lstlisting}
  automatically satisfies this requirement.
  \end{hint}
\cprotEnv \begin{solution}
   See \verb|quadinf.cpp|.
  \end{solution}
 \end{subproblem}

 \begin{subproblem}[1]
  Study the convergence as $n \rightarrow \infty$ of the quadrature method implemented in \ref{subprb:transformed_int_cpp_impl} for the integrand $h(t) := \exp(-(t-1)^2)$ (shifted Gaussian). What kind of convergence do you observe?
  \begin{hint}
  \begin{align}
   \int_{-\infty}^{\infty} h(t) dt = \sqrt{\pi}
  \end{align}
  \end{hint}
\cprotEnv \begin{solution}
   We observe exponential convergence. See \verb|quadinf.cpp|.
  \end{solution}
 \end{subproblem}

\end{problem}
 