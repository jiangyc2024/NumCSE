
\begin{problem}[Nested numerical quadrature]
 
 A laser beam has intensity
 \begin{align*}
  I(x,y) = \exp(- \alpha ((x-p)^2 + (y-q)^2) )
 \end{align*}
 on the plane orthogonal to the direction of the beam.
 
 \begin{subproblem}[1] \label{subprb:nested_quad_1}
  Write down the radiant power absorbed by the triangle
  \begin{align*}
   \bigtriangleup := \{(x,y)^T \in \IR^2 \; | \; x \geq 0, y \geq 0, x+y \leq 1 \}
  \end{align*}
  as a double integral.
  
  \begin{hint}
   The  radiant power absorbed by a surface is the integral of the intensity over the surface.
  \end{hint}
  
  \begin{solution}
   The radiant power absorbed by $\bigtriangleup$ can be written as:
   \begin{align*}
    \int_\bigtriangleup I(x,y) dx dy = \int_0^1 \int_0^{1-y} I(x,y) dx dy.
   \end{align*}
  \end{solution}
 \end{subproblem}

 \begin{subproblem}[3] \label{subprb:nested_quad_2}
  Write a \Cpp{} function 
  \begin{lstlisting}[language=c++]
template <class func>
double evalgaussquad(double a, double b, func&& f, const QuadRule & Q);
  \end{lstlisting}
  that evaluates an the $N$-point quadrature for an integrand passed in \texttt{f}
  in $[a,b]$. It should rely on the quadrature rule on the reference interval
  $[-1,1]$ that supplied through an object of type \texttt{QuadRule}. 
  (The vectors \verb|weights| and \verb|nodes| denote the weights and
  nodes of the reference quadrature rule respectively.)
  
  \begin{hint}
   Use the function \verb|gauleg| declared in \verb|gauleg.hpp| 
   and defined in \verb|gauleg.cpp| to compute nodes and weights in $[-1,1]$.
   See \ref{prb:effquadsing} for further explanations. 
  \end{hint}
  
  \begin{hint}
  You can use the template \verb|laserquad_template.cpp|.
  \end{hint}
  
  \cprotEnv \begin{solution}
   See \verb|laserquad.cpp| and \verb|CMakeLists.txt|.
  \end{solution}
 \end{subproblem}

 \begin{subproblem}[4] \label{subprb:nested_quad_3}
  Write a \Cpp{} function
  \begin{lstlisting}[language=c++]
template <class func>
double gaussquadtriangle(func&& f, int N)
  \end{lstlisting}
  for the computation of the integral
  \begin{align} \label{eq:subprb_nested_quad_2}
   \int_\bigtriangleup f(x,y) dx dy,
  \end{align}
  using nested $N$-point, 1D Gauss quadratures (using the functions \verb|evalgaussquad| of \ref{subprb:nested_quad_2} and  \verb|gauleg|).
  
  \begin{hint}
   Write \eqref{eq:subprb_nested_quad_2} explicitly as a double integral. Take particular care to correctly find the intervals of integration.
  \end{hint}
  
  \begin{hint}
   Lambda functions of \Cpp{} are well suited for this kind of implementation.
  \end{hint}

  
  \cprotEnv \begin{solution}
   The integral can be written as
   \begin{align*}
    \int_\bigtriangleup f(x,y) dx dy = \int_0^1 \int_0^{1-y} f(x,y) dx dy.
   \end{align*}
   In the \Cpp{} implementation, we define the auxiliary (lambda) function $f_y$:
   \begin{align*}
    \forall y \in [0,1], f_y: [1,1-y] \rightarrow \IR, x \mapsto f_y(x) := f(x,y)
   \end{align*}
   We also define the (lambda) approximated integrand:
   \begin{align}
    g(y) := \int_0^{1-y} f_y(x) dx \approx \frac{1}{1-y} \sum_{i = 0}^N w_i f_y\left(\frac{x_i + 1}{2}  (1-y)\right) =: \mathcal{I}(y),
   \end{align}
   the integral of which can be approximated, using a nested Gauss quadrature:
   \begin{align}
    \int_\bigtriangleup f(x,y) dx dy  = \int_0^1  \int_0^{1-y} f_y(x) dx dx = \int_0^1 g(y) dy \approx \frac{1}{2} \sum_{j = 1}^N w_j \mathcal{I}\left(\frac{y_j + 1}{2}\right).
   \end{align}
   The implementation can be found in \verb|laserquad.cpp|.
  \end{solution}

 \end{subproblem}

 \begin{subproblem}[1]
  Apply the function \verb|gaussquadtriangle| of \ref{subprb:nested_quad_3} to the subproblem \ref{subprb:nested_quad_1} using the parameter $\alpha = 1, p = 0, q = 0$. Compute the error w.r.t to the number of nodes $N$. What kind of convergence do you observe? Explain the result.
  
  \begin{hint}
   Use the ``exact'' value of the integral $0.366046550000405$.
  \end{hint}
  
  \cprotEnv \begin{solution}
  As one expects from theoretical considerations, the convergence is exponential. The implementation can be found in \verb|laserquad.cpp|.
  \end{solution}


 \end{subproblem}

 
\end{problem}
