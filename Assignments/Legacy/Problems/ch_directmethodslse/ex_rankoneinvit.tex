\renewcommand{\chpt}{ch_directmethodslse \coreproblem}

\begin{problem}[Rank-one perturbations \coreproblem] \label{prb:rankoneinvit}
This problem is another application of the Sherman-Morrison-Woodbury formula, see
\lref{lem:SMW}: please revise \lref{par:smw} of the lecture carefully.

Consider the \matlab{} code given in Listing~\ref{mc:rankoneinvit}.
\lstinputlisting[caption={Matlab function \texttt{rankoneinvit}}, label={mc:rankoneinvit}]{\problems/ch_directmethodslse/MATLAB/rankoneinvit.m}

\begin{subproblem}[2]
    Write an equivalent implementation in \Eigen{} of the Matlab function \texttt{rankoneinvit}. The C++ code should use exactly the same operations.
  \begin{solution}
See file \texttt{rankoneinvit.cpp}.
    \end{solution}
 Do not expect top understand what is the purpose of the function.
\end{subproblem}

\begin{subproblem}[1]
  What is the asymptotic complexity of the loop body of the function
  \texttt{rankoneinvit}? More precisely, you should look at the asymptotic
  complexity of the code in the lines 8-12 of Listing~\ref{mc:rankoneinvit}.
\begin{solution}
The total asymptotic complexity is dominated by the solution of the linear system with matrix $M$ done in line 10, which has asymptotic complexity of $O(n^3)$.
\end{solution}
\end{subproblem}

\begin{subproblem}[3]
 Write an efficient implementation in \Eigen{} of the loop body, possibly with optimal asymptotic complexity. Validate it by comparing the result with the other implementation in \Eigen{}.

 \begin{hint}
   Take the clue from \lref{smw}.
 \end{hint}

 \begin{solution}
   See file \texttt{rankoneinvit.cpp}.
 \end{solution}
 
\end{subproblem}

\begin{subproblem}[1]
 What is the asymptotic complexity of the new version of the loop body?
  \begin{solution}
The loop body of the C++ function \texttt{rankoneinvit\_fast} only consists in vector-vector multiplications, and so the asymptotic complexity is $O(n)$.
     \end{solution}
\end{subproblem}

\begin{subproblem}[2]
Tabulate the runtimes of the two \emph{inner loops} of the C++ implementations with different vector sizes
{$n=2^{k}$}, {$k=1,2,3,\ldots,9$}. Use, as test vector
\begin{verbatim}
 Eigen::VectorXd::LinSpaced(n,1,2)
\end{verbatim}
How can you read off the asymptotic
complexity from these data?

\begin{hint}
  Whenever you provide figure from runtime measurements, you have to specify
  the operating system and compiler (options) used. 
\end{hint}

\begin{solution}
  See file \texttt{rankoneinvit.cpp}.
\end{solution}
\end{subproblem}
\end{problem}
