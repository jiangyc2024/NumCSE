% ncse_new/p1_SystemsOfEquations/ch3_IterativeMethodsNonLinear/ex_RecursionOrder.tex
\renewcommand{\chpt}{ch_iterativenonlinear}

\begin{problem}[Order of convergence from error recursion \coreproblem] \label{prb:RecursionOrder}

In \lref{ex:sec} we have observed \emph{fractional} orders of convergence ($\to$
\ncsedef{def:cvgord}) for both the secant method, see \ncseref[Code]{mc:secant},
and the quadratic inverse interpolation method. This is fairly typical for 2-point
methods in 1D and arises from the underlying recursions for error bounds. The
analysis is elaborated for the secant method in \lref{rem:secfraccvg}, where a
linearized error recursion is given in \lref{eq:secanterrprop}.

Now we suppose the recursive bound for the norms of the iteration errors
\begin{gather}    
  \label{eq:prb:RecursionOrder}
\|e^{(n+1)}\|\leq \|e^{(n)}\| \sqrt{\|e^{(n-1)}\|}\;,
\end{gather}
where $e^{(n)}=x^{(n)}-x^{*}$ is the error of $n$-th iterate. 

\begin{subproblem}[1]\label{prb:RecursionOrder_1}
Guess the maximal order of convergence of the method from a numerical experiment
conducted in MATLAB. \\
\hint \ncserem{rem:eoc}

\begin{solution}
See Listing~\ref{mc:recursionorder}.
\lstinputlisting[caption={MATLAB script for Sub-problem
    ~\ref{prb:RecursionOrder_1}}, label={mc:recursionorder}]{\problems/ch_iterativenonlinear/MATLAB/recursionorder.m}
  \end{solution}
  \end{subproblem}

\begin{subproblem}[4]\label{prb:RecursionOrder_2}
Find the maximal
guaranteed order of convergence of this method through analytical considerations.

\begin{hint}
 First of all note that we may assume equality in both the error
recursion \eqref{eq:prb:RecursionOrder} and the bound $\|e^{(n+1)}\|\leq C\|e^{(n)}\|^{p}$ that
defines convergence of order $p>1$, because in both cases equality corresponds to a
worst case scenario. Then plug the two equations into each other and obtain
an equation of the type $\ldots = 1$, where the left hand side involves an 
error norm that can become arbitrarily small. This implies a condition on
$p$ and allows to determine $C>0$. A formal proof by induction (not required)
can finally establish that these values provide a correct choice. 
\end{hint}



\begin{solution}
Suppose $\|e^{(n)}\|=C \|e^{(n-1)}\|^{p}$ ($p$ is the largest convergence order and C is some constant).\\
Then
\begin{gather}
  \label{RO:1}
  \|e^{(n+1)}\|=C\|e^{(n)}\|^{p}
  =C(C\|e^{(n-1)}\|^p)^{p}
  =C^{p+1}\|e^{(n-1)}\|^{p^2}
\end{gather}
In \eqref{eq:prb:RecursionOrder} we may assume equality, because this is the worst case.
Thus,
\begin{gather}
  \label{RO:2}
\|e^{(n+1)}\| = \|e^{(n)}\|\cdot \|e^{(n-1)}\|^\frac{1}{2}
=C\|e^{(n-1)}\|^{p+\frac{1}{2}}
\end{gather}
Combine \eqref{RO:1} and \eqref{RO:2},
$$C^{p+1}\|e^{(n-1)}\|^{p^2} = C\|e^{(n-1)}\|^{p+\frac{1}{2}}$$
i.e.
 \begin{gather}
   \label{RO:3}
 C^{p}\|e^{(n-1)}\|^{p^2-p-\frac{1}{2}} = 1.
 \end{gather}
Since \eqref{RO:3} holds for each $n\geq 1$, we have \\
$$p^2-p-\frac{1}{2}=0$$
$$p=\frac{1+\sqrt{3}}{2} ~~~~~{\rm or}~~~~~~p=\frac{1-\sqrt{3}}{2}~
{\rm(dropped)}.$$
For $C$ we find the maximal value $1$.
 \end{solution}
  \end{subproblem}
  \end{problem}