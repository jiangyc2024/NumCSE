\begin{problem}[Piecewise cubic Hermite interpolation]
 
 Piecewise cubic Hermite interpolation with exact slopes on a mesh
 \[
\mathcal{M} := \{ a = x_0 < x_1 < \dots < x_n = b \}
 \]
 was defined in \ncseref{sec:HermiteInterpolation}. For $f \in C^4([a,b])$ it
 enjoys $h$-convergence with rate $4$ as we have seen in \lref{ex:hermcvg1}.
 
 Now we consider cases, where perturbed or reconstructed slopes are used. For
 instance, this was done in the context of monotonicity preserving piecewise cubic
 Hermite interpolation as discussed in \lref{sec:hipshp}.
 
 \begin{subproblem}[2] \label{subprb:pchi_1}
  Assume that piecewise cubic Hermite interpolation is based on perturbed slopes, that is, the piecewise cubic function $s$ on $\mathcal{M}$ satisfies:
  \begin{align*}
   s(x_j) = f(x_j)\quad,\quad s'(x_j) = f'(x_j) + \delta_j,
  \end{align*}
  where the $\delta_j$ may depends on $\mathcal{M}$, too.
  
  Which rate of asymptotic $h$-convergence of the $\sup$-norm of the approximation error can
  be expected, if we know that for all $j$
  
  \begin{align*}
   \lvert \delta_j \rvert = O(h^\beta)\;,\quad \beta \in \IN_0\;,
  \end{align*}
  for mesh-width $h \rightarrow 0$.
  
  \begin{hint}
   Use a local generalized cardinal basis functions, cf. \ncseref{par:pchiploc}.
  \end{hint}
  
  \begin{solution}
   Let $s$ be the piecewise cubic polynomial interpolant of $f$. We can rewrite $s$ using the local representation with cardinal basis:
   \begin{align*}
    s(t) & = y_{i-1} H_1(t) + y_i H_2(t) + c_{i-1} H_3(t) + c_i H_4(t) \\
     & = y_{i-1} H_1(t) + y_i H_2(t) + (f'(t_{i-1}) + \delta_{i-1}) H_3(t) + (f'(t_i) + \delta_i) H_4(t) \\
     & = y_{i-1} H_1(t) + y_i H_2(t) + f'(t_{i-1}) H_3(t) + f'(t_i)  H_4(t) + \delta_{i-1} H_3(t) + \delta_i H_4(t)
   \end{align*}
   Hence, if we denote by $\tilde{s}$ the Hermite interpolant with exact slopes:
   \begin{align*}
    \lVert f - s \rVert_{L^\infty([a,b])} & \leq \lVert f - \tilde{s} + \tilde{s} - s \rVert_{L^\infty([a,b])} \leq \lVert f - \tilde{s} \rVert_{L^\infty([a,b])} + \lVert \tilde{s} - s \rVert_{L^\infty([a,b])} \\
    & \leq O(h^4) + \max_{i} \lVert \delta_{i-1} H_3(t) + \delta_i H_4(t) \rVert_{L^\infty([t_{i-1},t_i])} \\
    & = O(h^4) + O(h^{b+1}) = O(\min(4,\beta + 1))
   \end{align*}
  since $\lVert H_3(t) \rVert_{L^\infty([t_{i-1},t_i])} = \lVert H_4(t) \rVert_{L^\infty([t_{i-1},t_i])} = O(h)$ (attain maximum at $t = \frac{1}{3h}(t_{i} - t)$ resp. minimum at $t = \frac{2}{3h}(t_{i} - t)$, with value $h (\left(\frac{2}{3}\right)^3 - \left(\frac{2}{3}\right)^2)$).
  \end{solution}

 \end{subproblem}

 \begin{subproblem}[3]
  Implement a strange piecewise cubic interpolation scheme in \Cpp{} that satisfies:
  \begin{align*}
   s(x_j) = f(x_j)\quad,\quad s'(x_j) = 0
  \end{align*}
  and empirically determine its convergence on a sequence of equidistant meshes of $[-5,5]$ with mesh-widths $h = 2^{-l}, l = 0,\dots,8$ and for the interpoland $f(t) := \frac{1}{1+t^2}$.
  
  As a possibly useful guideline, you can use the provided \Cpp{} template, see the file \verb|piecewise_hermite_interpolation_template.cpp|.
   
  Compare with the insight gained in \ref{subprb:pchi_1}.
  
\cprotEnv \begin{solution}
   According to the previous subproblem, since $s'(x_j) = f'(x_j) - f'(x_j)$, i.e. $\lvert \delta_j \rvert = O(1)$, $\beta = 0$, the convergence order is limited to $O(h)$.
   
   For the \Cpp{} solution, cf. \verb|piecewise_hermite_interpolation.cpp|.
  \end{solution}
 \end{subproblem}

 \begin{subproblem}[1]
   Assume equidistant meshes and reconstruction of slopes by a particular
   averaging. More precisely,
   the $\mathcal{M}$-piecewise cubic function $s$ is to satisfy the generalized 
   interpolation conditions
  \begin{align*}
   s(x_j) & = f(x_j), \\
   s'(x_j) & = \begin{cases}
              \frac{-f(x_2) + 4f(x_1) - 3f(x_0)}{2h} &\text{for } j = 0\;, \\
              \frac{f(x_{j+1}) - f(x_{j-1})}{2h} &\text{for } j = 1,\dots,n-1\;, \\
              \frac{3f(x_n) - 4f(x_{n-1}) + f(x_{n-2})}{2h} &\text{for } j = n\;. \\
             \end{cases}
  \end{align*}
  What will be the rate of $h$-convergence of this scheme (in $\sup$-norm)?
  
  (You can solve this exercise either theoretically or  determine an empiric convergence rate in a numerical experiment.)
  
  \begin{hint}
    If you opt for the theoretical approach, you can
    use what you have found in \autoref{subprb:pchi_1}. To find perturbation
    bounds, rely on the Taylor expansion formula with remainder, see
    \lref{ex:taylorcancel}. 
  \end{hint}
  
\cprotEnv \begin{solution}
   First, we show that the approximation $s'(x_j) = f'(x_j) + O(h^2)$. This follows from Taylor expansion:
   \begin{align*}
    f(x) = f(x_j) + f'(x_j) ( x - x_j ) + f''(x_j) (x - x_j)^2 / 2 + O(h^3)
   \end{align*}
  Using $x = x_{j-1}$ and $x = x_{j+1}$ (and $h = x_{j+1}- x_j$):
   \begin{align*}
    \frac{f(x_{j+1}) - f(x_{j})}{h} = f'(x_j) + f''(x_j) h / 2 + O(h^2) \\
    \frac{f(x_{j-1}) - f(x_{j})}{h} = -f'(x_j) + f''(x_j) h / 2 + O(h^2)
   \end{align*}
   Subtracting the second equation to the first equation:
   \begin{align*}
    \frac{f(x_{j+1}) - f(x_{j-1})}{h} = 2f'(x_j) + O(h^2)
   \end{align*}
   For the one-sided difference we expand at $x = x_{j+2}$ and $x = x_{j+1}$:
   \begin{align*}
    \frac{f(x_{j+1}) - f(x_{j})}{h} = f'(x_j) + f''(x_j) h / 2 + O(h^2) \\
    \frac{f(x_{j+2}) - f(x_{j})}{2h} = f'(x_j) + f''(x_j) h + O(h^2)
   \end{align*}
   Subtracting the first equation to half of the second equation:
   \begin{align*}
    \frac{f(x_{j+2}) - f(x_{j})}{4h} - \frac{f(x_{j+1}) - f(x_{j})}{h} = f'(x_j) (1/2 - 1) + O(h^2) \\
    \frac{f(x_{j+1}) - f(x_{j}) - 4 f(x_{j+1}) + 4 f(x_{j})}{4h} = f'(x_j) (1/2 - 1) + O(h^2) \\
    \frac{- f(x_{j+2}) + 4 f(x_{j+1}) - 3 f(x_{j})}{2h} = f'(x_j) + O(h^2) \\
   \end{align*}
   The other side is analogous.

   According to the previous subproblem, since $s'(x_j) = f'(x_j) + O(h^2)$ and $\beta = 2$, the convergence order is limited to $O(h^3)$.
   
   For the \Cpp{} solution, cf. \verb|piecewise_hermite_interpolation.cpp|.
  \end{solution}
 \end{subproblem}

 
\end{problem}
