% ncse_new/\problems/ch_piecewisepolynomials/ex_CubicSplines.tex
%  solution:  ex_CubicSpline.m  ex_CubicSpline.eps

\begin{problem}[Natural cubic Splines \coreproblem] \label{prb:NatCubicSplines}
  In \lref{sec:csi} we learned about cubic spline interpolation and its variants,
  the complete, periodic, and natural cubic spline interpolation schemes. 
  
  \begin{subproblem}[1]
    \label{csi:sp1}
    Given a knot set {$\Ct=\{t_{0}<t_{1}<\dots<t_{n}\}$}, which also serves
    as the set of interpolation nodes, and values {$y_{j}$},
    {$j=0,\ldots,n$}, write down the linear system of equations that yields
    the slopes $s'(t_{j})$ of the natural cubic spline interpolant $s$ of the data
    points $(t_{j},y_{j})$ at the knots.
    
  \begin{solution}
   Let $h_i := t_i - t_{i-1}$. Given the natural condition on the spline, one can remove the columns relative to $c_0 := s'(t_0)$ and $c_n := s'(t_n)$ from the system matrix, which becomes:
   \newcommand{\squarelineheight}{0.9ex}
   \begin{align}
    \vec{A} & := \begin{pmatrix}
                2 / h_1 & 1 / h_1 & 0 & 0 & \cdots & & 0 \\[\squarelineheight]
                b_0 & a_1 & b_1 & 0 & \cdots & & 0 \\[\squarelineheight]
                0 & b_1 & a_2 & b_2 & \ddots & & 0 \\[\squarelineheight]
                \vdots & & \ddots & \ddots & \ddots & & \vdots \\[\squarelineheight]
                \vdots & &  & b_{n-3} & a_{n-2} & b_{n-2} & 0 \\[\squarelineheight]
                0 & & \cdots & 0 & b_{n-2} & a_{n-1} & b_{n-1} \\[\squarelineheight]
                0 & & \cdots & \cdots & 0 & 1 / h_n & 2 / h_n \\
               \end{pmatrix}, a_i := \frac{2}{h_i} + \frac{2}{h_{i+1}}, b_i := \frac{1}{h_{i+1}} \\
    \vec{c} & := [ c_0, c_1, \dots, c_{n} ] \\
    \vec{b} & := [ r_0, \dots, r_{n} ]
   \end{align}
  Where $c_i := s'(t_i)$. We define $r_i := 3 \left( \frac{y_{i} - y_{i-1}}{h_{i}^2} + \frac{y_{i+1} - y_{i}}{h_{i+1}^2} \right), i = 1, \dots, n-1$, and
  \begin{align*}
   r_0 = 3 \frac{y_1 - y_0}{h_1^2}, r_n = 3 \frac{y_n - y_{n-1}}{h_n^2}
  \end{align*}
  The system becomes $\vec{A}\vec{c} = \vec{b}$.
  \end{solution}
  \end{subproblem}
  


  \begin{subproblem}[3]
    \label{csi:sp2}
    Argue why the linear system found in \autoref{csi:sp1} has a unique solution.
    
    \begin{hint}
    Look up \lref{lem:ddomspd} and apply its assertion.
    \end{hint}
    
  \begin{solution}
    Notice that $a_i := \frac{2}{h_i} + \frac{2}{h_{i+1}} > \frac{1}{h_{i+1}} + \frac{1}{h_{i}} =: b_i + b_{i-1}$. The matrix is (strictly) diagonally dominant and, therefore, invertible.
  \end{solution}
  \end{subproblem}

  \begin{subproblem}[3]
  Based on \Eigen{} devise an \emph{efficient} implementation of a \Cpp{} class for the 
  computation of a natural cubic spline interpolant with the following definition:
  \begin{lstlisting}[language=c++]
class NatCSI {
public:
    //! \brief Build the cubic spline interpolant with natural boundaries
    //! Setup the data structures you need.
    //! Pre-compute the coefficients of the spline (solve system)
    //! \param[in] t, nodes of the grid (for pairs (t_i, y_i)) (sorted!)
    //! \param[in] y, values y_i at t_i (for pairs (t_i, y_i))
    NatCSI(const const std::vector<double> & t, const const std::vector<double> & y);
    
    //! \brief Interpolant evaluation at x
    //! \param[in] x, value x where to evaluate the spline
    //! \return value of the spline at x
    double operator() (double x) const;
    
private:
    // TODO: store data for the spline
};
\end{lstlisting}
\begin{hint}
Assume that the input array of knots is sorted 
and perform binary searches for the evaluation of the interpolant.
\end{hint}
\end{subproblem}

\cprotEnv \begin{solution}
 See \verb|natcsi.cpp|.
\end{solution}
\end{problem}




