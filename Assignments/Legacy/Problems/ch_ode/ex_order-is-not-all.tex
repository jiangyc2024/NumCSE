\begin{problem}[Order is not everything \coreproblem]\label{prb:order-is-not-all} 
In \lref{sec:ssmcvg} we have seen that Runge-Kutta single step methods when
applied to initial value problems with sufficiently smooth solutions will converge
algebraically (with respect to the maximum error in the mesh points) with a rate
given by their intrinsic order, see \lref{def:ssmord}. 

In this problem we perform empiric investigations of orders of convergence of
several explicit Runge-Kutta single step methods. We rely on two IVPs, one of
which has a perfectly smooth solution, whereas the second has a solution that is
merely piecewise smooth. Thus in the second case the smoothness assumptions of the
convergence theory for RK-SSMs might be violated and it is interesting to study
the consequences. 

\begin{subproblem}[3]\label{sp:errors}
Consider the autonomous ODE
\begin{equation}
       \label{eq:general-ode}
	\dot{\Vy} = \Vf(\Vy),\quad \Vy(0) = \Vy_0,
\end{equation}
where $\Vf\colon \R^n\to\R^n$ and $\Vy_0\in\R^n$. Using the class \texttt{RKIntegrate} of \ref{prb:RK3} write a C++ function
\begin{lstlisting}
template <class Function>
void errors(const Function &f, const double &T, const Eigen::VectorXd &y0, const Eigen::MatrixXd &A,
const Eigen::VectorXd &b)
\end{lstlisting}
that computes an approximated solution $\Vy_N$ of \eqref{eq:general-ode} up to
time $T$ by means of an explicit Runge-Kutta method with $N=2^k$, $k=1,\dots,15$,
uniform timesteps. The method is defined by the Butcher scheme described by the
inputs \texttt{A} and \texttt{b}. The input \texttt{f} is an object with an
evaluation operator (e.g.\ a lambda function) for arguments of type
\lstinline{const VectorXd &} representing $\Vf$. The input \texttt{y0} passes the
initial value $\Vy_0$.

For each $k$, the function should show the error at the final point
$E_N = \norm{ \Vy_N(T)-\Vy_{2^{15}}(T)}$, $N=2^{k}$, $k=1,\ldots,13$, accepting
$\Vy_{2^{15}}(T)$ as exact value. Assuming algebraic convergence for
$E_N\approx C N^{-r}$, at each step show an approximation of the order of
convergence $r_k$ (recall that $N=2^k$). This will be an expression involving
$E_N$ and $E_{N/2}$.

Finally, compute and show an approximate order of convergence by averaging the relevant
$r_N$s (namely, you should take into account the cases before machine precision is
reached in the components of $\Vy_N(T)-\Vy_{2^{15}}(T)$).
\begin{solution}
Let us find an expression for the order of convergence. Set $N_k=2^k$. We readily derive
\[
\frac{E_{N_{k-1}}}{E_{N_{k}}}\approx \frac{C 2^{-(k-1) r_k }}{C 2^{-k r_k}}=2^{r_k},\qquad r_k\approx \log\left( \frac{E_{N_{k-1}}}{E_{N_{k}}}\right) / \log(2).
\]
A reasonable approximation of the order of convergence is given by
\begin{equation}\label{eq:r}
r \approx \frac{1}{\# K}\sum_{k\in K} r_k,\qquad K=\{k=1,\dots,15:E_{N_k} >5n\cdot 10^{-14}\}.
\end{equation}
See file \texttt{errors.hpp} for the implementation.
\end{solution}
\end{subproblem}

\begin{subproblem}[3]\label{sp:analytic}
Calculate the analytical solutions of the logistic ODE (see \lref{ex:logode})
\begin{equation}
	\label{eq:logistic}
	\dot{y} =(1-y)y,\quad y(0) = 1/2,
\end{equation}
and of the initial value problem 
\begin{equation}
	\label{eq:nichtdiff}
	\dot{y} = |{1.1-y}| + 1,\quad y(0) = 1.
\end{equation}

\begin{solution}
As far as \eqref{eq:logistic} is concerned, the solution is $y(t)=(1+e^{-t})^{-1}$ (see \lref{eq:logodesol}).

Let us now consider \eqref{eq:nichtdiff}. Because of the absolute value on the right hand side of the differential equation, we have to distinguish two cases $y(t)<1.1$ and $y(t)>1.1$. Since the initial condition is given by $y(0) = 1< 1.1$, we start with the case $y(t)<1.1$. For $y(t)<1.1$, the differential equation is $\dot{y} = 2.1 - y$. Separation of variables
\begin{equation*}
	\int_1^{y(t)}\frac{1}{2.1-\tilde{y}} d\,\tilde{y} = \int_0^t d\,\tilde{t}
\end{equation*}
yields the solution
\begin{equation*}
	y(t) = 2.1 - 1.1 e^{-t},\quad \text{for $y(t) < 1.1$}.
\end{equation*}
For $y(t) > 1.1$, the differential equation is given by $\dot{y} = y-0.1$ with initial condition $y(\ln(\frac{11}{10})) = 1.1$, where the initial time $t^*$ was derived from the condition $y(t^*) = 2.1 - 1.1 e^{-t^*}\overset{\text{!}}{=} 1.1$. 
Separation of variables yields the solution for this IVP
\begin{equation*}
	y(t) = \tfrac{10}{11} e^t + 0.1.
\end{equation*}

Together, the solution of the initial IVP is given by
\begin{equation*}
	y(t) =
	\begin{cases}
	  2.1 - 1.1 e^{-t},&\quad \text{for $t \leq \ln(1.1)$}\\
	  \tfrac{10}{11} e^t + 0.1,&\quad \text{for $t > \ln(1.1)$.}
	\end{cases}
\end{equation*}
\end{solution}
\end{subproblem}



\begin{subproblem}[2] \label{sp:numerical}
Use the function \texttt{errors} from \ref{sp:errors} with the ODEs \eqref{eq:logistic}  and \eqref{eq:nichtdiff} and the methods:
\begin{itemize}\setlength\itemsep{0pt}
	\item the explicit Euler method, a RK single step method of order $1$,
	\item the explicit trapezoidal rule, a RK single step method of order $2$,
	\item an RK method of order $3$ given by the Butcher tableau
	\begin{center}
	\begin{tabular}[c]{c|ccc}
		$0$   &       &       &       \\
		$1/2$ & $1/2$ &       &       \\
		$1$   & $-1$  & $2$   &       \\\hline
		& $1/6$ & $2/3$ & $1/6$ \\
	\end{tabular}
	\end{center}
	\item the classical RK method of order $4$.
\end{itemize}
(See \lref{ex:rkexpex} for details.) Set $T=0.1$.

Comment the calculated order of convergence for the different methods and the two different ODEs.
\begin{solution}
Using the expression for the order of convergence given in \eqref{eq:r} we find for \eqref{eq:logistic}:
\begin{itemize}\setlength\itemsep{0pt}
	\item Eul: 1.06
	\item RK2: 2.00
	\item RK3: 2.84
	\item RK4: 4.01
\end{itemize}
This corresponds to the expected orders. However, in the case of the ODE \eqref{eq:nichtdiff} we obtain
\begin{itemize}\setlength\itemsep{0pt}
	\item Eul: 1.09
	\item RK2: 1.93
	\item RK3: 1.94
	\item RK4: 1.99
\end{itemize}
The convergence orders of the explicit Euler and Runge--Kutta $2$ methods are as expected, but we do not see any relevant improvement in the convergence orders of RK3 and RK4. This is due to the fact that the right hand side of the IVP is not continuously differentiable: the convergence theory breaks down.

See file \texttt{order\_not\_all.cpp} for the implementation.
\end{solution}
\end{subproblem}

\end{problem}
