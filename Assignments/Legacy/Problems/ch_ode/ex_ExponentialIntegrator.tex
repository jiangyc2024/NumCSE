\begin{problem}[Exponential integrator]\label{prb:ExponentialIntegrator}

A modern class of single step methods developed for special initial value
problems that can be regarded as perturbed linear ODEs are the exponential
integrators, see 
\begin{quote}
  {\sc M.~Hochbruck and A.~Ostermann}, {\em Exponential integrators}, Acta
  Numerica, 19 (2010), pp.~209--286.
\end{quote}
These methods fit the concept of single step methods as introduced in 
\ncsedef{def:esv} and, usually, converge algebraically according to
\ncseeq{eq:ssmerrest}.

A step with size $h$ of the so-called \textit{exponential Euler} single step
method for the ODE $\dot\Vy=\Vf(\Vy)$ with continuously differentiable
$\Vf:\IR^d\rightarrow\IR^d$ reads
\begin{equation} \label{eq:ExponentialIntegrator_ExpEul} \Vy_1 =
  \Vy_0+h\;\varphi\big(h\Derv\Vf(\Vy_0)\big)\:\Vf(\Vy_0), \end{equation} where
$\Derv\Vf(\Vy)\in\bbR^{d,d}$ is the Jacobian of $\Vf$ at $\Vy\in\bbR^{d}$, and the
matrix function $\varphi:\IR^{d,d}\rightarrow\IR^{d,d}$ is defined as
$\varphi(\VZ) = (\mathrm{exp}(\VZ)-\Id)\:\VZ^{-1}$.  Here $\exp(\VZ)$ is the
matrix exponential of $\VZ$, a special function
$\exp:\IR^{d,d}\rightarrow\IR^{d,d}$, see \lref{eq:linodesolexp}. 

The function $\varphi$ is implemented in the provided file
\texttt{ExpEul\_template.cpp}. When plugging in the exponential series, it is
clear that the function $z\mapsto \varphi(z):=\frac{\exp(z)-1}{z}$ is analytic on
$\bbC$. Thus, $\varphi(\VZ)$ is well defined for all matrices $\VZ\in\bbR^{d,d}$. 
%%%%%%%%%%%% SUBPROBLEM 0



\begin{subproblem}[2] \label{subprb:ExponentialIntegrator_1}
Is the exponential Euler single step method defined in
\eqref{eq:ExponentialIntegrator_ExpEul} consistent with the ODE $\dot\Vy=\Vf(\Vy)$
(see \lref{def:consssm})? Explain your answer. 
\begin{solution}
In view of \eqref{eq:ExponentialIntegrator_ExpEul}, consistency is equivalent to $\varphi\big(h\Derv\Vf(\Vy_0)\big)\:\Vf(\Vy_0)=\Vf(\Vy_0)$ for $h=0$. Since $\varphi$ is not defined for the zero matrix, this should be intended in the limit as $h\to 0$. By definition of the matrix exponential we have $e^{hZ}=\Id+ hZ +O(h^2)$ as $h\to 0$. Therefore
\[
\varphi(h\Derv\Vf(\Vy_0))= (\mathrm{exp}(h\Derv\Vf(\Vy_0))-\Id)(h\Derv\Vf(\Vy_0))^{-1}=
 (\Derv\Vf(\Vy_0))+O(h))\Derv\Vf(\Vy_0)^{-1},
\]
whence
\[
\lim_{h\to 0}\, \varphi\big(h\Derv\Vf(\Vy_0)\big)\:\Vf(\Vy_0)=\Vf(\Vy_0),
\]
as desired.
\end{solution}
\end{subproblem}

%%%%%%%%%%%% SUBPROBLEM 1


\begin{subproblem}[2] 
\label{subprb:ExponentialIntegrator_1}
Show that the exponential Euler single step method defined in \eqref{eq:ExponentialIntegrator_ExpEul} solves the linear initial value problem 
$$\dot\Vy = \VA\;\Vy\;,\quad \Vy(0) = \Vy_{0}\in\bbR^{d}\;,\qquad \VA\in\IR^{d,d}\;,$$
exactly.

\begin{hint}
  Recall \ncseex{eq:linodesolexp}; the solution of the IVP is
  $\Vy(t)= \mathrm{exp}(\VA t)\Vy_{0}$. To facilitate formal calculations, you may
  assume that $\VA$ is regular.
\end{hint}

\begin{solution}
Given $\Vy(0)$, a step of size $t$ of the method gives
\begin{align*}
\Vy_1 & = \Vy(0) + t\; \varphi\big(t\; \Derv\Vf(\Vy(0)) \big)\;\Vf(\Vy(0))
\\&= \Vy(0) + t\; \varphi\big(t\; \VA \big)\;\VA\; \Vy(0) \\
& = \Vy(0) + t\; \big(\mathrm{exp}(\VA t) - \Id\big) \;(t\VA)^{-1}\; \VA\; \Vy(0)
\\& = \Vy(0) + \mathrm{exp}(\VA t)\;\Vy(0) - \Vy(0) \\
& = \mathrm{exp}(\VA t)\; \Vy(0)
\\ & = \Vy(t).
\end{align*}
\end{solution}
\end{subproblem}

%%%%%%%%%%%% SUBPROBLEM 1.5


\begin{subproblem}[2] 
Determine the region of stability of the exponential Euler single step method
defined in \eqref{eq:ExponentialIntegrator_ExpEul} (see \lref{def:stabdom}).

\begin{solution}
The discrete evolution $\Psibf^h$ associated to \eqref{eq:ExponentialIntegrator_ExpEul} is given by
\[
\Psibf^h(y) = y+h\varphi(h\Derv\Vf(\Vy))\Vf(\Vy).
\]
Thus, for a scalar linear ODE of the form $\dot y=\lambda y$ for $\lambda\in\C$ we have ($f(y)=\lambda y$, $Df(y)=\lambda$)
\[
\Psibf^h(y)=y+h\frac{e^{h\lambda}-1}{h\lambda}\lambda y = e^{h\lambda} y.
\]
Thus, the stability function $S\colon \C\to\C$ is given by $S(z)=e^z$. Therefore, the region of stability of this method is
\[
\mathcal{S}_{\Psibf} =\{z\in\C:|e^z|<1\}=\{z\in\C:e^{\Re z}<1\}=\{z\in\C:\Re z < 0\}.
\]
It is useful to compare this with \lref{ex:stabdom}.

Alternatively, one may simply appeal to \ref{subprb:ExponentialIntegrator_1}
to see that the method has the ``ideal'' region of stability as introduced in 
\lref{par:idealsd}.  
\end{solution}
\end{subproblem}

%%%%%%%%%%%% SUBPROBLEM 2

\begin{subproblem}[2] \label{subprb:ExponentialIntegrator_2}
Write a C++ function
\begin{lstlisting}
template <class Function, class Function2>
Eigen::VectorXd ExpEulStep(Eigen::VectorXd y0, Function f, Function2 df, double h)
\end{lstlisting}
that implements \eqref{eq:ExponentialIntegrator_ExpEul}.
Here \texttt{f} and \texttt{df} are objects with evaluation operators representing the ODE right-hand side function $\Vf:\IR^d\rightarrow\IR^d$ and its Jacobian, respectively.

\begin{hint}
Use the supplied template \texttt{ExpEul\_template.cpp}.
\end{hint}

\begin{solution}
See \texttt{ExpEul.cpp}.
\end{solution}
\end{subproblem}

%%%%%%%%%%%% SUBPROBLEM 3

%\andrea{fix who is column or row? I think this way works}

\begin{subproblem}[3] \label{subprb:ExponentialIntegrator_3}
What is the order of the single step method \eqref{eq:ExponentialIntegrator_ExpEul}?
To investigate it, write a C++ routine
that applies the method to the scalar logistic ODE
$$  \dot y  = y\;(1-y)\;,\quad y(0) = 0.1\;, $$
in the time interval $[0,1]$.  Show the error at the final time against the stepsize $h=T/N$, $N=2^k$ for $k=1,\dots,15$. As in \ref{prb:order-is-not-all} in \ref{PS12}, for each $k$ compute and show an approximate order of convergence.


\begin{hint}
The exact solution is
$$ y(t) = \frac{y(0)}{y(0)+\big(1-y(0)\big)e^{-t}}.$$
\end{hint}

\begin{solution}
Error $= O(h^2)$. See \texttt{ExpEul.cpp}.
\end{solution}
\end{subproblem}

\end{problem}
