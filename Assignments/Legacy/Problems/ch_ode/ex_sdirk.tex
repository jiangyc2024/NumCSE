\begin{problem}[Singly Diagonally Implicit Runge-Kutta Method]\label{prb:sdirk} 
  SDIRK-methods (\textbf{S}ingly \textbf{D}iagonally
  \textbf{I}mplicit \textbf{R}unge-\textbf{K}utta methods) are distinguished by
  Butcher schemes of the particular form
  \begin{gather}
    \label{eq:BSexpl}
    \begin{array}[c]{c|c}
      \Vc & \FA \\\hline
          & \Vb^{T}
    \end{array} \quad:= \quad
    \arraycolsep=1.4pt\def\arraystretch{2.2}
    \begin{array}[c]{c|cccccc}
      c_{1} & \gamma  & & & \cdots & & 0 \\
      c_{2} & a_{21} & \ddots &&&& \vdots \\
      \vdots & \vdots &&&& \ddots & \vdots \\
      c_{s} & a_{s1} && \cdots & & a_{s,s-1} & \gamma \\\hline
            & b_{1} && \cdots && b_{s-1} & b_{s}
    \end{array}\;,
\end{gather}  
with \Blue{$\gamma\not=0$}. 

More concretely, in this problem the scalar linear initial value problem of second
order
\begin{equation} \label{eq:awp}
  \ddot{y} + \dot{y} + y = 0,
  \qquad
  y(0)=1,
  \quad
  \dot{y}(0) = 0
\end{equation}
should be solved numerically using a SDIRK-method (\textbf{S}ingly \textbf{D}iagonally
\textbf{I}mplicit \textbf{R}unge-\textbf{K}utta Method).
It is a Runge-Kutta method described by the Butcher scheme
\begin{gather}\label{eq:1}
  \begin{array}{c|cc}
    \gamma & \gamma & 0 \\
    1-\gamma & 1-2\gamma & \gamma \\\hline
    & 1/2 & 1/2\rule{0pt}{2.2ex}
  \end{array}.
\end{gather}
%

\begin{subproblem}[1]\label{sp:0}
  Explain the benefit of using SDIRK-SSMs compared to using Gauss-Radau RK-SSMs as
  introduced in \lref{ex:RADAU}. In what situations will this benefit matter much? 

  \begin{hint}
    Recall that in every step of an implicit RK-SSM we have to solve a non-linear
    system of equations for the increments, see \lref{rem:stagenewton}. 
  \end{hint}
\end{subproblem}

\begin{subproblem}[1]\label{sp:1}
State the equations for the increments $\vec{k}_{1}$ and $\vec{k}_{2}$ of the Runge-Kutta method \eqref{eq:1} applied to the initial value problem corresponding to the differential equation $\displaystyle \dot{\By} = \Vf(t,\By)$.

\begin{solution}
The increments $\Bk_i$ are given by
\begin{align*}
	\Bk_1 &= f\left(t_0 + h\gamma, \By_0 + h\gamma\Bk_1\right),\\
	\Bk_2 &= f\left(t_0 + h(1-\gamma), \By_0 + h(1-2\gamma)\Bk_1 + h\gamma\Bk_2\right).
\end{align*}
\end{solution}

\end{subproblem}

\begin{subproblem}[1]\label{sp:2}
Show that, the stability function $S(z)$ of the SDIRK-method \eqref{eq:1} is given by
\begin{equation*}
S(z) = \frac{1 + z(1 - 2\gamma) + z^2(1/2 - 2\gamma + \gamma^2)}{(1-\gamma z)^2}
\end{equation*}
and plot the stability domain using the template \texttt{stabdomSDIRK.m}.

For $\gamma=1$ is this method:
\begin{itemize}
	\item A--stable?
	\item L--stable?
\end{itemize}
\begin{hint}
Use the same theorem as in the previous exercise.
\end{hint}

\begin{solution}
The stability function $S(z)$ of a method is derived by applying the method to the scalar, linear test equation
\begin{equation*}
	\dot{y}(t) = \lambda y(t)
\end{equation*}
The solution can be written as
\begin{equation*}
	y_{k+1} = S(z)y_k
\end{equation*}
where $S(z)$ is the stability function and $z:= h\lambda$.

In the case of the SDIRK-method, we get
\begin{align*}
	k_1 &= \lambda(y_k + h\gamma k_1)\qquad \\
	k_2 &= \lambda(y_k + h(1-2\gamma)k_1 + h\gamma k_2),
\end{align*}
therefore
\begin{align*}
	k_1 &= \frac{\lambda}{1 - h\lambda\gamma}y_k,\\
	k_2 &= \frac{\lambda}{1 - h\lambda\gamma}(y_k + h(1-2\gamma)k_1).
\end{align*}
Furthermore
\begin{equation*}
	y_{k+1} = y_k + \frac{h}{2}(k_1 + k_2),
\end{equation*}
with $z:= h\lambda$ and by plugging in $k_1$ and $k_2$ we arrive at
\begin{equation*}
	y_{k+1} = \underbrace{\left(1 + \frac{z}{2(1-\gamma z)}\left(2 + \frac{z(1-2\gamma)}{1-\gamma z}\right)\right)}_{=: S(z)}y_k.
\end{equation*}
Hence
\begin{align*}
	S(z) &= \frac{2(1-\gamma z)^2 + 2z(1-\gamma z)^2 + z^2(1-2\gamma)}{2(1-\gamma z)^2}
	\intertext{
and after collecting the powers of $z$ in the numerator we get
	}
	S(z) &= \frac{1 + z(1-2\gamma) + z^2(\gamma^2 - 2\gamma + \frac{1}{2})}{(1-\gamma z)^2}.
\end{align*}
For $\gamma=1$ the stability function is therefore
\begin{equation} \label{eq:stabfun1}
	S_1(z) := \frac{1 - z - \frac{z^2}{2}}{(1-z)^2}.
\end{equation}

\textbf{Verification of the A-stability of \eqref{eq:stabfun1}:}\\
By definition \ncseref{par:Astab},  we need to show that $|S_1(z)| \le 1$ for all  $z\in\bbC^-:= \{ z \in \bbC \; | \;  \Re z < 0 \}$.
In order to do this we consider the stability function on the imaginary axis
\begin{align*}
	|S_1(i y)|^2 &= \frac{|1 - i y - (i y)^2/2|^2}{|1-i y|^4}\\
	&= \frac{|1 -i y + y/2|^2}{|1-i y|^4}\\
	&= \frac{(1+y^2/2)^2 + y^2}{(1+y^2)^2}\\
	&= \frac{1 + 2y^2 + y^4/4}{1 + 2y^2 + y^4} \le 1,\quad y\in\bbR.
\end{align*}
Since the only pole ($z=1$) of the rational function $S_1(z)$ lies in the positive half plane of $\bbC$, the function $S_1$ is holomorphic in the left half plane. Furthermore $S_1$ is bounded by 1 on the boundary of this half plane (i.e. on the imaginary axis). So by the maximum principle for holomorphic functions (hint) $S_1$ is bounded on the entire left half plane by 1. This implies in particular that $S_1$ is $A$-stable.
% (see Ahlfors, \emph{Complex Analysis} or Conway, \emph{Functions of One Complex Variable}).

\textbf{Verification of the L-stability of \eqref{eq:stabfun1}:}\\
$S_1$ is not $L$-stable (cf. definition \ncseref{par:Lstab}), because
\begin{equation*}
	\lim_{\Re z\to -\infty} |S_1(z)| = \lim_{\Re z\to-\infty} \left|\frac{1-z-z^2/2}{1-2z+z^2}\right| = \frac{1}{2} \ne 0.
\end{equation*}

\end{solution}
\end{subproblem}

\begin{subproblem}[1] \label{sp:3}
Formulate \eqref{eq:awp} as an initial value problem for a linear first order system for the function $\Bz(t) = (y(t),\dot{y}(t))^\top$.

\begin{solution}
Define $z_1=y$, $z_2=\dot{y}$, then the initial value problem
\begin{equation}\label{eq:awp2}
	\ddot{y} + \dot{y} + y = 0,\qquad y(0)=1,\quad \dot{y}(0) = 0
\end{equation}
is equivalently to the first order system
\begin{align}\label{eq:system}
	\dot{z}_1 &= z_2 \notag\\
	\dot{z}_2 &= -z_1 -z_2,
\end{align}
with initial values $z_1(0)=1$, $z_2(0)=0$.
\end{solution}

\end{subproblem}

\begin{subproblem}[1]\label{sp:4}
Implement a \Cpp{}-function
\begin{lstlisting}[language=c++]
template <class StateType>
StateType sdirtkStep(const StateType & z0, double h, double gamma);
\end{lstlisting}
that realizes the numerical evolution of one step of the method \eqref{eq:1} for the differential equation determined in \autoref{sp:3} starting from the value \verb|z0| and returning the value of the next step of size \verb|h|.

\begin{hint}
See \verb|sdirk_template.cpp|.
\end{hint}

\cprotEnv \begin{solution}
Let
\[
	\VA:=\left(\begin{array}{c c}0&1\\-1&-1\end{array}\right).
\]
Then
\begin{align*}
k_1&= \VA \By_0+h\gamma\VA k_1\\
k_2&= \VA\By_0+h(1-2\gamma)\VA k1+h\VA k_2,
\end{align*}
so
\begin{align*}
k_1&= (\VI-h\gamma\VA)^{-1} \VA \By_0\\
k_2&= (\VI-h\gamma\VA)^{-1}(\VA\By_0+h(1-2\gamma)\VA k_1).
\end{align*}
See the implementation in \verb|sdirk.cpp|.
% \lstinputlisting[ emph={sdirkStep}, caption={Perform one iteration of \eqref{eq:1}},
%                   label={code:sdirkStep}]{matlab/sdirkStep.m}
% }
\end{solution}

\end{subproblem}

\begin{subproblem}[1]\label{sp:5}
  Use your \Cpp{} code to conduct a numerical experiment, which gives an
  indication of the order of the method (with $\gamma=\frac{3+\sqrt{3}}{6}$) for
  the initial value problem from \autoref{sp:3}.  Choose $\By_0=[1,0]^\top$ as
  initial value, \texttt{T=10} as end time and \texttt{N=20,40,80,\dots,10240} as
  steps.
	
\cprotEnv \begin{solution}
The numerically estimated order of convergence is $3$, see \verb|sdirk.cpp|.
\end{solution}
\end{subproblem}
	
% \lstinputlisting[ emph={sdirkConv}, caption={Calculate the order of convergence of \eqref{eq:1} },
%                   label={code:sdirkConv}]{matlab/sdirkConv.m}
% \end{solution}
% \begin{center}
%   \begin{math}
%     \label{eq:sdirk1}
%     \begin{tabular}{c|cc}
%       $\gamma$ & $\gamma$ & $0$ \\
%       $1-\gamma$ & $1-2\gamma$ & $\gamma$ \\\hline
%       & $1/2$ & $1/2$
%     \end{tabular}
%   \end{math}
% \end{center}
% \end{subproblem}
%%%%%%%%%%%%%%%%%%%%%%%%%%%%%%%%%%%%%%%%%%%%%%%%%%%%%%%%%%%%%%%%%%%%%%%%%

\end{problem}
%{\small
%\lstinputlisting[ emph={}, caption={Testcalls for \autoref{prb:sdirk}},
   %               label={code:test_call}]{matlab/test_call.m}

%\lstinputlisting[ emph={}, caption={Output for Testcalls for \autoref{prb:sdirk}},
   %               label={code:test_call_out}]{matlab/test_call_out.txt}
%}
