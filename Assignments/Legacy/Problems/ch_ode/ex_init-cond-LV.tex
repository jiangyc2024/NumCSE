\begin{problem}[Initial Condition for Lotka-Volterra ODE]\label{prb:init-cond-LV} 
%%%%%%%%%%%%%%%%%%%%%%%%%%%%%%%%%%%%%%%%%%%%%%%%%%%%%%%%%%%%%%%%%%%%%%%%%%
% INSTEAD OF COMMENTING OUT SUBPROBLEMS, USE \hidesubprb,
% so the subproblem is still displayed in the catalogue!
%%%%%%%%%%%%%%%%%%%%%%%%%%%%%%%%%%%%%%%%%%%%%%%%%%%%%%%%%%%%%%%%%%%%%%%%%%
\textbf{Introduction.} 
In this problem we will face a situation, where we need to compute the derivative
of the solution of an IVP with respect to the initial state. This paragraph will
show how this derivative can be obtained as the solution of another differential
equation. Please read this carefully and try to understand every single argument.

We consider IVPs for the autonomous ODE
\begin{equation}\label{eq:LV-1}
\dot \Vy = \Vf (\Vy)
\end{equation}
with smooth right hand side $\Vf\colon D\to\R^d$, where $D\subseteq \R^d$ is the
state space. We take for granted that for all initial states, solutions exist for
all times (global solutions, see \lref{ass:globsol}).

By its very definition given in \lref{def:evolop}), the evolution operator
\[
\BPhi\colon \R\times D\to D,\quad (t,\Vy)\mapsto \BPhi(t,\Vy)
\]
satisfies
\[
\frac{\partial\BPhi}{\partial t}(t,\Vy)=\Vf(\BPhi(t,\Vy)).
\]
Next, we can differentiate this identity with respect to the state variable
$\Vy$. We assume that all derivatives can be interchanged, which can be justified
by rigorous arguments (which we won't do here). Thus, by the chain rule, we obtain
and after swapping partial derivatives $\frac{\partial}{\partial t}$ and
$\Derv_\Vy$
\[
\frac{\partial \Derv_\Vy \BPhi}{\partial t}(t,\Vy)=\Derv_\Vy\frac{\partial\BPhi}{\partial t}(t,\Vy)=\Derv_\Vy(\Vf(\BPhi(t,\Vy)))=\Derv\Vf(\BPhi(t,\Vy))\Derv_\Vy\BPhi(t,\Vy).
\]
Abbreviating $\VW(t,\Vy):=\Derv_\Vy\BPhi(t,\Vy)$ we can rewrite this as the non-autonomous ODE
\begin{equation}\label{eq:LV-2}
\dot \VW = \Derv\Vf(\BPhi(t,\Vy))\VW.
\end{equation}
Here, the state $\Vy$ can be regarded as a parameter.  Since $\BPhi(0,\Vy)=\Vy$,
we also know $\VW(0,\Vy)=\VI$ (identity matrix), which supplies an initial
condition for \eqref{eq:LV-2}. In fact, we can even merge \eqref{eq:LV-1} and
\eqref{eq:LV-2} into the ODE
\begin{gather}
  \label{eq:LV-2a}
  \frac{d}{dt}\left[\Vy(\cdot)\; ,\; \VW(\cdot,\Vy_{0})\right] = 
  \left[\Vf(\Vy(t))\;,\; \Derv\Vf(\Vy(t))\VW(t,\Vy_{0})\right]\;,
\end{gather}
which is autonomous again. 

Now let us apply \eqref{eq:LV-2}/\eqref{eq:LV-2a}.
As in \lref{ex:LV}, we consider the following autonomous Lotka-Volterra differential equation of a predator-prey model
\begin{equation}\label{eq:LVsystem}
  \begin{array}{ccl}
    \dot{u} &=& (2-v)u\\
    \dot{v} &=& (u-1)v
  \end{array}
\end{equation}
on the state space $D=\R_+^2$, $\R_+=\{\xi\in\R:\xi>0\}$. All the solutions of \eqref{eq:LVsystem} are periodic and their period depends on the initial state $[u(0),v(0)]^T$. In this exercise we want to develop a numerical method which computes a suitable initial condition for a given period.

\begin{subproblem}[1]
For fixed state $\Vy\in D$, \eqref{eq:LV-2} represents an ODE. What is its state space?
\begin{solution}
By construction, $\BPhi$ is a function with values in $\R^d$. Thus,
$\Derv_\Vy\BPhi$ has values in $\R^{d,d}$, and so the state space of
\eqref{eq:LV-2} is $\R^{d,d}$, a space of matrices. 
\end{solution}
\end{subproblem}

\begin{subproblem}[1]
What is the right hand side function for the ODE \eqref{eq:LV-2}, in the case of the $\dot\Vy=\Vf(\Vy)$ given by the Lotka-Volterra ODE \eqref{eq:LVsystem}? You may write $u(t),v(t)$ for solutions of \eqref{eq:LVsystem}.
\begin{solution}
Writing $\Vy=[u,v]^T$, the map $\Vf$ associated to \eqref{eq:LVsystem} is $\Vf(\Vy)=[(2-v)u,(u-1)v]^T$. Therefore
\[
\Derv\Vf(\Vy)=\begin{bmatrix} 2-v & -u \\ v & u-1 \end{bmatrix}.
\]
Thus,  \eqref{eq:LVsystem} becomes
\begin{equation}\label{eq:W}
\dot \VW = \begin{bmatrix} 2-v(t) & -u(t) \\ v(t) & u(t)-1 \end{bmatrix} \VW.
\end{equation}
This is a non-autonomous ODE. 
\end{solution}
\end{subproblem}

%\begin{subproblem}[3]
%A C++ implementation of an adaptive embedded Runge-Kutta method is available, with a functionality similar to \Matlab{}'s \texttt{ode45}. Relying on this write a C++ function
%\begin{lstlisting}
%void LVsolve (double u0, double v0, double T)
%\end{lstlisting}
%that writes the states $[u_k,v_k]^T\in\R^2$ computing during timestepping to \texttt{stdout}.
%\begin{solution}
%See file \texttt{...cpp}.
%\end{solution}
%\end{subproblem}

\begin{subproblem}[1] \label{sp:evolutionop}
From now on we write $\BPhi\colon\R\times\R^2_+\to\R^2_+$ for the evolution operator associated with \eqref{eq:LVsystem}. Based on $\BPhi$ derive a function $\VF: \R_+^2 \to \R_+^2$ which evaluates to zero for the input $\Vy_0$ if the period of the solution of system \eqref{eq:LVsystem} with initial value
\[
	\Vy_0=\left[\begin{array}{c}u(0)\\ v(0)\end{array}\right]
\]
is equal to a given value $T_P$. 
\begin{solution}
Let $\VF$ be defined by 
\[
	\VF(\Vy) := \BPhi(T_P,\Vy) - \Vy.
\]
If the solution of the system \eqref{eq:LVsystem} with initial value $ \Vy_0$ has period $T_P$, then we have $\BPhi(T_P,\Vy_0) = \Vy_0$, so $\VF(\Vy_0) = 0$. 
\end{solution}
\end{subproblem}

\begin{subproblem}[1]
We write $\VW(T,\Vy_0)$, $T\ge 0$, $\Vy_0\in\R^2_+$ for the solution of \eqref{eq:LV-2} for the underlying ODE  \eqref{eq:LVsystem}. Express the Jacobian of $\VF$ from \ref{sp:evolutionop} by means of $\VW$.
\begin{solution}
By definition of $\VW$ we immediately obtain 
\begin{align*}
\Derv\VF(\Vy) = \VW(T_P,\Vy)-\VI
\end{align*}
where $\VI$ is the identity matrix.
\end{solution}
\end{subproblem}

\begin{subproblem}[3]\label{sp:uniq}
  Argue, why the solution of $\VF(\Vy)=0$ will, in gneneral, not be unique. 
  When will it be unique?

  \begin{hint}
    Study \lref{par:autode} again. Also look at \lref{lv1}. 
  \end{hint}

  \begin{solution}
    If {$\Vy_{0}\in\bbR^{2}_{+}$} is a solution, then every state on the
    trajectory \Blue{$\Vy_{0}$} will also be a solution. Only if the trajectory
    collapses to a point, that is, if $\Vy_{0}$ is a stationary point,
    $\Vf(\Vy_{0})=0$, we can expect uniqueness.
  \end{solution}
\end{subproblem}

\begin{subproblem}[2] \label{sp:wronski}
A C++ implementation of an adaptive embedded Runge-Kutta method is available, with
a functionality similar to \Matlab{}'s \texttt{ode45} (see \ref{prb:matrix-diff-eq}). Relying on this implement a
C++ function
\begin{lstlisting}
std::pair<Vector2d,Matrix2d> PhiAndW(double u0, double v0, double T)
\end{lstlisting}
that computes $\BPhi(T,[u_0,v_0]^T)$ and $\VW(T,[u_0,v_0]^T)$. The first component of the output pair should contain $\BPhi(T,[u_0,v_0]^T)$ and the second component the matrix $\VW(T,[u_0,v_0]^T)$. See \texttt{LV\_template.cpp}.
\begin{hint}
  As in \eqref{eq:LV-2a}, both ODEs (for $\BPhi$ and $\VW$) must be combined into a
  single autonomous differential equation on the state space
  $D\times \bbR^{d\times d}$.
\end{hint}

\begin{hint}
The equation for $\VW$ is a matrix differential equation. These cannot be solved directly using \texttt{ode45}, because the solver expects the right hand side to return a vector.  Therefore, transform matrices into vectors (and vice-versa).
\end{hint}
\begin{solution}
Writing $\Vw = [u,v,W_{11}, W_{21}, W_{12},W_{22}]^T$, by \eqref{eq:LVsystem} and \eqref{eq:W} we have that
\begin{align*}
& \dot w_1 = (2-w_2)w_1  && \dot w_3 = (2-w_2)w_3 - w_1 w_4 && \dot w_5 = (2-w_2)w_5 - w_1 w_6\\
& \dot w_2 = (w_1-1)w_2 && \dot w_4=w_2 w_3 + (w_1 -1) w_4 && \dot w_6 = w_2 w_5 + (w_1 - 1) w_6\\
\end{align*}
with initial conditions
\[
w_1(0)=u_0,\quad w_2(0)=v_0, \quad w_3(0)=1, \quad w_4(0)=0, \quad w_5(0)=0, \quad w_6(0)=1,
\]
since $\VW(0,\Vy)=\Derv_\Vy\BPhi(0,\Vy)=\Derv_\Vy \Vy =\VI$. 

The implementation of the solution of this system is given in the file \texttt{LV.cpp}.
\end{solution}
\end{subproblem}


\begin{subproblem}[3] \label{sp:LVperiod} Using \texttt{PhiAndW}, write a C++
  routine that determines initial conditions $u(0)$
  and $v(0)$ such that the solution of the system \eqref{eq:LVsystem} has period
  $T=5$.  Use the multi-dimensional \emph{Newton method} for $\VF(\Vy)= 0$ with
  $\VF$ from \ref{sp:evolutionop}. As your initial approximation, use
  $[3,2]^T$. Terminate the \emph{Newton method} as soon as
  $\norm{\VF(\Vy)} \leq 10^{-5}$. Validate your implementation by comparing the
  obtained initial data $\Vy$ with $\BPhi(100,\Vy)$.

\begin{hint}
  Set relative and absolute tolerances of \texttt{ode45} to $10^{-14}$ and $10^{-12}$, respectively. See file \texttt{LV\_template.cpp}.
\end{hint}

\begin{hint}
The correct solutions are $u(0)\approx 3.110$ and $v(0)=2.081$.
\end{hint}

\begin{solution}
See file \texttt{LV.cpp}.
\end{solution}
\end{subproblem}

%%%%%%%%%%%%%%%%%%%%%%%%%%%%%%%%%%%%%%%%%%%%%%%%%%%%%%%%%%%%%%%%%%%%%%%%%%

%{\small
%\lstinputlisting[ emph={}, caption={Testcalls for \autoref{prb:init-cond-LV}},
  %                label={code:test_call}]{matlab/test_call.m}

%\lstinputlisting[ emph={}, caption={Output for Testcalls for \autoref{prb:init-cond-LV}},
  %                label={code:test_call_out}]{matlab/test_call_out.txt}
%}
\end{problem}
