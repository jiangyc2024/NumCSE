\begin{problem}[Semi-implicit Runge-Kutta SSM\coreproblem]\label{prb:sirk-ssm}
  General implicit Runge-Kutta methods as introduced in \lref{ss:IRK} entail
  solving systems of non-linear equations for the increments, see
  \lref{rem:stagenewton}. Semi-implicit Runge-Kutta single step methods, also
  known as Rosenbrock-Wanner (ROW) methods \lref{eq:sirkinc} just require the solution
  of linear systems of equations. This problem deals with a concrete ROW method,
  its stability and aspects of implementation.

  We consider the following  autonomous ODE
\begin{align}
 \dot{\vec{y}} = \vec{f}(\vec{y})
\end{align}
and discretize it with a \emph{semi-implicit} Runge-Kutta SSM (\emph{Rosenbrock method}):
\begin{gather} \label{method:rosenbrock}
  \begin{aligned}
    \vec{W} \vec{k}_1 & = \vec{f}(\vec{y}_0) \\
    \vec{W} \vec{k}_2 & = \vec{f}(\vec{y}_0 + \frac{1}{2} h \vec{k}_1) - a h \vec{J} \vec{k}_1 \\
    \vec{y}_1 & = \vec{y}_0 + h \vec{k}_2
  \end{aligned}
\end{gather}
where
\begin{align*}
 \vec{J} = D\vec{f}(\vec{y}_0) \\
 \vec{W} = \vec{I} - ah \vec{J} \\
 a = \frac{1}{2+\sqrt{2}}.
\end{align*}

% Cf. \cite{} as an extra reading.

\begin{subproblem}[2]
 Compute the stability function $S$ of the Rosenbrock method \eqref{method:rosenbrock}, that is, compute the (rational) function $S(z)$, such that
 \begin{align*}
  y_1 = S(z) y_0, \quad z := h \lambda,
 \end{align*}
 when we apply the method to perform one step of size $h$, starting from $y_0$, of the linear scalar model ODE $\dot{y} = \lambda y, \lambda \in \IC$.

 \begin{solution}
  For a scalar ODE, the Jacobian is just the derivative w.r.t. $y$, whence $J = D f(y) = f'(y) = (\lambda y)' = \lambda$. The quantity $\vec{W}$ is, therefore, a scalar quantity as well:
  \begin{align*}
   W = 1 - a h J = 1 - a h \lambda
  \end{align*}
  If we plug everything together and assume $h$ is small enough:
  \begin{align*}
   y_1 & = y_0 + h \frac{\vec{f}(\vec{y}_0 + \frac{1}{2} h \vec{k}_1) - a h \vec{J} \vec{k}_1}{W} \\
   & = y_0 + h \frac{\vec{f}(\vec{y}_0 + \frac{1}{2} h \frac{\vec{f}(\vec{y}_0)}{W}) - a h \vec{J} \frac{\vec{f}(\vec{y}_0)}{W})}{W} \\
   & = y_0 + h \frac{\lambda (\vec{y}_0 + \frac{1}{2} h \frac{\lambda \vec{y}_0}{1 - a h \lambda}) - \frac{a h \lambda^2 y_0}{1 - a h \lambda}}{1 - a h \lambda} \\
   & = \frac{(1 - a h \lambda)^2 + h \lambda (1 - a h \lambda) + \frac{1}{2} h^2 \lambda^2 - a h^2 \lambda^2}{(1 - a h \lambda)^2} y_0 \\
   & = \frac{(1 + a^2 z^2 - 2az) + z (1 - a z) + \frac{1}{2} z^2 - a z^2}{(1 - a z)^2} y_0 \\
   & = \frac{1 + (1-2a) z +  (\frac{1}{2}-2a+ a^2) z^2}{(1 - a z)^2} y_0
  \end{align*}
  Since $\frac{1}{2}-2a+ a^2 = 0$, it follows
  \begin{align*}
   S(z) = \frac{1 + (1-2a) z}{(1 - a z)^2}
  \end{align*}
 \end{solution}
\end{subproblem}

\begin{subproblem}[3]
 Compute the first $4$ terms of the Taylor expansion of $S(z)$ around $z = 0$. What is the maximal $q \in \IN$ such that
 \begin{align*}
  \lvert S(z) - \exp(z) \rvert = O(\lvert z \rvert^q)
 \end{align*}
 for $\lvert z \rvert \rightarrow 0$? Deduce the maximal possible 
 order of the method \eqref{method:rosenbrock}.
 
 \begin{hint}
   The idea behind this sub-problem is elucidated in \lref{rem:stabexp}. Apply
   \lref{lem:Sapproxexp}. 
 \end{hint}

 \begin{solution}
  We compute:
  \begin{align*}
   S(0) & = 1, \\
   S'(0) & = 1, \\
   S''(0) & = 4a - 2 a^2, \\
   S'''(0) & = 18a^2 - 12a^3,
  \end{align*}
   therefore
   \begin{align*}
    S(z) = 1 + z + (2a - a^2) z^2 + (3a^2 - 2a^3) z^3 + O(z^4).
   \end{align*}
  We can also compute the expansion of $\exp$:
   \begin{align*}
    \exp(z) = 1 + z + \frac{1}{2} z^2 + \frac{1}{6} z^3 + O(z^4)
   \end{align*}
   and since $2a - a^2 = \frac{1}{2}$ but $3a^2 - 2a^3 = \frac{1}{2}(\sqrt{2}-1) \neq \frac{1}{6}$, we have:
 \begin{align*}
  \lvert S(z) - \exp(z) \rvert = O(\lvert z \rvert^3)
 \end{align*}
 Using \ncseref{lem:Sapproxexp}, we deduce that the maximal order $q$ of the scheme is $q = 2$.
 \end{solution}
\end{subproblem}

\begin{subproblem}[2]
 Implement a \Cpp{} function:
\begin{lstlisting}[language=c++]
template <class Func, class DFunc, class StateType>
std::vector<StateType> solveRosenbrock(
		      const Func & f, const DFunc & df,
		      const StateType & y0,
		      unsigned int N, double T)
 \end{lstlisting}
 taking as input function handles for $\vec{f}$ and $D\vec{f}$ (e.g. as lambda
 functions), an initial data (vector or scalar) \verb|y0| $= \vec{y}(0)$, a number
 of steps $N$ and a final time $T$. The function returns the sequence of states generated
 by the single step method up to $t = T$, using $N$ equidistant steps of the
 Rosenbrock method \eqref{method:rosenbrock}.
 
 \cprotEnv \begin{hint}
  See \verb|rosenbrock_template.cpp|.
 \end{hint}
 
 \cprotEnv \begin{solution}
  See \verb|rosenbrock.cpp|.
 \end{solution}
\end{subproblem}

\begin{subproblem}[3]
 Explore the order of the method \eqref{method:rosenbrock} empirically by applying it to the IVP for the limit cycle \ncseref{ex:limitcycle}:
 \begin{align}
  \vec{f}(\vec{y}) := \begin{bmatrix}
                       0 & -1 \\ 1 & 0
                      \end{bmatrix} \vec{y} + \lambda ( 1 - \lVert \vec{y} \rVert^2 ) \vec{y},
 \end{align}
 with $\lambda = 1$ and initial state $\vec{y_0} = [1,1]^\top$ on $[0,10]$. Use fixed timesteps of size $h = 2^{-k}, k = 4,\dots,10$ and compute a reference solution with $h = 2^{-12}$ step size. Monitor the maximal mesh error:
 \begin{align*}
  \max_j \lVert  \vec{y}_j - \vec{y}(t_j) \rVert_2.
 \end{align*}
 \cprotEnv \begin{solution}
  The Jacobian $Df$ is:
  \[
  D\vec{f}(\vec{y}) = \begin{bmatrix}
                       \lambda ( 1 - \lVert \vec{y} \rVert^2 ) - 2 \lambda y_0^2 & -1 -2 \lambda y_1 y_0 \\
                       1 -2 \lambda y_1 y_0 & \lambda ( 1 - \lVert \vec{y} \rVert^2 ) - 2 \lambda y_1^2
                      \end{bmatrix}.
  \]
  For the implementation, cf. \verb|rosenbrock.cpp|.
 \end{solution}
\end{subproblem}

\begin{subproblem}[4]
 Show that the method \eqref{method:rosenbrock} is $L$-stable (cf. \ncseref{par:Lstab}).
 \begin{hint}
To investigate the $A$-stability, calculate the complex norm of $S(z)$ on the imaginary axis $\Re{z}=0$ and apply the following maximum principle for holomorphic functions:

\begin{theorem}[Maximum principle for holomorphic functions]
 Let \[\IC^- := \{ z \in \IC \; | \; Re(z) < 0 \}.\] Let $f: D \subset \IC \rightarrow \IC$ be non-constant, defined on $\overline{\IC^-}$, and analytic in $\IC^-$. Furthermore, assume that $w := \lim_{ \lvert z \rvert \rightarrow \infty} f(z)$ exists and $w \in \IC$, then:
 \begin{align*}
  \forall z \in \IC^- \lvert f(z) \rvert < \sup_{\tau \in \IR} \lvert f(i \tau) \rvert.
 \end{align*}
\end{theorem}
 \end{hint}

 \begin{solution}
  We start by proving that the method is $A$-stable \ncseref{par:Astab}, meaning that $S(z) < 1, \forall z \in \IC, \mathrm{Re}(z) < 0$. First of all, we can compute the complex norm of the stability function at $z = iy, y \in \IR$ as:
  \begin{align*}
   \lvert S(iy) \rvert^2 = \frac{\lvert 1 + (1-2a) iy \rvert^2}{\lvert(1 - a iy)^2 \rvert^2} = \frac{1+ (1-2a)^2y^2}{(1+ a^2 y^2)^2} = \frac{1+ (1-4a+4a^2)y^2}{(1+ 2a^2 y^2 + a^4 y^4)}.
  \end{align*}
  Notice that $1 - 4a + 4a^2 = 2a^2$, therefore:
  \begin{align*}
   \lvert S(iy) \rvert^2 = \frac{1+ 2 a^2 y^2}{1+ 2a^2 y^2 + a^4 y^4} < 1.
  \end{align*}
  The norm of the function $S$ is bounded on the imaginary axis. Observe that $\lvert S(z) \rvert \rightarrow 0, \lvert z \rvert \rightarrow \infty$, which, follows from the fact that the degree of the denominator of $S$ is bigger than the polynomial degree of the numerator.
  Notice that $1-2a = 1-2\frac{1}{\sqrt(2)+2} = \sqrt{2} - 2$.
  
  The only pole of the function is at $z = 1/a$, therefore the function is holomorphic on the left complex plane.
  
  Applying the theorem of the hint on concludes that the absolute value of the function is bounded by $1$ on the left complex plane.
  
  The $L$-stability follows immediately with $A$-stability and the fact that the absolute value of the function converges to zero as $\lvert z \rvert \rightarrow \infty$.
 \end{solution}
\end{subproblem}


 
\end{problem}
 