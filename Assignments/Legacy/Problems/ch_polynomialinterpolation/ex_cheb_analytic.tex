\begin{problem}[Chebychev interpolation of analytic functions \coreproblem] 
 
This problem concerns Chebychev interpolation
(cf. \ncsesect{sec:ChebychevInterpolation}). Using techniques from complex
analysis, notably the residue theorem \lref{thm:residue}, in class we derived an
expression for the interpolation error \lref{eq:recipform} and from it an error
bound \lref{eq:intperrbd}, as much sharper alternative to \ncseref{cor:polintperr}
and \ncseref{thm:polintperr} for \emph{analytic} interpolands. The bound tells us
that for all $t\in [a,b]$
\begin{align*}
  \lvert f(t) - \Op{L}_{\Ct} f(t) \rvert 
   \leq 
   \left\lvert \frac{w(x)}{2 \pi i} \int_\gamma 
   \frac{f(z)}{(z-t) w(z)} dz \right\rvert 
  \leq \frac{\lvert \gamma \rvert}{2 \pi} 
  \frac{\max_{a \leq \tau \leq b}\lvert w(\tau) \rvert}{\min_{z \in  \gamma}\lvert
  w(z) \rvert} 
  \frac{ \max_{z \in \gamma} \lvert f(z) \rvert }{d([a,b], \gamma)}\;,
\end{align*}
where {$d([a,b],\gamma)$} is the geometric distance of the integration contour
$\gamma\subset\bbC$ from the interval $[a,b]\subset\bbC$ in the complex plane. The
contour $\gamma$ must be contractible in the domain $D$ of analyticity of $f$ and
must wind around $[a,b]$ exactly once, see \lref{analintp}. 

Now we consider the interval $[-1,1]$. Following \ncserem{rem:chebipanal}, our
task is to find an upper bound for this expression, in the case where $f$
possesses an analytical extension to a complex neighbourhood of $[-1,1]$.

For the analysis of the Chebychev interpolation of analytic functions we used the
elliptical contours, see \lref{ellipses},
\begin{align}
 \gamma_\rho(\theta) := \cos(\theta - i \log(\rho) )\;,\quad
  \forall 0 \leq \theta \leq 2 \pi\;,\quad \rho > 1\;.
\end{align}

\begin{subproblem}[2]
 Find an upper bound for the length $\lvert \gamma_\rho \rvert$ of the contour $\gamma_\rho$.
 
 \begin{hint}
  You may use the arc-length formula for a curve $\gamma: I \rightarrow \IR^2$:
  \begin{align}
   \lvert \gamma \rvert = \int_I \lVert \dot\gamma(\tau) \rVert d\tau,
  \end{align}
  where $\dot\gamma$ is the derivative of $\gamma$ w.r.t the parameter
  $\tau$. Recall that the ``length'' of a complex number $z$ viewed as a vector in 
  $\bbR^{2}$ is just its modulus. 
 \end{hint}

 \begin{solution}
  $\frac{\partial \gamma_\rho(\theta)}{\partial \theta} := -\sin(\theta - i \log(\rho) )$, therefore:
  \begin{align}
   \lvert \gamma_\rho \rvert & = \int_{[0,2 \pi]} \lvert \sin(\tau - i \log(\rho) ) \rvert d\tau \\
   & = \frac{1}{2 \rho} \int_{[0,2 \pi]} \sqrt{ \sin^2(\tau) (1 + \rho^2)^2 + \cos^2(\tau) (1 - \rho^2)^2 } d\tau \\
   & \leq \frac{1}{2 \rho} \int_{[0,2 \pi]} \sqrt{ 2 + 2\rho^4 } d\tau
   & \leq \frac{1}{\rho} \pi \sqrt{ 2 (1 + \rho^4) }
  \end{align}
 \end{solution}
\end{subproblem}

Now consider the $S$-curve function (the logistic function):
\begin{align*}
 f(t) := \frac{1}{1+e^{-3t}}\;,\quad t \in \IR\;.
\end{align*}

\begin{subproblem}[2] \label{subprb:max_dom_analyticity}
 Determine the maximal domain of analyticity of the extension of $f$ to the complex plane $\IC$.
 \begin{hint}
  Consult \lref{rem:analfunc}.
 \end{hint}
 
 \begin{solution}
  $f$ is analytic in $D := \IC \setminus \{ \frac{2}{3} \pi i c - \frac{1}{3} \pi i \; | \; c \in \IZ \}$. In fact, $g(t) := 1 + exp(-3t)$ is an entire function, whereas $h(x) := \frac{1}{x}$ is analytic in $\IC \setminus \{ 0 \}$. Therefore, using \ncseref{thm:analchain}, $f$ is analytic in $\IC \setminus \{ z \in \IC \; | \; g(z) = 0 \} =: \IC \setminus S$. 
  Let $z := a + ib$, $a,b \in \IR$. Since:
  \begin{align}
   -1 = \exp(z) = \exp(a + ib) = \exp(a) (\cos(b) + i \sin(b)) & \Leftrightarrow a = 0, b \in 2 \pi \IZ + \pi \\
   \exp(z) = -1 & \Leftrightarrow z \in i(2 \pi  \IZ + \pi) \\
   \exp(-3z) = -1 & \Leftrightarrow z \in \frac{i(2 \pi \IZ - \pi)}{3}
  \end{align}
  Therefore $S = \frac{2}{3} \pi i \IZ - \frac{1}{3} \pi i$.
 \end{solution}
\end{subproblem}

\begin{subproblem}[2] \label{subprb:compute_upper_bound_cheby}
 Write a \Matlab{} function that computes an approximation $M$ of:
 \begin{align}
  \min_{\rho > 1} \frac{ \max_{z \in \gamma_\rho} \lvert f(z) \rvert }{d([-1,1], \gamma_\rho)},
 \end{align}
 by sampling, where the distance of $[a,b]$ from $\gamma_{\rho}$ is formally
 defined as
 \begin{align}
  d([a,b], \gamma) := \inf \{ \lvert z - t \rvert \; | \; z \in \gamma, t \in [a,b] \}.
 \end{align}

 \begin{hint}
  The result of \ref{subprb:max_dom_analyticity}, together with the knowledge that $\gamma_\rho$ describes an ellipsis, tells you the maximal range $(1,\rho_{max})$ of $\rho$. Sample this interval with $1000$ equidistant steps.
 \end{hint}
 
 \begin{hint}
  Apply geometric reasoning to establish that the
  distance of $\gamma_\rho$ and $[-1,1]$ is $\frac{1}{2}(\rho + \rho^{-1}) - 1$.
 \end{hint}
 
 \begin{hint}
  If you cannot find $\rho_{max}$ use $\rho_{max} = 2.4$.
 \end{hint}
 
 \begin{hint}
  You can exploit the properties of $\cos$ and the hyperbolic trigonometric functions $\cosh$ and $\sinh$.
 \end{hint}

\cprotEnv \begin{solution}
  The ellipse must be restricted such that the minor axis has length $\leq 2 \pi / 3$ ($2$ times the smallest point, in absolute value, where $f$ is not-analytic). Since this corresponds to the imaginary part of $\gamma_\delta(\theta)$, when $\theta = \pi / 2$, we find:
  \begin{align*}
   \cos(\pi / 2 - i \log(\rho_{max})) = 1 / 3 \pi i \Leftrightarrow \sinh(\log(\rho_{max})) = \pi/3 \Leftrightarrow \rho_{max} = \exp(\sinh^{-1}(\pi / 3)).
  \end{align*}
  See \verb|cheby_approx.m| and \verb|cheby_analytic.m| for the \Matlab{} code.
 \end{solution}
\end{subproblem}

\begin{subproblem}[2]
  \label{cheb:sp:bd}
 Based on the result of \ref{subprb:compute_upper_bound_cheby}, and  \ncseref{eq:chebanalest}, give an ``optimal'' bound for
 \begin{align*}
  \lVert f - L_n f \rVert_{L^\infty([-1,1])},
 \end{align*}
 where $L_n$ is the operator of Chebychev interpolation on $[-1,1]$ into the space
 of polynomials of degree $\leq n$.

 \begin{solution}
 Let $M$ be the approximation of \ref{subprb:compute_upper_bound_cheby}. Then
 \begin{align*}
   \lVert f - L_n f \rVert_{L^\infty([-1,1])} \lesssim \frac{M \sqrt{2 (\rho^{-2} + \rho^{2})}}{\rho^{n+1} - 1}.
 \end{align*}
 \end{solution}
\end{subproblem}

\begin{subproblem}[1]
  Graphically compare your result from \ref{cheb:sp:bd} with the measured supremum
  norm of the approximation error of Chebychev interpolation of $f$ on $[-1,1]$
  for polynomial degree $n = 1,\dots,20$. To that end, write a \Matlab{}-code and
  rely on the provided function \verb|intpolyval| (cf. \ncseref{trigpolyval}).
 \begin{hint}
  Use semi-logarithmic scale for your plot \verb|semilogy|.
 \end{hint}
 \cprotEnv \begin{solution}
   See \verb|cheby_analytic.m|.
 \end{solution}
\end{subproblem}

\begin{subproblem}[4]
 Rely on pullback to $[-1,1]$ to discuss how the error bounds in
 \ncseref{eq:chebanalest} will change when we consider Chebychev interpolation on
 $[-a,a], a > 0$, instead of $[-1,1]$, whilst keeping the function $f$ fixed.

 \begin{solution}
   The rescaled function $\Phi^{\ast}f$ will have a different domain of
     analyticity and a different growth behavior in the complex plane. The larger
     $a$, the closer the pole of $\Phi^{\ast}f$ will move to $[-1,1]$, the more
     the choice of the ellipses is restricted (i.e. $\rho_{max}$ becomes smaller). This will result in a larger bound.
   
   Using \ncseref{eq:tpdft}, if follows immediately that the asymptotic behaviour of the interpolation does not change after rescaling of the interval. In fact, if $\Phi^*$ is the affine pullback from $[-a,a]$ to $[-1,1]$, then:
   \begin{align*}
     \lVert f - \hat{(L_n)} f \rVert_{L^\infty([-a,a])} = \lVert \Phi^* f - L_n \Phi^* f \rVert_{L^\infty([-1,1])},
   \end{align*}
   where $\hat{(L_n)}$ is the interpolation on $[-a,a]$.
 \end{solution}
 \end{subproblem}

\end{problem}
