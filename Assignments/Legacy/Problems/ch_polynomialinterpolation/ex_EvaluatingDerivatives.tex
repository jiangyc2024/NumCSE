% ncse_new/p2_InterpolationApproximation/ch1_PolynomialInterpolation/ex_EvaluatingDerivatives.tex
% exercise requires:    -
% solutions require:    dipoleval.m  dipoleval_test.m  dipoleval_test.eps

\begin{problem}[Evaluating the derivatives of interpolating polynomials \coreproblem]
\label{prb:EvaluatingDerivatives}

In \lref{sec:polipsingleeval} we learned about an efficient and
``update-friendly'' scheme for evaluating Lagrange interpolants at a single or a
few points. This so-called Aitken-Neville algorithm, see \lref{AitkenNeville}, can
be extended to return the derivative value of the polynomial interpolant as well.
This will be explored in this problem. 

\begin{subproblem}[1]
  Study the Aitken-Neville scheme introduced in \lref{par:AIN}. 
\end{subproblem}

\begin{subproblem}[3]\label{dpe:sp:1}
Write an efficient MATLAB function
\begin{center}
\texttt{dp = dipoleval(t,y,x)}
\end{center}
that returns the row vector $(p'(x_{1}),\ldots,p'(x_{m}))$, when the argument
\texttt{x} passes $(x_{1},\ldots,x_{m})$, $m\in\bbN$ small.  Here, $p'$ denotes the
\emph{derivative} of the polynomial $p\in\Pol{n}$ interpolating the
data points $(t_{i},y_{i})$, $i=0,\ldots,n$, for pairwise different
$t_{i}\in\bbR$ and data values $y_{i}\in\bbR$.

\begin{hint}
Differentiate the recursion formula \lref{eq:ipolrec} and
devise an algorithm in the spirit of the Aitken-Neville algorithm
implemented in \lref{AitkenNeville}.
\end{hint}

\begin{solution}
Differentiating the recursion formula \ncseeq{eq:ipolrec} we obtain
%
\begin{equation*}
\begin{aligned}
p_i(t) &\equiv y_i, \qquad i = 0, \dots, n, \\
p'_i(t) &\equiv 0, \qquad i = 0, \dots, n, \\
p_{i_0,\dots,i_m}(t) &= \frac{ (t-t_{i_0}) p_{i_1,\dots,i_m}(t) - (t-t_{i_m}) p_{i_0,\dots,i_{m-1}}(t) }{t_{i_m}-t_{i_0}}, \\
p'_{i_0,\dots,i_m}(t) &= \frac
{ p_{i_1,\dots,i_m}(t) + (t-t_{i_0}) p'_{i_1,\dots,i_m}(t) - p_{i_0,\dots,i_{m-1}}(t) - (t-t_{i_m}) p'_{i_0,\dots,i_{m-1}}(t) } {t_{i_m}-t_{i_0}}.
\end{aligned}
\end{equation*}
%
The implementation of the above algorithm is given in file \texttt{dipoleval\_test.m}.
\end{solution}
\end{subproblem}

\begin{subproblem}[2]
  For validation purposes devise an alternative, less efficient, implementation of
  \texttt{dipoleval} (call it \texttt{dipoleval\_alt}) based on the following
  steps:
  \begin{enumerate}
  \item Use \matlab{}'s \texttt{polyfit} function to compute the monomial
    coefficients of the Lagrange interpolant.
  \item Compute the monomial coefficients of the derivative.
  \item Use \texttt{polyval} to evaluate the derivative at a number of points.
  \end{enumerate}
  Use \texttt{dipoleval\_alt} to verify the correctness of your implementation of
  \texttt{dipoleval} with \texttt{t = linspace(0,1,10)}, \texttt{y = rand(1,n)}
  and \texttt{x = linspace(0,1,100)}.

\begin{solution}
See file \texttt{dipoleval\_test.m}.

% We verify the correctness of the implementation in \autoref{mc:EvaluatingDerivatives_dipoleval_test}.
% See \autoref{fig:EvaluatingDerivatives_dipoleval_test} for the results.
% %
% \lstinputlisting[label={mc:EvaluatingDerivatives_dipoleval_test},caption={Implementation of \texttt{dipoleval\_test}}]
% {\problems/ch_polynomialinterpolation/MATLAB/dipoleval_test.m}
% %
% \begin{figure}[ht]
% \centering
% \includegraphics[width=0.8\textwidth]{\problems/ch_polynomialinterpolation/PICTURES/dipoleval_test.eps}
% \caption{Testing routine \texttt{dipoleval} by comparing to divided differences}
% \label{fig:EvaluatingDerivatives_dipoleval_test}
% \end{figure}
\end{solution}
\end{subproblem}
\end{problem}
