\begin{problem}[Piecewise linear interpolation]
  \lref{ex:pwlin} introduced piecewise linear interpolation as a simple linear
  interpolation scheme. It finds an interpolant in the space spanned by the
  so-called tent functions, which are \emph{cardinal basis functions}. Formulas
  are given in \lref{eq:pwlinbas}. 

% Recall that given a vector of nodes $t_i$, 
%   $i = 0, \dots, n$ ($t_i \neq t_j, i \neq j$), the tent functions $b_i$ satisfy:
%  \begin{itemize}
%   \item the basis is local:
%   \[
% \supp b_i = \{ x \in \IR \; b_i(x) = 0 \} = (t_{i-1}, t_{i+1});
%   \]
%   \item the basis is cardinal:
%   \[
%     b_i(t_j) = \delta_{i,j} = \begin{cases} 1 & i = j \\ 0 & i \neq j \end{cases};
%   \]
%   \item $b_i$ is piecewise linear in $[t_j, t_{j+1}], \forall j$;
%  \end{itemize}
%  We can then define the piecewise linear interpolant $\ell$ to be a linear combination of the basis functions $b_i: \IR \rightarrow \IR$:
%  \[
%   \ell = \sum_{i = 0}^n y_i b_i, c_i \in \IR
%  \]
%  we call $c_i$ the coefficients of the representation of $\ell$ in the basis
%  $\{ b_i \}$. Notice that an interpolant is well defined, if we provide values at
%  each node, i.e. if we provide the data $(t_i, y_i)$.

%  
%  \begin{subproblem}[1]
%   Think on how to represent the basis in memory.
%   
% \cprotEnv \begin{solution}
%    See \verb|linearinterpolant.cpp|.
%   \end{solution}
%  \end{subproblem}
%  
 \begin{subproblem}[3]
  Write a \Cpp{} class \verb|LinearInterpolant| representing the piecewise linear interpolant. Make sure your class has an efficient internal representation of a basis. Provide a constructor and an evaluation \verb|operator()| as described in the following template:
  \begin{lstlisting}[language=c++]
class LinearInterpolant {
    public:
        LinearInterpolant( /*  TODO: pass pairs */) {
	    // TODO: construct your data from  (t_i, y_i)'s
        }
        
        double operator() (double x) {
            // TODO: return I(x)
        }
    private:
        // Your data here
};
  \end{lstlisting}
  
  \begin{hint}
   Recall that \Cpp{} provides containers such as \verb|std::vector| and \verb|std::pair|.
  \end{hint}

\cprotEnv \begin{solution}
   See \verb|linearinterpolant.cpp|.
  \end{solution}
\end{subproblem}

 \begin{subproblem}[1]
  Test the correctness of your code.
 \end{subproblem}
 
\end{problem}
