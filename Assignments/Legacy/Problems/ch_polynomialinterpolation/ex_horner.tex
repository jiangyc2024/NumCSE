% ncse_new/p2_InterpolationApproximation/ch1_PolynomialInterpolation/ex_EvaluatingDerivatives.tex
% exercise requires:    -
% solutions require:    dipoleval.m  dipoleval_test.m  dipoleval_test.eps

\begin{problem}[Evaluating the derivatives of interpolating polynomials \coreproblem]
\label{prb:EvaluatingDerivatives}
This problem is about the Horner scheme, that is a way to efficiently evaluate a polynomial in a given point, see \lref{rem:Horner-scheme}.
\begin{subproblem}[3]
Using the Horner scheme, write an efficient C++ implementation of a function
\begin{lstlisting}
template <typename CoeffVec>
std::pair<double,double> evaldp ( const CoeffVec & c, double x )
\end{lstlisting}
which returns the pair $(p(x),p'(x))$, where $p$ is the polynomial with coefficients in \texttt{c}. The vector \texttt{c} contains the coefficient of the polynomial in the monomial basis, using Matlab convention (leading coefficient in \texttt{c[0]}).
\begin{solution}
See file \texttt{horner.cpp}.
\end{solution}
\end{subproblem}
\begin{subproblem}[2]
  For the sake of testing, write a naive C++ implementation of the above function
\begin{lstlisting}
template <typename CoeffVec>
std::pair<double,double> evaldp_naive ( const CoeffVec & c, double x )
\end{lstlisting}
which returns the same pair $(p(x),p'(x))$. This time, $p(x)$ and $p'(x)$ should be calculated with the simple sums of the monomials constituting the polynomial.
\begin{solution}
See file \texttt{horner.cpp}.
\end{solution}
\end{subproblem}
\begin{subproblem}[1]
What are the asymptotic complexities of the two implementations?
\begin{solution}
  In both cases, the algorithm requires $\approx n$ multiplications and additions,
  and so the asymptotic complexity is $O(n)$. The naive implementation also calls
  the \texttt{pow()} function, which may be costly. 
\end{solution}
\end{subproblem}
\begin{subproblem}[1]
Check the validity of the two functions and compare the runtimes for polynomials of degree up to $2^{20}-1$.
\begin{solution}
See file \texttt{horner.cpp}.
\end{solution}
\end{subproblem}
\end{problem}
