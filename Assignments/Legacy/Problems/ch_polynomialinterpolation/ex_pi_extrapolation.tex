\begin{problem}[Approximation of $\pi$]
In \lref{subsec:extrapolation} we learned about the use of polynomial extrapolation
(= interpolation outside the interval covered by the nodes) to compute
inaccessible limits $\lim\nolimits_{h\to 0}\Psi(h)$. In this problem we apply
extrapolation to obtain the limit of a sequence $x^{(n)}$ for $n\to\infty$.

We consider a quantity of interest that is defined as a limit
  \begin{gather}
    \label{eq:1}
    x^{\ast} = \lim\limits_{n\to\infty} T(n)\;,
  \end{gather}
  with a function $T:\{n,n+1,\ldots\}\mapsto\bbR$. However, computing $T(n)$ for
  very large arguments $k$ may not yield reliable results.

  The idea of \emph{extrapolation} is, firstly, to compute a few values
  $T(n_{0}),T(n_{1}),\ldots,T(n_{k})$, $k\in\bbN$, and to consider them as the values 
  $g(1/n_{0}),g(1/n_{1}),\ldots,g(1/n_{k})$ of a continuous function
  $g:]0,1/n_{\min}]\mapsto\bbR$, for which, obviously
  \begin{gather}
    \label{eq:2}
    x^{\ast} = \lim\limits_{h\to 0} g(h)\;.
  \end{gather}
  Thus we recover the usual setting for the application of polynomial
  extrapolation techniques. Secondly, according to the idea of extrapolation to
  zero, the function $g$ is approximated by an interpolating polynomial
  $p\in\Cp_{k-1}$ with $p_{k-1}(n_{j}^{-1}) = T(n_{j})$, $j=1,\ldots,k$. In many
  cases % (see \ncsesect{subsec:extrapolation} for detailed information and the
  % connection with asymptotic expansions)
  we can expect that $p_{k-1}(0)$ will
  provide a good approximation for $x^{\ast}$.  In this problem we study the
  algorithmic realization of this extrapolation idea for a simple example.

  The unit circle can be approximated by inscribed regular polygons with $n$ edges.
  The length of half of the circumference of such an $n$-edged polygon
  can be calculated by elementary geometry:
   
    \[
    \begin{array}{c|ccccccc}
      n & 2 & 3 & 4 & 5 & 6 & 8 & 10 \\
      \hline
        &  &  &  &  &  &  &  \\[-\smallskipamount]
      T(n) := \frac{U_n}{2} &\; 2 &\; \frac{3}{2}\sqrt{3} &\; 2 \sqrt{2} &\;
                                                                           \frac{5}{4}\sqrt{10-2\sqrt{5}} &\; 3 &\; 4\sqrt{2-\sqrt{2}} &\;
                                                                                                                                         \frac{5}{2} \left ( \sqrt{5}-1 \right )
    \end{array}
    \]

Write a \Cpp{} function
\begin{center}
  \texttt{double pi\_approx(int k);}
\end{center}
that uses the \emph{Aitken-Neville scheme}, see \ncseref{AitkenNeville}, to
approximate $\pi$ by extrapolation from the data in the above table, using
the first $k$ values, $k=1,\ldots,7$.


\cprotEnv \begin{solution}
 See \verb|pi_approx.cpp|.
\end{solution}


% \bigskip
%  
% We want to approximate $pi$ as half of the circumference of the unit
% circle. The circumference of the unit circle is approximated by the
% circumference of polygons fitted into the unit circle. These are given
% by geometry. Thus, we obtain the function $U(n)$, which gives us the
% circumference of the polygon with $n$ edges. We then have the
% relation, that
% \begin{equation*}
%   \lim_{n\to \infty} \frac{U(n)}{2} = \pi\ .
% \end{equation*}
% The idea is, to use interpolation, to approximate $\pi$. But for the
% function $U(n)$, we would have to evaluate the interpolation
% polynomial at $\infty$. Thus, we reformulate the problem by defining a
% function $\widetilde{U}(\frac{1}{n}) = U(n)$. Now we have
% \begin{equation*}
%   \lim_{n\to \infty} \frac{U(n)}{2} 
%   = \lim_{\frac{1}{n}\to 0} \frac{\widetilde{U}(\frac{1}{n})}{2} 
%   = \pi\ ,
% \end{equation*}
% thus, we have to interpolate $\widetilde{U}$ at zero, which is more
% reasonable than interpolating at $\infty$. Since we only need to
% interpolate at one specific point, we can use the Aitken-Neville
% scheme. This is done in \texttt{pi\_approx}.
% 
% % \lstinputlisting[caption={Matlab Code for
% %   \texttt{pi\_approx}},label={lst:pi_approx}]{pi_approx.m}
% 
% Here, we first build up the data given in the table. Then, we do a
% vectorized version of the Aitken-Neville scheme.
% 
% In the script \texttt{a1.m}, we plot the unit circle and the first few
% polygons, then approximate $\pi$ with a polynomial of degree 1 to 7,
% print out the approximation and plot the error.
% % 
% % \lstinputlisting[caption={Matlab Code for
% %   Problem 1.},label={lst:a1}]{a1.m}
% 
% The output is given below.
% 
% \begin{verbatim}
%    >> a1
%    Degree = 1, pi ~ 2.000000
%    Degree = 2, pi ~ 3.794229
%    Degree = 3, pi ~ 3.244731
%    Degree = 4, pi ~ 3.125325
%    Degree = 5, pi ~ 3.140488
%    Degree = 6, pi ~ 3.141676
%    Degree = 7, pi ~ 3.141595
% \end{verbatim}

% \begin{center}
%   \includegraphics[width=\textwidth]{a1}
% \end{center}

% We also used the function \texttt{plot\_circle}.

% \lstinputlisting[caption={Matlab Code for
%   \texttt{plot\_circle}.},label={lst:plot_circle}]{plot_circle.m}
 
\end{problem}
