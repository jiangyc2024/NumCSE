\begin{problem}[Monotonicity preserving interpolation \coreproblem]
This problem is about monotonicity preserving interpolation. Before starting, you should revise \lref{def:linip}, \lref{par:datshp} and \lref{sec:hipshp} carefully.
\begin{subproblem}[4] 
Prove \lref{thm:monlinip}:

\fbox{\parbox{0.95\textwidth}{
If, for fixed node set {$\{t_{j}\}_{j=0}^{n}$}, {$n\geq2$}, an
interpolation scheme {$\Op{I}:\bbR^{n+1}\to C^{1}(I)$} is
\com{\emph{linear}} as a mapping from data values to continuous functions on
the interval covered by the nodes ($\to$ \lref{def:linip}), and
\com{\emph{monotonicity preserving}}, then \com{$\Op{I}(\Vy)'(t_{j})=0$} for
all {$\Vy\in\bbR^{n+1}$} and {$j=1,\ldots,n-1$}.}}

\begin{hint}
Consider a suitable basis $\{\mathbf{s}^{(j)}:j=0,\dots,n\}$ of $\R^{n+1}$ that consists of monotonic vectors, namely such that $s^{(j)}_i \le s^{(j)}_{i+1}$ for every $i=0,\dots,n-1$.
\end{hint}
\begin{hint}
Exploit the phenomenon explained next to \lref{monintp}.
\end{hint}
\begin{solution}
Without loss of generality assume that $t_0 < t_1 < \dots <t_n$. For $j=0,\dots,n$ let $\mathbf{s}^{(j)}\in\R^{n+1}$ be defined by
\[
s^{(j)}_i=
\begin{cases}
0 & i=0,\dots,j-1 \\
1 & i=j,\dots,n.
\end{cases}
\]
Clearly, $(t_i,s^{(j)}_i)_i$ are monotonic increasing data, according to \lref{def:mondata}.

%In particular, $\mathbf{s}^{(0)}=(1,\dots,,1)^T$ is the constant vector, hence it is monotonic increasing and decreasing. Since $\Op{I}$ is monotonicity preserving, this implies that $\Op{I}(\mathbf{s}^{(0)})$ is both increasing and decreasing, and so it is constant. As a consequence, its derivative is constantly zero. In particular,
%\begin{equation}
%label{eq:s0}
%(\Op{I}(\mathbf{s}^{(0)}))'(t_i)=0,\quad i=1,\dots,n-1.
%\end{equation}

Take $j=0,\dots,n$. Note that $\mathbf{s}^{(j)}$ has a local extremum in $t_i$ for every $ i=1,\dots,n-1$. Thus, since $\Op{I}$ is monotonicity preserving (see \lref{par:shppres}),
 $\Op{I}(\mathbf{s}^{(j)})$ has to be flat in $t_i$ for every $ i=1,\dots,n-1$ (see \lref{sec:hipshp}). As a consequence,
\begin{equation}
\label{eq:sj}
(\Op{I}(\mathbf{s}^{(j)}))'(t_i)=0,\quad i=1,\dots,n-1.
\end{equation}

Note now that $\{\mathbf{s}^{(j)}:j=0,\dots,n\}$ is a basis for $\R^{n+1}$ (indeed, they constitute a linearly independent set with cardinality equal to the dimension of the space). As a consequence, every $\mathbf{y}\in\R^{n+1}$ can be written as a linear combination of the $\mathbf{s}^{(j)}$s, namely
\[
\mathbf{y}=\sum_{j=0}^n \alpha_j \mathbf{s}^{(j)}.
\]
Therefore, by the linearity of $\Op{I}$ and using \eqref{eq:sj}, for every $i=1,\dots,n-1$ we obtain
\[
(\Op{I}(\mathbf{y}))'(t_i)=\left(\Op{I}\left(\sum_{j=0}^n \alpha_j \mathbf{s}^{(j)}\right)\right)'(t_i) 
=\left(\sum_{j=0}^n \alpha_j \Op{I}(\mathbf{s}^{(j)})\right)'(t_i) 
=\sum_{j=0}^n \alpha_j \Op{I}(\mathbf{s}^{(j)})'(t_i) 
=0,
\]
as desired.
\end{solution}
\end{subproblem}
\end{problem}
