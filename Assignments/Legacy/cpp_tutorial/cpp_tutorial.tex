\documentclass{beamer}

\usetheme{Warsaw}
\usecolortheme{spruce}
\setbeamertemplate{itemize items}[default]
\setbeamertemplate{enumerate items}[default]

\setbeamercolor{math text}{fg=black}
\setbeamercolor{math text inlined}{parent=math text}
\setbeamercolor{math text displayed}{parent=math text}
\setbeamercolor{normal text in math text}{}

\usepackage[utf8]{inputenc}
\usepackage{default}

\usepackage{listings}

\definecolor{morange}{HTML}{e67e22}
\definecolor{mblue}{HTML}{3498db}
\definecolor{mgreen}{HTML}{1abc9c}
\definecolor{mgray}{HTML}{34495e}
\definecolor{myell}{HTML}{F1C4F0}
\definecolor{mred}{HTML}{e74c3c}

\lstdefinestyle{cpp}{
  belowcaptionskip=1\baselineskip,
  breaklines=true,
%   frame=L,
%   xleftmargin=\parindent,
  escapechar=\%,
  language=C++,
  showstringspaces=false,
  basicstyle=\footnotesize\ttfamily,
  keywordstyle=\bfseries\color{mblue},
  commentstyle=\itshape\color{mgreen!40!black},
  identifierstyle=\color{mgray},
  stringstyle=\color{morange},
  frame=single
}
\definecolor{cgreen}{HTML}{27ae60}
\newcommand{\code}[1]{{\color{cgreen}\texttt{#1}}}

\title{C++ tutorial}

\lstset{style=cpp}

\begin{document}

\begin{frame}
 \begin{center}
  \Huge C++ tutorial
 \end{center}
\end{frame}

\section{C++}

\begin{frame}[fragile,allowframebreaks]
 \frametitle{Introduction}
 I will assume that you know some basics of C++:
 \begin{lstlisting}
#include <iostream>

int main(void) {
    std::cout << "Hello world!" << std::endl;
}
 \end{lstlisting}
 \framebreak
 I hope that, by the end, you can tell what this does:
 \begin{lstlisting}
#include <iostream>
#include <vector>
template <typename T> struct hello {
    T operator*() { return "Hello"; }
    static const char e = '!';
    template <char c> char put_char() { return c; }
};
int main(void) {
    std::vector<char>  world = {'w','o','r','l','d'};
    hello<std::string> h;
    std::cout << *h << h.put_char<' '>();
    for(auto & word : world) { std::cout << word; }
    std::cout << hello<std::string>::e << "\n";
}
\end{lstlisting}
\end{frame}

\begin{frame}[fragile]
  \frametitle{Contents}
  \begin{itemize}
  \item Terminology
  \item Object Oriented Programming
  \item Template metaprogramming
  \item STL containers
  \item C++11
  \item Eigen
  \end{itemize}
\end{frame}

\begin{frame}[fragile]
 \frametitle{Some terminology}
 \begin{itemize}
  \item Keywords: reserved syntactical elements \\
  e.g. \code{template}, \code{class}, \code{return} ... 
  \item Build-in types: reserved for basic types \\
  e.g. \code{int}, \code{char}, \code{double}, ...
  \item Operators: basically functions but ``better looking'' \\
  *, +, \&, :: , ?:, ... 
 \end{itemize}
\end{frame}

\begin{frame}[fragile]
 \frametitle{Terminology: \code{lvalue} and \code{rvalue}}
 Variables can be placed in two categories. \\
 We differentiate\footnote{Actually things are a bit more complicated than this...} between a \code{lvalue} and a \code{rvalue}:
 \begin{itemize}
  \item \code{lvalue} can be seen as a physical memory object;
  \item \code{rvalue} is just a temporary value.
 \end{itemize}
 \textbf{Rule of thumb:} can be put it on the left of \code{=} ? \\
 Then \code{lvalue}, otherwise \code{rvalue}.
 \begin{lstlisting}
  int incr(int a) { return a+1; }
  int a;
  a = 9; // a is lvalue, 9 is rvalue
  9 = a; // invalid!
  incr(a) = 3; // invalid, f(a) is rvalue
  a = incr(a); // valid
 \end{lstlisting}
\end{frame}
 
\begin{frame}[fragile]
 \frametitle{Passing by value and by reference}
 Consider a generic function:
 \begin{lstlisting}
void function(%\only<3>{\color{mred}const }%MatrixXd %\only<2->{\color{mred}\& }%A);
 \end{lstlisting}
 \only<1>{All data from \code{A} copied to a new temporary. \\ We say that \code{A} is \textbf{passed by value}. \\}
 \only<2>{Now, \code{A} can be modified inside \code{function}. \\ We say that \code{A} is \textbf{passed by reference}. \\ We do not want that. \\}
 \only<3>{Now, \code{A} cannot be modified inside \code{function}. \\ No copy is performed. \\ We say that \code{A} is \textbf{passed by constant reference}. \\ Passing by const. reference is very common in C++. \\}
 \only<3>{\textbf{Remark:} not needed for built-ins (e.g. \code{int}, \code{double}, ...). \\}
 \only<3>{\textbf{Remark:} can also return by reference, but it is dangerous.}
\end{frame}
 
\begin{frame}[fragile, allowframebreaks]
 \frametitle{Function and operator overload}
  Let's say that I have functions:
  \begin{lstlisting}
int Id(int a) { std::cout << "int"; return a; }
int Id(double a) { std::cout << "double"; return a; }
int Id() { std::cout << "void"; return 1; }
  \end{lstlisting}
  This is valid in C++ and called \textbf{function overloading}. \\
  \begin{lstlisting}
Id(1); // prints "int"
Id(1.); // prints "double"
Id(); // prints "void"
  \end{lstlisting}
  Compiler must be able to pick the ``best fit'': \\
  \begin{lstlisting}
// error: ambiguating new declaration:
double Id() { return 1; } 
double Id(int a = 1) { return a; }
  \end{lstlisting}
  \framebreak
  Most commonly used with operators. \\
  Assume \code{Complex} is a type for complex numbers\footnote{Here define \code{Complex} to be \code{std::complex<double>}, cf. later.}:
  \begin{lstlisting}
bool operator<(Complex a, Complex b) {
  return a.real() < b.real();
}
  \end{lstlisting}
  overloads a lot of built-in \code{operator<} (undefined for $\mathbb{C}$).
  \begin{lstlisting}
Complex(3,4) < Complex(3,4); // now makes sense
  \end{lstlisting}
  Could also define a weird \code{operator<}:
  \begin{lstlisting}
   std::string operator<(Complex a, bool b);
  \end{lstlisting}
 $\Rightarrow$ completely new syntactical constructs.
\end{frame}
 
\begin{frame}[fragile]
 \frametitle{Pointers}
 \emph{Pointers} refer to a specific memory location:
 \begin{lstlisting}
  int * a; // ptr. to int, undefined memory location
  int * b = new int; // ptr. to allocated int
  a = b; // a points to the same as b
  *a = 9; // value at the memory pointed by a is 9
  std::cout << *b; // *b = 9
  int c;
  a = &c; // a points to the address of c
  delete b;
 \end{lstlisting}
 Four important operators:
 \begin{itemize}
  \item \code{*} (unary) return value of pointer (dereference);
  \item \code{\&} (unary) return address of variable (reference);
  \item \code{new} and \code{delete} allocate/deallocate memory for pointers.
 \end{itemize}
 \textbf{Rule:} for each \code{new} there must be a \code{delete}. \\
 \textbf{Remark:} use pointers with extreme care.
\end{frame}

\begin{frame}[fragile]
 \frametitle{Casting}
 Why does this make sense?
 \begin{lstlisting}
  double i = 9; // <=> double i = (double) 9.
 \end{lstlisting}
 The \code{int 9} is \emph{casted} implicitly to a \code{double}. \\
 Most of the times you do not have to worry about it. \\
 If you \textbf{really need} you can use the operators: \\
  \code{static\_cast}, \code{dynamic\_cast}, \code{reinterpret\_cast}, \code{const\_cast}.

\end{frame}

\begin{frame}[fragile]
 \frametitle{Namespaces}
 \emph{Namespaces} are like boxes with labels. \\
 \begin{itemize}
  \item Access with \code{::} \emph{operator}
  \item or import them with the \emph{keyword} \code{using}.
 \end{itemize}
 \begin{lstlisting}
namespace MySpace {
  int i = 9;
}
std::cout << i; // error: 'i' was not declared ...
std::cout << MySpace::i << "\n"; // ok
using namespace MySpace;
std::cout << i << "\n"; // now it works ok
int i = 7;
std::cout << i << "\n"; // i = 7 here
 \end{lstlisting}
 \code{Eigen} and \code{std} are examples of namespaces. \\
%  \textbf{Remark:} what if I define \code{int cout;}? \\
 \textbf{Terminology:} if it contains \code{::} is called \code{qualified}.
\end{frame}

\begin{frame}[fragile]
 \frametitle{Keyword: \code{typedef}}
 This keyword allows to rename types for convenience:
 \begin{lstlisting}
  typedef std::vector<double> custom_vector;
  custom_vector cv; // cv is std::vector<double>
 \end{lstlisting}
 or, with \code{using} (in C++11):
 \begin{lstlisting}
  using custom_vector = std::vector<double> ;
  custom_vector cv1; // cv is std::vector<double>
 \end{lstlisting}
 \code{using} can be used with templates.
%  we'll see that, in C++11, we can avoid this using \code{auto}.
\end{frame}

\subsection{OOP}

\begin{frame}[fragile]
 \frametitle{OOP: \code{struct} and \code{class} in short}
 Both \code{struct} and \code{class} keywords to define an \emph{Object}. \\
 Mainly, an Object is a collection of:
 \begin{itemize}
  \item other Objects or types (called members);
  \item functions (methods).
 \end{itemize}
 \begin{lstlisting}
struct myStruct {
    int     some_int;
    double  some_double;
    void    incr() { some_int++; } // function
};
class myClass {
    int        I;
    myStruct   S; // contains a struct
};
 \end{lstlisting}
\end{frame}

\begin{frame}[fragile]
 \frametitle{OOP: constructors}
 Constructors: called to build an Object
 \begin{lstlisting}
class myClass {
  public:
    myClass() { I = 9; }; // default constructor
    myClass(int I_) : I(I_) {} // any constructor
    myClass(const myClass & other) // copy constructor
      : I(other.I) {} // special syntax I = other.I
    int        I;
};
  \end{lstlisting}
  The \code{: I(I\_)} is like writing \code{I = I\_;} in the first line. Usage:
 \begin{lstlisting}
myClass m; // default construct
myClass n(1); // construct with ints
myClass o(n); // copy m to o
myClass p(); // NOT what you think
int a(); // function declaration like this
 \end{lstlisting}
\end{frame}

\begin{frame}[fragile]
 \frametitle{OOP: operators}
 Operators: a fancier way to define special function
  \begin{lstlisting}
class myClass {
  int operator*(const myClass & other) const {
    return 5*I + other.I;
  }
  int weird_mult(const myClass & other) const {
    return 5*I + other.I;
  }
}
  \end{lstlisting}
  Calling  sequence:
 \begin{lstlisting}
myClass m, n;
int ret = m * n; // a custom uncommon operator
int ret2 = m.weird_mult(n); // is exactly the same
 \end{lstlisting}
\end{frame}

\begin{frame}[fragile]
 \frametitle{OOP Keyword: \code{this}}
 Inside a class, \code{this} is a pointer to the class itself.
 \begin{lstlisting}
struct T {
  int i = 9; // syntax sugar to initialize i = 9
  T & operator=(const T & other) {
    this->i = other.i; // this not required (implicit)
    // this->i <=> *this.
    return *this;
  }
};
 \end{lstlisting}
 Then I can write:
 \begin{lstlisting}
  T e1, e2, e3;
  e1 = e2 = e3; // because (e2 = e3) returns e2
 \end{lstlisting}

\end{frame}

\begin{frame}[fragile]
 \frametitle{OOP: class Namespaces}
 Each \code{class} and \code{struct} comes with a ``free'' namespace. \\
 Functions, other objects and statics members are included in it.
 \begin{lstlisting}
struct N {
  class N_c { // nested class
    // stuff ...
  };
  typedef double N_t;
  static int N_i;
  void N_f(); // function declaration
};
 \end{lstlisting}
 They are accessed with \code{::}:
 \begin{lstlisting}
N::N_c n_c; // <=> n_c is a class N_c
N::N_t n_d; // <=> double N_d
N::N_i = 9; // variable of type int
void N::N_f() { ++N_i; } // definition of function N_f
 \end{lstlisting}
\end{frame}

\begin{frame}[fragile]
 \frametitle{OOP: \code{const}, some special syntax}
 A \code{const} member function promises not to modify the class:
 \begin{lstlisting}
struct myClass {
    int nonconst_function() {
	// something
	I = 8; // fine
    }
    int const_function() const {
	// something
	I = 8; // is illegal: modifies myClass
    }
};
 \end{lstlisting}
 Then you can use this function also on constant classes:
 \begin{lstlisting}
const myClass mc;
mc.nonconst_function(); // illegal
mc.const_function(); // legal
 \end{lstlisting}
\end{frame}
 
\begin{frame}[fragile]
 \frametitle{OOP: \code{static} members}
 A \code{static} member is shared amongst all instances:
 \begin{lstlisting}
class myClass {
  static int static_I;
  void incr() { static_I++; }
}
 \end{lstlisting}
 A \code{static} member is linked more to \code{myClass} than to \code{m} or \code{n}. \\
 No need for an instance of \code{myClass} to access \code{static\_I}.
 \begin{lstlisting}
int myClass::static_I = 9;
int main() {
    myClass m,n;
    std::cout << m.static_I << n.static_I; // 9  9
    m.incr();
    std::cout << m.static_I << n.static_I; // 10 10
}
 \end{lstlisting}
\end{frame}

\begin{frame}[fragile]
 \frametitle{OOP: \code{static} functions}
 A \code{static} member function is shared amongst all instances:
 \begin{lstlisting}
class myClass {
  static int function() {
      int a = static_I; // valid
      int b = I; // is illegal: cannot see non static
      return 8;
  }
}
 \end{lstlisting}
 No need for an instance of \code{myClass} to use \code{function}:
 \begin{lstlisting}
int y = myClass::function();
 \end{lstlisting}
\end{frame}
  
\begin{frame}[fragile]
 \frametitle{OOP: member access: \code{private}, \code{protected} and \code{public}}
 Until now, we never worried about the ``privacy'' of class members. \\
 Each member of an Object has a ``privacy setting'':
 \begin{lstlisting}
class C {
  // Default: private, accessed only within class
  int a; // C.a invalid outside
public: // can be accessed only by derived class
  int b; // C.b allowed only in child
protected: // can be accessed by anyone
  int c; // C.c always valid
}
 \end{lstlisting}
 Finally, we can tell the difference between \code{struct} and \code{class}:
 \begin{itemize}
  \item \code{struct}: members are \code{public} by default;
  \item \code{class}: members are \code{private} by default.
 \end{itemize}
\end{frame}

\subsection{Template metaprogramming}

\begin{frame}[fragile]
 \frametitle{Template metaprogramming}
 The idea behind templates is: let the compiler write code for you. \\
 \textbf{Idea:} write function and classes with generic types. \\
 Templates are resolved at compile time
\end{frame}

\begin{frame}[fragile]
 \frametitle{Function templates}
 I want a function:
 \begin{lstlisting}
double AXPY(double X, double Y, double A)
  { return A*X+Y; }
 \end{lstlisting}
 What if I want the same for \code{int}? \code{complex}? \code{MatrixXd}? \\
 Define a ``skeleton'' function:
 \begin{lstlisting}
template <class whatever>
whatever AXPY(const whatever & X,
	      const whatever & Y, double A)
	      { return A*X+Y; }
 \end{lstlisting}
 as long as \code{*} and \code{+} are well defined for \code{whatever}. \\
 The compiler defines the functions for us. \\
%  \textbf{Remark:} we pass by reference for efficiency (cf. prev. sect.). \\
 \textbf{Remark:} \code{template <typename whatever>} exactly the same.
\end{frame}

\begin{frame}[fragile,allowframebreaks]
 \frametitle{Function templates: example}
 We define the templated function \code{myMin}: \\
 taking 2 arguments and returning the minimum.
 \begin{lstlisting}
template <class T1, class T2> // s.t. can mix T1/T2
T1 myMin(const T1 & a, const T2 & b) {
  return a < b ? a : b; // ternary operator
}

int a = myMin(8.,9.); // T1 = T2 = double
double b = myMin(8,9); // T1 = T2 = int
double c = myMin(8,9.); // T1 = int, T2 = double
// a = b = c = 8
\end{lstlisting}
\textbf{Remark:} \code{myMin} uses \code{operator<} from \code{int} and \code{double}. \\
\textbf{Remark:} operator \code{expr1 ? expr2 : expr3} returns: \\
\code{expr2} if \code{expr1} otherwise returns \code{expr3}
\framebreak

This way, it can be used with custom types:
 \begin{lstlisting}
struct S {
  int i;
  S(int i) : i(i) { }; // constructor
  operator int() const { return i; } // S to int
  bool operator<(const S & other) const {
    return i > other.i; // swap < for >
  }
};

// Uses operator< for S, which is actually >
double d = myMin(S(8),S(9)); // d = 9
\end{lstlisting}
\end{frame}
 
\begin{frame}[fragile]
 \frametitle{Class template}
 Just as with functions, we can template any class:
 \begin{lstlisting}
template <class T>
struct myComplex {
  T Re, Im; // real and imaginary part
  
  myComplex(T Re, T Im) : Re(Re), Im(Im) {};
};
 \end{lstlisting}
  Which has to be instantiated by specifying the type:
 \begin{lstlisting}
myComplex<int>  cmplx1(8,9); // cmplx = 8*i+9
myComplex<double>  cmplx2(2.3,3); // cmplx = 2.3*i+3
 \end{lstlisting}
\end{frame}

\begin{frame}[fragile]
 \frametitle{Class template}
 You can then go crazy and template everything:
 \begin{lstlisting}
template <class T1, class T2 = int> // T2 default int
class myComplex2 {
  T1 Re; // real part has type T1
  T2 Im; // imaginary part part has type T2
public:
  template <class U>
  U sum() { return Re + Im; }
};
 \end{lstlisting}
 Usage involve lots of \code{<>}:
 \begin{lstlisting}
myComplex2<double> cmplx3; // Re = double, Im = int
myComplex2<double, double> cmplx4; // Re = Im = double

cmplx3.sum<int>(); // must specify U above
double s1 = cmplx3.sum<double>(); // must specify U
 \end{lstlisting}
\end{frame}
 
\begin{frame}[fragile]
 \frametitle{Keyword: \code{typename} alternative meaning}
 \begin{lstlisting}
template <class T>
void templated_function() {
   %\only<4>{{\color{mred}typename }}%T::t * x; // %\only<1>{multiplication or pointer def.?}\only<2-3>{C++: multiplication}\only<4>{now is a pointer def.} \only<3>{{\color{mred}illegal!}}%
}
 \end{lstlisting}
 \only<1>{What does this mean? \\}
 \only<1>{Is \code{T::t} a type or a (static) variable? \\}
 \only<2>{C++ interprets \code{T::t} and \code{x} as variables, \code{*} as multiplication, as if:}
 
\defverbatim{\Lst}{
 \begin{lstlisting}
struct T {
  static int t; // t is a variable
};
 \end{lstlisting}
 }
 \only<2>{\Lst}
 \only<3>{But what if we meant \code{x} to be a pointer of type \code{T::t} instead?}
 \only<4>{Keyword \code{typename} declares that a qualified name is a type.}
\defverbatim{\Lst}{
 \begin{lstlisting}
struct T {
  class t { // t is a class name
    // stuff...
  };
};
 \end{lstlisting}
 }
 \only<3-4>{\Lst}
\end{frame}
 
\begin{frame}[fragile]
 \frametitle{Keyword: \code{template} alternative meaning}
 \only<1>{Besides the normal use \code{template <class T> ...}, there is a uncommon usage of the keyword template:}
 \only<2>{You must tell the compiler that \code{myMemberFunc} is a template function and not something to compare $<$ with \code{int}?}
 \begin{lstlisting}
template <typename T>
struct tStruct {
    template <typename U>
    U tMemberFunc(void) { return U(); }
};

template <typename T>
void tFunc() {
    T s;
    int a = %\only<1>{s.}\only<2>{{\color{mred}s.template }}%tMemberFunc<int>(); // %\only<1>{error}\only<2>{OK}%
}

int main() {
    tFunc<tStruct<int>>();
}
 \end{lstlisting}
\end{frame}

\begin{frame}[fragile]
\frametitle{Recap on templates}
 Templates are very useful:
 \begin{itemize}
  \item allow focusing on algorithms rather than types;
  \item allow writing efficient programs;
  \item reduce code duplication.
 \end{itemize}
 However, when using templates:
 \begin{itemize}
  \item your program may become unreadable;
  \item you could get very complicated errors;
  \item in general, it takes longer to compile.
 \end{itemize}
\end{frame}

\subsection{STL}
 
\begin{frame}[fragile]
 \frametitle{STL containers}
 \textbf{Containers} let you store collections of elements (objects). \\
 All have different underlying data structure and complexities.
 \begin{center}
 \begin{tabular}{l|lccc}
  \hline
  name	 & descr.		& push back 	& iterator insert	& lookup \\
  \hline
  vector & dyn. array 		& O(1)\footnote{amortized amongst many \code{push\_back}.}	      	& O(n)		& O(1) \\
  list	 & linked list 		& O(1)		& O(1)		& O(n) \\
  map	 & hash map		& N/A		& O(log(n))	& O(log(n))
 \end{tabular}
 \end{center}
 For instance: 
 \begin{itemize}
  \item vector\footnote{\code{std::vector} $\neq$ \code{Eigen::VectorXd}.} is good for lookup;
  \item list is good for inserts;
  \item maps allows different type of indices.
 \end{itemize}
 $\Rightarrow$ choose the appropriate container for your needs
\end{frame}

\begin{frame}[fragile]
 \frametitle{Example: \code{std::vector}}
 A \code{std::vector} is a dynamic array like
 \begin{itemize}
  \item a C like \code{array} with variable size or
  \item a Java \code{ArrayList}
 \end{itemize}
 Needs (at least) a template argument: type of item contained. \\
 Usage:
 \begin{lstlisting}
  #include <vector>
  #include <complex>
  std::vector<int> vec; // int vec., similar to int *
  std::vector< std::complex<double> > cmplx_vec;
 \end{lstlisting}

 Guarantees data is sequential in memory (i.e. array), hence:
 \begin{itemize}
  \item changing size is expensive (reallocate everything);
  \item looking up an element is cheap.
 \end{itemize}
\end{frame}

\begin{frame}[fragile,allowframebreaks]
 \frametitle{Iterators}
 \textbf{Iterators} are something that facilitates the iteration over a loop. \\
 A container \code{c} of type \code{C} has:
 \begin{itemize}
  \item an iterator type \code{C::iterator}, pointing at an element of \code{C}:
  \begin{itemize}
   \item \code{*it} return the element pointed by \code{it};
   \item \code{++it} return the iterator pointing to the next element of \code{c};
  \end{itemize}
  \item a \code{begin()} and an \code{end()} (both are interators \code{C::iterator}).
 \end{itemize}
 Instead of:
 \begin{lstlisting}
std::vector<int> v;
// ... fill v
for(int i = 0; i < v.size(); ++i) {
    v.at(i) = i+1;
    // loop over all v
}
 \end{lstlisting}
 Used to loop easily (and efficiently):
 \begin{lstlisting}
for(std::vector<int>::iterator it = v.begin(); it  != v.end(); ++it) {
    *it = i+1;
    // loop over all v, *it is like v.at(i)
}
// range based alternative:
for(int & c: v) {
    c = i+1;
    // loop over all v, c is like v.at(i)
}
 \end{lstlisting}

\end{frame}

\subsection{C++11 features}

\begin{frame}[fragile]
 \frametitle{Keyword: \code{decltype} (C++11)}
 Copy and paste the type of a variable:
 \begin{lstlisting}
template <typename T>
struct declClass {
    T val;
};

declClass<int> m_i;
declClass<double> m_d;
decltype(m_i.val) i_type; // <=> int i_type;
decltype(m_d.val) d_type; // <=> double d_type;
\end{lstlisting}
Useful to infer the type of a variable automatically.
\end{frame}

\begin{frame}[fragile]
\frametitle{Keyword: \code{auto} (C++11)}
 In short: avoid specifying the type (let the \textbf{compiler decide})
\begin{lstlisting}
std::vector< Eigen::MatrixXd > some_function(int a);
\end{lstlisting}
\only<1>{instead of calling:}\only<2>{just call:}
\begin{lstlisting}
%\only<1>\only<2>{{\color{mred}auto}}% v = some_function(8);
for(%\only<1>\only<2>{{\color{mred}auto}}% it = v.begin();
    it < v.end(); ++it) {
  // some code
}
\end{lstlisting}
 \textbf{Remark:} \code{auto} cannot be used everywhere. \\
 Very useful, but do not abuse this feature.
\end{frame}


\begin{frame}[fragile]
\frametitle{Lambda functions (C++11)}
 Lambda functions are syntax sugar for C++. \\
 Are unnamed functions similar to Matlab function handles. \\
 Syntax: \\
 \begin{lstlisting}
[] (arguments) { function_body; }
 \end{lstlisting}
 return value is deducted by the compiler. If it can't:
 \begin{lstlisting}
[] (arguments) -> return_type { function_body; }
 \end{lstlisting}
 You can also store the lambda function in a variable:
\begin{lstlisting}
#include <functional>
auto f = [] (int i) { return i++; }; // using auto
std::function<int (int)> g = [] (int i)
	{ return i++; }; // manual type
\end{lstlisting}
\end{frame}

\begin{frame}[fragile]
\frametitle{Lambda functions (C++11): variable capture}
 Lambda functions can capture the outer scope:
 \begin{lstlisting}
int j = 9, i = 8;
[] () { std::cout << j; }; // illegal!
// legal, j captured by reference:
auto f = [&] () { std::cout << j; };
f();
[i,&j] () { i++; j++ }; // illegal for i, legal for j
 \end{lstlisting}
 Inside \code{[]} you can put many ``capture modes'':
 \begin{itemize}
  \item \code{[]}: doesn't perform any capture;
  \item \code{[\&]}: captures any variable by reference;
  \item \code{[=]}: captures any variable by making a copy;
  \item \code{[a,\&b]}: captures \code{a} by copy, \code{b} by reference;
  \item \code{[this]} captures pointer to \code{this};
 \end{itemize}
\end{frame}
 
\section{Eigen}

\begin{frame}[fragile]
\frametitle{Aliasing}

What is wrong with the following code?
\begin{lstlisting}
    Eigen::VectorXd A(4);
    A << 1,2,3,4;
    %\only<1>{A.tail(2) = A.head(2);}\only<2>{A[3] = A[2]; A[1] = A[0]; A[2] = A[1]; }%
    std::cout << A;
\end{lstlisting}
Should print:
\begin{verbatim}
 1,1,2,3
\end{verbatim}
instead prints
\begin{verbatim}
 1,1,1,3
\end{verbatim}
\only<2>{\textbf{Answer:} \emph{aliasing}, Eigen overrides the values before it used them.}
Sometimes Eigen warns you about that, but not always:
\begin{lstlisting}
 A = A.transpose(); // aborts during execution
\end{lstlisting}
\end{frame}

\begin{frame}[fragile]
\frametitle{Alignment issues in Eigen}
 A word of notice if you use \textbf{fixed size} Eigen Matrix or Vectors inside:
 \begin{itemize}
  \item structs;
  \item STL containers.
 \end{itemize}
 Alignment issues in Eigen mean that you could get \textbf{wrong} code. \\
 \textbf{Read:} \verb|http://eigen.tuxfamily.org/dox/|
 \verb|group__DenseMatrixManipulation__Alignement.html|.
\end{frame}

\section{Debugging your code}

\begin{frame}[fragile]
\frametitle{Debugging C++}
 To debug your cose you can use two tools:
 \begin{itemize}
  \item \code{std::cout} is always fine;
  \item \code{gdb} (or any other debugger): run a program with
  \begin{lstlisting}[language=bash]
   gdb ./myprogram
  \end{lstlisting}
  and you have step by step execution.
 \end{itemize}
 \textbf{Remark:} you should always test your code for correctness.
\end{frame}


\section{Efficiency}

\begin{frame}[fragile,allowframebreaks]
 \frametitle{Efficient loops}
 What is the difference between:
\begin{lstlisting}
Eigen::MatrixXd A(n,n); // A is Col. Major format
for(int i = 0; i < n; ++i) {
  for(int j = 0; j < n; ++j) {
    A(i,j) = i*j;
  }
}
\end{lstlisting}
and
\begin{lstlisting}
for(int j = 0; j < n; ++j) {
  for(int i = 0; i < n; ++i) {
    A(i,j) = i*j;
  }
}
\end{lstlisting}
\framebreak
The second: it is \textbf{not jumping} all over the memory. \\
i.e. it has better memory locality \\
$\Rightarrow$ better use of the cache \\
$\Rightarrow$ faster execution


\textbf{Rule:} fastest index should loop over the closest memory locations.
\end{frame}

\begin{frame}[fragile]
 \frametitle{Final remarks}
 Some final remark:
 \begin{itemize}
  \item The compiler is smarter than you think.
  \item Most of the time is also smarter than you.
  \item Think of cleanliness and correctness before efficiency.
  \item Do not avoid for loops: they are not evil as in Matlab.
 \end{itemize}
\end{frame}

\begin{frame}[fragile]
 \frametitle{Further help}
 \begin{itemize}
  \item Eigen doc: \\
  http://eigen.tuxfamily.org/
  \item Matlab/Eigen dictionary: \\
  http://eigen.tuxfamily.org/dox/AsciiQuickReference.txt
  \item C++ reference:
  \begin{itemize}
   \item http://www.cplusplus.com/
   \item http://en.cppreference.com/
  \end{itemize}
 \end{itemize}
\end{frame}

\end{document}
